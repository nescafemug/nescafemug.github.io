% \iffalse meta-comment
%<*internal>
\begingroup
\input docstrip
\preamble

 Copyright 2009-2016 by Ivan Valbusa

 This program is provided under the terms of the
 LaTeX Project Public License distributed from CTAN
 archives in directory macros/latex/base/lppl.txt.

 Author: Ivan Valbusa
         ivan dot valbusa at univr dot it

 This work has the LPPL maintenance status "author-maintained".

\endpreamble

\keepsilent
\askforoverwritefalse

\Msg{*** Generating the class file ***}
\generate{\file{suftesi.cls}{\from{suftesi.dtx}{class}}
   \nopreamble\nopostamble
   \file{suftesi.bib}{\from{suftesi.dtx}{bib}}
 }
 
\Msg{***********************************************************}
\Msg{*}
\Msg{* To finish the installation you have to move the following}
\Msg{* files into a directory searched by TeX:}
\Msg{*}
\Msg{* \space\space suftesi.cls}
\Msg{*}
\Msg{*}
\Msg{* To produce the documentation on suftesi's code run}
\Msg{* the file ending with `.dtx' through (pdf)LaTeX. See the}
\Msg{* README file for more details.}
\Msg{*}
\Msg{* Happy TeXing}
\Msg{***********************************************************}

\endgroup
%</internal>
%
% Copyright (C) 2009-2016 by Ivan Valbusa 
% <ivan dot valbusa at univr dot it>
% -------------------------------------------------------
% 
% This work may be distributed and/or modified under the
% conditions of the LaTeX Project Public License, either version 1.3
% of this license or (at your option) any later version.
% The latest version of this license is in
%   http://www.latex-project.org/lppl.txt
% and version 1.3 or later is part of all distributions of LaTeX
% version 2005/12/01 or later.
%
% This work consists of all files listed in README
%
% \fi
%
% \iffalse
%<*driver>
\ProvidesFile{suftesi.dtx}
%</driver>
%<class>\NeedsTeXFormat{LaTeX2e}[2005/12/01]
%<class>\ProvidesClass{suftesi}
%<*class>
    [2016/04/04 v2.9 A class for typesetting theses, books and articles]
%</class>
%<*driver>
\documentclass{ltxdoc}

\usepackage[T1]{fontenc}
\usepackage[utf8]{inputenc}
\usepackage[greek.ancient,english]{babel}
\usepackage[final]{microtype}
\usepackage{siunitx}
% To use the cochineal inside the document we have to define the family
% because the .fd files of the font refer to conditionals 
% defined in 'cochineal.sty’:
\DeclareFontFamily{T1}{Cochineal-LF}{}
\DeclareFontShape{T1}{Cochineal-LF}{m}{n}{
      <-> s*[1.0] Cochineal-Roman-osf-t1}{}
% Doc facilities
\let\cs\relax
\let\cmd\relax
\usepackage{ltxdockit}
% Graphics
\usepackage[svgnames]{xcolor}
  \definecolor{sufred}{rgb}{0.5,0,0}
  \definecolor{sufgray}{rgb}{0.5,0.5,0.5}
\usepackage[framemethod=TikZ]{mdframed}
  \mdfsetup{roundcorner=3pt,linecolor=white,backgroundcolor=gray!10}
\usepackage{tikz}
  \usetikzlibrary{shadows}
\usepackage{afterpage}
% Tables
\usepackage{array}
\usepackage{booktabs}
\usepackage{multirow}
%
\usepackage{fancyhdr}
\fancyhf{}
\fancyfoot[C]{\iffloatpage{}{\thepage}}
\renewcommand\headrulewidth{0pt}
\pagestyle{fancy}
% Bibliography
\usepackage[autostyle]{csquotes}
\usepackage[style=philosophy-classic,backend=biber]{biblatex}
  \addbibresource{suftesi.bib}
\usepackage{metalogo}
\usepackage{guit}
%%% Greek examples
\def\latintxt{Aliquam auctor, pede consequat
  laoreet varius, eros tellus maris quam, pellentesque hendrerit.}  
\def\latintxtb{Morbi luctus, wisi viverra faucibus pretium,
  nibh est placerat.}
\def\greektxt{Πρῶτον εἰπεῖν περὶ τί καὶ τίνος ἐστὶν ἡ σκέψις, ὅτι περὶ 
ἀπόδειξιν καὶ ἐπιστήμης ἀποδεικτικῆς· εἶτα διορίσαι τί 
ἐστι πρότασις καὶ τί ὅρος καὶ τί συλλογισμός}    
\newcommand{\greekexample}[4]{%
	\noindent\paragraph{#3}{\fontfamily{#1}\selectfont\latintxt}% 
		{ \fontfamily{#2}\fontsize{#4}{13}\selectfont%
		\textgreek{\greektxt}} {\fontfamily{#1}\selectfont\latintxtb}}
% New commands
\newcommand{\argstyle}{\itshape}
\DeclareRobustCommand*{\ar}[1]{\texttt{\char`\{}\textrm{\argstyle#1}\texttt{\char`\}}}
\DeclareRobustCommand*{\oar}[1]{\texttt{[}\textrm{\argstyle#1}\texttt{]}}
\DeclareRobustCommand*{\meta}[1]{%
  $\langle${\argstyle\rmfamily#1\kern0.12em}$\rangle$}
\DeclareRobustCommand*{\arm}[1]{\ar{\meta{\argstyle#1}}}
\DeclareRobustCommand*{\oarm}[1]{\oar{\meta{\argstyle#1}}}
% New environments
\newenvironment{ttquote}
  {\begin{mdframed}[default]
  \ttfamily\microtypesetup{activate=false}}
  {\end{mdframed}
  }
\newenvironment{latexcode}
  {\begin{mdframed}[default]
  }
  {\end{mdframed}}
% Correct (?) \optitem to use \meta inside #2
\makeatletter
\renewcommand*{\ltd@ol@optitem}[3][]{%
  \ifblank{#1}
    {\ltd@option{}{#2}{#3}{}}
    {\ltd@option{}{#2}{#3}{\ltd@textverb{#1}}}}%
\makeatother
% hyperref setup
\usepackage{hyperref}
\hypersetup{%
    pdftitle={User's Guide to \textsf{suftesi}},
    pdfsubject={A document class for typesetting theses, 
      books and articles},
    pdfauthor={Ivan Valbusa},
    pdfkeywords={thesis, humanities, books}}   
% Informations
\author{Ivan Valbusa\thanks{Dipartimento di Scienze Umane, Università degli Studi di Verona --- 
	\texttt{ivan dot valbusa at univr dot it}}}
\title{\vspace*{-\baselineskip}User's Guide to \textsf{suftesi}\\
	\large A document class for typesetting\\ theses,  books and articles}
\date{\fileversion{} -- \filedate}

\EnableCrossrefs         
\CodelineIndex
\RecordChanges

\begin{document}
  \DocInput{suftesi.dtx}
\end{document}
%
%</driver>
% \fi
%
% \CheckSum{4650}
%
% \CharacterTable
%  {Upper-case    \A\B\C\D\E\F\G\H\I\J\K\L\M\N\O\P\Q\R\S\T\U\V\W\X\Y\Z
%   Lower-case    \a\b\c\d\e\f\g\h\i\j\k\l\m\n\o\p\q\r\s\t\u\v\w\x\y\z
%   Digits        \0\1\2\3\4\5\6\7\8\9
%   Exclamation   \!     Double quote  \"     Hash (number) \#
%   Dollar        \$     Percent       \%     Ampersand     \&
%   Acute accent  \'     Left paren    \(     Right paren   \)
%   Asterisk      \*     Plus          \+     Comma         \,
%   Minus         \-     Point         \.     Solidus       \/
%   Colon         \:     Semicolon     \;     Less than     \<
%   Equals        \=     Greater than  \>     Question mark \?
%   Commercial at \@     Left bracket  \[     Backslash     \\
%   Right bracket \]     Circumflex    \^     Underscore    \_
%   Grave accent  \`     Left brace    \{     Vertical bar  \|
%   Right brace   \}     Tilde         \~}
%
% 
% \changes{v2.9}{2016/04/04}{New options \opt{tocauthorfont} and \opt{toctitlefont}. Added macros\cmd{suftesi@MakeTextTOCLowercase} and \cmd{SUF@TOCtitlesmallcaps}. Improved \cmd{xheadbreak} command. New templates \texttt{book}, \texttt{theses-template-article}, \texttt{theses-template-book}.}
% \changes{v2.8}{2016/03/30}{\sty{textcase} no longer loaded. Changed definitions of \cmd{suftesi@MakeTextLowercase}, \cmd{SUF@titlesmallcaps}, \cmd{headbreak}, \cmd{xheadbreak}.}
% \changes{v2.7.1}{2016/03/22}{Corrected a bug in v.2.7 (missing some \cmd{fi} commands).}
% \changes{v2.7}{2016/03/22}{New value \opt{cscreen} for \opt{version} option. Updated documentation. Removed value \opt{elements} for options:\opt{pagelayout}, \opt{headerstyle}, \opt{captionstyle}, \opt{chapstyle}, \opt{style}; \opt{sufelements} for  option: \opt{style}. The \cmd{chapnumfont} command is no longer available.}
% \changes{v2.6}{2016/03/13}{New \opt{mathfont} option for non-standard fonts. New command \cmd{makecover} for printing the cover page. Updated documentation.}
% \changes{v2.5}{2016/03/02}{Corrected a bug when loading CB Greek fonts. Renamed option \opt{documentstructure} to \opt{structure}.  Updated documentation.}
% \changes{v2.4}{2016/02/27}{Changed the default font to Cochineal with Linux Biolinum O sans serif and Inconsolata monospaced. The Palatino is now loaded with \sty{newpxtext} and \sty{newpxmath}. Bera Mono and Iwona have been substituted by Linux Biolinum O and Inconsolata. New values for font options: \opt{cochineal}, \opt{libertine}, \opt{bodoni} (greek only). Value \opt{compatibility} for \opt{defaultfont} option to get the fonts of suftesi v2.3 or previous. Updated documentation.}
% \changes{v2.3}{2015/09/19}{Maintenance release.}
% \changes{v2.2}{2015/09/10}{Maintenance release.}
% \changes{v2.1}{2015/06/13}{New option \opt{toc\meta{level}font}. Updated documentation. Removed \cmd{includeart} command}
% \changes{v2.0.1}{2015/03/31}{Corrected a bug in \opt{version} option.}
% \changes{v2.0}{2015/03/29}{New page layout \opt{standardaureo}. Enhanced support for \LuaLaTeX. Headers settings are now executed \cmd{AtBeginDocument}. Loaded \sty{ifluatex} and \sty{ifthen} packages. Added \opt{listparindent} to \cmd{setlist} options. Updated documentation.}
% \changes{v1.9c}{2014/04/05}{The \opt{cbgreek} value for the \opt{greekfont} option allows now to use the full set of the CB Greek font together with the default font (Palatino).}
%  \changes{v1.9b}{2014/02/16}{Corrected a bug in ``toc'', ``lof'' and ``lot'' commands.}
%  \changes{v1.9a}{2014/02/11}{Maintenance release. Updated greek support. Updated documentation.}
% \changes{v1.9}{2013/09/21}{Added \opt{italic}, \opt{smallcaps} and \opt{sanserif} values for \opt{captionstyle} option. New \opt{twocolcontents} option. New \opt{supercompactaureo} page layout. Changed \cmd{toclabelspace} command and renamed to \cmd{toclabelwidth}. New \opt{collection} document structure}
% \changes{v1.8}{2013/07/18}{Maintenance release. Corrected a bug in the definition of \cmd{appendix}}
% \changes{v1.7}{2013/06/20}{Renamed \opt{viewmode} option to \opt{version}:  
% renamed \opt{print} value to \opt{draft} and 
% added \opt{final} value. New \opt{smallcapsstyle} option.}
% \changes{v1.6}{2013/04/17}{New option \opt{viewmode}. Corrected bugs in the definition of \cmd{listoffigures}, \cmd{listoftables} and \cmd{tableofcontens} commands. Updated documentation.}
% \changes{v1.5}{2013/03/22}{The \sty{microtype} package is loaded for all engines. Improved compatibility with \sty{todonotes} package and with \sty{mdframed} and \sty{bookmark} packages when using \sty{style} class option. New option \opt{viewmode}}
% \changes{v1.4}{2013/03/15}{New value \opt{periodicalaureo} for option \opt{pagelayout}. Changed code for \cmd{xfootnote} command.}
% \changes{v1.3}{2013/03/05}{Updated documentation. Added \opt{FSPLa}, 
% \opt{FSPLb} and \opt{FSPLc} styles.}
% \changes{v1.2}{2012/10/24}{Maintenance release. Added the 
% \cmd{toclabelspace} command.}
% \changes{v1.1}{2012/10/20}{Added the \sty{fixltxhyph} package. 
% Documentation updated.}
% \changes{v0.9c}{2012/09/25}{Maintenance release, no changes}
% \changes{v0.9b}{2012/09/23}{Restored \opt{11pt} and \opt{12pt} option 
% to default}
% \changes{v0.9a}{2012/08/31}{Added \option{inline} option to 
% \sty{enumitem} package. Deleted \opt{centertitle} option. 
% Deleted \opt{sctitles} option. 
% Changed code for toc, lot and lof elements.}
% \changes{v0.9}{2012/04/22}{Improved compatibility 
% with \sty{mathspec}}
% \changes{v0.8}{2012/03/19}{Added macro for using \sty{frontespizio} 
% package with \opt{compact} and \opt{supercompact} options.}
% \changes{v0.7}{2011/02/05}{Maintenance release, no changes. 
% Provided a thesis template}
% \changes{v0.6c}{2011/11/16}{Maintenance release, no changes}
% \changes{v0.6b}{2011/11/07}{Improved full compatibility with 
% \XeLaTeX. 
% \sty{varioref} and \sty{footmisc} packages are no longer loaded}
% \changes{v0.6a}{2011/10/24}{Maintenance release, no changes}
% \changes{v0.6}{2011/10/21}{English documentation. Renamed options 
% \opt{plain} and \opt{fullplain} to \opt{centerheadings} 
% and \opt{sufplain}.}
% \changes{v0.5}{2011/10/21}{First public release. 
% The frontispiece of \sty{suftesi}{} is now
% included in the package \sty{frontespizio}. 
% New option \opt{fullplain} 
% Changed option \opt{sctitles}.}
% \GetFileInfo{suftesi.dtx}
%
%
% \DoNotIndex{\hskip,\newcommand,\newenvironment,\def,\begin,\vskip,\ }
% \DoNotIndex{\DeclareOption,\ExecuteOptions,\RequirePackage}
% \DoNotIndex{\@@end,\@empty,\@ifclassloaded,\@nameuse,\@nil}
% \DoNotIndex{\@undefined,\\,\`,\addtocounter,\advance,\bfseries}
% \DoNotIndex{\centering,\closeout,\define@key,\documentclass}
% \DoNotIndex{\edef,\else,\end,\endinput,\endtitlepage,\expandafter}
% \DoNotIndex{\extracolsep,\fi,\fill,\fontsize,\g@addto@macro,\toks}
% \DoNotIndex{\hrule,\hspace,\if,\if@twoside,\ifcase,\ifdefined}
% \DoNotIndex{\iffalse,\IfFileExists,\ifnum,\ifx,\immediate}
% \DoNotIndex{\jobname,\let,\long,\MakeUppercase,\MessageBreak}
% \DoNotIndex{\newcount,\newif,\newpage,\newtoks,\newwrite,\next}
% \DoNotIndex{\noexpand,\nofiles,\normalfont,\normalsize,\null}
% \DoNotIndex{\openout,\or,\styage,\styageError,\styageWarning}
% \DoNotIndex{\styageWarningNoLine,\paperheight,\paperwidth,\par}
% \DoNotIndex{\parbox,\parindent,\relax,\scshape,\selectfont,\setkeys}
% \DoNotIndex{\sffamily,\space,\stretch,\string,\textheight,\textwidth}
% \DoNotIndex{\the,\thispagestyle,\unexpanded,\unless,\unskip,\upshape}
% \DoNotIndex{\usepackage,\vbox,\vfill,\vfil,\vspace,\write,\z@}
% \DoNotIndex{\addvspace,\setcounter,\addcontentsline}
% \DoNotIndex{\filleft,\filcenter,\filright,\geometry}
% \DoNotIndex{\CurrentOption,\AtEndDocument,\@ne,\c@page,\m@ne}
% \DoNotIndex{\@firstofone,\@gobble,\@makeother,\begingroup,\endgroup}
% \DoNotIndex{\eTeXversion,\hbox,\hsize,\includegraphics,\newlinechar}
% \DoNotIndex{\titlepage,\vss,\vtop,\xdef,\@gobbletwo,\color,\dimexpr}
% \DoNotIndex{\huge,\large,\makebox,\ProcessOptions,\renewcommand}
%
%\maketitle
% 
% \begin{abstract}
% The standard document classes allow you to typeset beautiful
% documents but their layout is quite far from the stylistic requests
% of some humanists (mainly Italian). The \sty{suftesi} class
% provides a set of ``humanistic'' page layouts, title styles 
% and heading styles to typeset books, articles and theses. 
% The styles are very simple and sober and also have the aim of
% finding an aesthetic harmony in the formal simplicity  \parencite[see][]{valbusa:20122}. They are
% largely inspired by some interesting readings, particularly
% \citetitle{Bringhurst:1992} by Robert \textcite{Bringhurst:1992}, \citetitle{Tschichold:1975} by Jan \textcite{Tschichold:1975} and
% \citetitle{Morison:1111} by Stanley \textcite{Morison:1111}.
% \end{abstract}
%\begin{multicols}{2}
% \tableofcontents
%\end{multicols}
%
% \section*{Legalese}
%
%\noindent Copyright \copyright\ 2009-\the\year{} Ivan Valbusa
% \smallskip
%
% \noindent This package is author-maintained. 
%Permission is granted to copy, distribute and/or modify this software under the 
%terms of the LaTeX Project Public License, version 1.3c ora later (\url{http://latex-project.org/lppl}). This software is provided ``as is'', without warranty of any kind, either expressed or implied, including, but not limited to, the implied warranties of merchantability and fitness for a particular purpose.
%
% The main feature of this class is the set of styles it
% provides. For this reason {\scshape\lsstyle do not modify the styles of this class with packages 
% and/or commands which change the layout of the document. 
% If you do not like these styles, use another class.}
% If you use \sty{suftesi}{} in a document typeset with another class
% (for example \sty{book}), remember to clean up the preamble from all
% the layout redefinitions.
%
%
% \section*{A brief history}\thispagestyle{empty}
%
% The \sty{suftesi} class was born as a result of the course \emph{Introduzione a \LaTeX{} per le scienze umane} (\emph{Introduction to \LaTeX{} for the human sciences}) which I held at the Graduate School of Human Sciences and Philosophy  (\textcolor{magenta}{\textbf{S}}cienze \textcolor{magenta}{\textbf{U}}mane e \textcolor{magenta}{\textbf{F}}ilosofia) of Verona University (Italy) in June 2010. Originally thought as a class to typeset theses, during these years it has developed many new features and styles. Finally it has become the referential class of the Joint Project \emph{Formal Style for PhD Theses with LaTeX} of the University of Verona. 
%
% \section*{Acknowledgments}
%
% I would like to thank Professors Paola Di~Nicola,
% Director of the Graduate School, and Ugo Savardi who suggested to propose the course precisely to that School; Gilberto D'Arduini, Matteo Lanza and Antonio Rinaldi, who installed \LaTeX{} on the computers used during the course; Catia Cordioli, for her help in the organization of the lessons; Corrado Ferreri, responsible of the \mbox{E-Learning} Center, who provided the \TeX{}~Live~2009 \textsc{DVD}s.
% Special thanks to Professors Enrico~Gregorio, the Italian guru of \TeX, for the \TeX pert support and his priceless ``magic formulas'', and Tommaso Gordini for his valuable advice in choosing and improving the class features during these years.
%I would finally thank all the participants to the course who patiently resisted the four intensive lessons. This work is dedicated to them.
%
%
%
% \section*{Feedback}
%
% \noindent If you have any questions, feedback or requests please email me at \texttt{ivan dot valbusa at univr dot it}. If you need specific features not already implemented, remember to attach the example files.
%
% \section{Use}
%
% The \sty{suftesi} class is called as usual by
% \begin{ttquote}
% \cmd{documentclass}\oarm{options}\{suftesi\}
% \end{ttquote}
% All the options defined by the class are in the \meta{key}=\meta{value} format.
% ^^A \begin{verbatim}
% ^^A \cmd{documentclass}[\\
% ^^A \opt{pagelayout=periodical},\\
% ^^A \opt{chapfont=roman},\\
% ^^A \opt{tocstyle=ragged},\\
% ^^A \opt{marginpar=false},\\
% ^^A ...\\
% ^^A ]\ar{suftesi}
% ^^A \end{verbatim}
% In addition to these you can also use the options defined by the standard \sty{book} class (on which \sty{suftesi} is based) except those modifying 
% the page dimensions (\opt{a4paper}, \opt{a5paper}, \opt{b5paper},  \opt{legalpaper}, \opt{executivepaper}  and \opt{landscape}), which are automatically disabled. The class loads the packages listed in table \ref{tab:pkgloaded}.
%
% For using the class with the \XeTeX{} and \LuaTeX{} engines you need to load the \sty{fontspec} package (\sty{mathspec} is an alternative only for \XeTeX) and to select a main font. 
%
%
%\begin{table}[h]
%\centering 
%\fontsize{9.5}{11.5}\selectfont
%\begin{tabular}{>{\sffamily}r@{}>{\sffamily}l>{\raggedright\arraybackslash}>{\sffamily}p{7.6cm}}
%\toprule
%&\textrm{Global}&{caption}, {color}, {crop}, {enumitem}, {emptypage}, {extramarks}, 
%{fancyhdr}, {fixltxhyph}, {fontenc}, {geometry}, {iftex},
%{microtype}, {multicol}, {titlesec}, {titletoc}, {xkeyval}, (cclicenses)\\\midrule
%&\textrm{pdf\LaTeX{} only}&fontenc, substitutefont\\\midrule
%defaultfont=&standard&lmodern\\
%
%&palatino&textcomp, newpxtext, biolinum, inconsolata, {newpxmath}\\
%&{libertine}&{textcomp}, {libertine}, {biolinum}, {inconsolata}, {newtxmath}\\
%&cochineal&{textcomp}, {cochineal}, {biolinum}, {inconsolata}, {newtxmath}\\
%&compatibility&{mathpazo}, {beramono}\\\midrule
%mathfont=&extended&amsthm, mathalfa\\\bottomrule
%\end{tabular}
% \caption{Packages loaded by \sty{suftesi}}\label{tab:pkgloaded}
%\end{table}
%
% \changes{v0.9b}{2012/09/23}{The \sty{epigraph} package is no longer loaded}
%
%
%
% \section{Class features}
% 
% The \sty{suftesi} class provides a large set of options to customize the style of page, titles, headings and other text elements (see section \ref{sec:options}). The simplest way to get to know the class features is therefore to typeset one of the templates which you can find in the \texttt{/doc/latex/suftesi} folder in your \LaTeX{} distribution \parencite[see also][]{valbusa:20122}. The class files as well as the templates are also available online on the CTAN’s website at 
%\textcolor{magenta}{\url{http://www.ctan.org/pkg/suftesi}}.
%
%
% \subsection{Document structure}
% 
% With \sty{suftesi} you can typeset book-style documents (with chapters), article-style documents (without chapters) and collections of papers (see section \ref{sec:collection}). The kind of document is set by the \opt{structure} option which defaults to \opt{book} (see section \ref{sec:options}).
%
%\begin{ttquote}
%\cmd{documentclass}[structure=\meta{value},\meta{options}]\{suftesi\}
%^^A\cmd{documentclass}[structure=article,\meta{options}]\{suftesi\}\\
%^^A\cmd{documentclass}[structure=collection,\meta{options}]\{suftesi\}
%\end{ttquote}
%
% \changes{v0.9b}{2012/09/23}{The \sty{mparhack} package is no longer loaded. It is not compatible with \sty{crop}.}
% 
% \subsection{Page layouts}
%
% The class provides nine page layouts which can be selected by the \opt{pagelayout} option (table \ref{tab:layouts}). The \opt{standard} layout (default) or \opt{standardaureo} layout are aimed to typeset  A4 documents, while the other seven layouts are particularly suitable to typeset compact books, articles or theses. For these compact layouts  the \opt{version=screen} and \opt{version=cscreen} options are particularly useful as they show the output in its real size. See section \ref{sec:page-styles} for details.
%
% \begin{table}[h]
%\centering
%\makebox[\textwidth][c]{\begin{tabular}{@{}lcccccccc@{}}
%\toprule
%&\multicolumn{4}{c}{Dimensions (mm)}&\multicolumn{4}{c}{Proportions}\\\cmidrule(lr){2-5}\cmidrule(lr){6-9}
%&\multicolumn{2}{c}{Typeblock}&\multicolumn{2}{c}{Page}& \multicolumn{2}{c}{Margins}&\multicolumn{2}{c}{Stock}\\\cmidrule(lr){2-3}\cmidrule(lr){4-5}\cmidrule(lr){6-7}\cmidrule(l){8-9}
%Layout         &w &	h&	w&	h&t\,/\,b&i\,/\,o&Typeblock& Page\\\cmidrule(r){2-3}\cmidrule(lr){4-5}\cmidrule(lr){6-7}\cmidrule(l){8-9}\cmidrule(r){1-1}
%\opt{standard} &	110&	220&	210&	297&$1:2$&$1:2$&$1:2$&$1:\kern-3pt\sqrt{2}$\\
%\opt{standardaureo} &	136&	220&	210&	297&$2:3$&$2:3$&$5:8$\makebox[0pt]{\ \textsuperscript{*}}&$1:\kern-3pt\sqrt{2}$\\
%\opt{compact}&	110&	165&	160&	240&$2:3$&$2:3$&$2:3$&$2:3$\\
%\opt{compactaureo}&	118&	191&	160&	240&$2:3$&$2:3$&$5:8$\makebox[0pt]{\ \textsuperscript{*}}&$2:3$\\
%\opt{supercompact}&	100	&150&	140&	210&$2:3$&$2:3$&$2:3$&$2:3$\\
%\opt{supercompactaureo}&	108	&175&	140&	210&$1:1$&$1:1$&$5:8$\makebox[0pt]{\ \textsuperscript{*}}&$2:3$\\
%\opt{periodical}	&110&	165&	170&	240&$2:3$&$2:3$&$2:3$&$1:\kern-3pt\sqrt{2}$\\
%\opt{periodicalaureo}&	120&	194&	170&	240&$2:3$&$2:3$&$5:8$\makebox[0pt]{\ \textsuperscript{*}}&$1:\kern-3pt\sqrt{2}$\\
%
% \bottomrule\\[-2ex]
% \multicolumn{9}{r}{\footnotesize\textsuperscript{*} $5:8\approx 1:1,618$}
%\end{tabular}}\\[-3ex]
% \caption{The layouts of \sty{suftesi}}\label{tab:layouts}
% \end{table}
% 
%^^A The form of the book has been changing through the centuries and every content requires a particular shape. Nowdays the most widespread form for literary essays is a compact one. So, if you are interested in such a topic, you should consider the compact layouts. The one I prefer is shown on figure \ref{fig:periodicalaureo}.
%
%^^A\begin{figure}[h!]\centering
%^^A\noindent\begin{tikzpicture}[scale=0.3]
%^^A\draw (0,0) rectangle (17,24);
%^^A\draw[draw=none,fill=gray!50,xshift=3cm,yshift=2.6cm] (0,0) rectangle (12,19.4);
%^^A\draw[xshift=17cm] (0,0) rectangle ++(17,24);
%^^A\draw[xshift=17cm,draw=none,fill=gray!50,xshift=2cm,yshift=2.6cm] (0,0) rectangle ++(12,19.4);
%^^A\end{tikzpicture}
%^^A\caption{The \opt{periodicalaureo} layout}\label{fig:periodicalaureo}
%^^A\end{figure}
%
% \subsection{Fonts}
%
% 
%The default roman font is Cochineal by Michael Sharpe,
%the sans serif font is Linux Biolinum O, designed by Philipp H. Poll,\index{Poll, Philipp H.} and the typewriter face is Inconsolata by Michael Sharpe. Cochineal is a fork of Crimson, by Sebastina Kosch, a font inspired by masterpieces like Sabon (Jan Tschichold) and Minion (Robert Slimbach). It is a very complete typeface and it offers full support for Greek and Cyrillic, real small caps (even in italic shape) and four figure styles. Moreover it provides support for mathematics through the \sty{newtxmath} package.
%
%The \opt{defaultfont}  option allows you to can change the default roman font to New PX (Palatino-like), Linux Libertine O, or Latin modern. 
%You can use other fonts as well, but in this case remember to load the \opt{defaultfont=none} option which turns off the default font settings (see section \ref{sec:options} for details). 
%
%Another beautiful Garamond-like font, complete of real small caps, bold weight and mathematical support, is made available by the \sty{garamondx} package by Michael Sharpe, which provides an extension of the \sty{ugm} fonts, (URW)++ GaramondNo8. They are not distributed as part of \TeX Live, but they may be downloaded using the \texttt{getnonfreefonts} script. See the instructions for installation at
%\url{http://tug.org/fonts/getnonfreefonts/}.
%
%
% \subsubsection[Mathematics]{Typesetting mathematics}
%
% The class loads by default a ‘minimal’ mathematical support for Cochineal (default), Linux Libertine O or New PX via the \sty{newtxmath} or \sty{newpxmath} packages by Michael Sharpe. Moreover the \opt{mathfont} option is provided to extend or disable the support for mathematics when using these fonts. These non-standard fonts have a very good mathematical support but if you aim at typesetting high mathematics the Latin Modern font family remains, of course, the better choice: \opt{defaultfont=standard}. See section \ref{sec:fonts} for details. 
%
% If some of the loaded packages conflict or if you simply want to change some of the default font settings you should first reset the font default with \opt{defaultfont=none} option and then load the packages in the right order and with your favorite options. 
%^^AFor example, these are the codes loaded for the three non-standard fonts:
%^^A
%^^A\paragraph{defaultfont=cochineal}
%^^A
%^^A\begin{latexcode}
%^^A\begin{verbatim}
%^^A\documentclass[defaultfont=none]{suftesi}
%^^A...
%^^A\usepackage{textcomp}
%^^A\usepackage{cochineal}
%^^A\usepackage[varqu,varl,var0]{inconsolata}
%^^A\usepackage{biolinum}
%^^A\usepackage{cochineal}
%^^A\usepackage{amsthm}
%^^A\usepackge[cochineal,bigdelims,cmintegrals,vvarbb]{newtxmath}
%^^A\usepackge[cal=boondoxo]{mathalfa}
%^^A\useosf
%^^A\useproportional
%^^A\end{verbatim}
%^^A\end{latexcode}
%^^A
%^^A\paragraph{defaultfont=palatino}
%^^A
%^^A\begin{latexcode}
%^^A\begin{verbatim}
%^^A\usepackage[full]{textcomp}
%^^A\usepackage{newpxtext}
%^^A\usepackage[scaled=1.06]{biolinum}
%^^A\usepackage[varqu,varl]{inconsolata}
%^^A\usepackage{amsthm}
%^^A\usepackage[bigdelims,vvarbb]{newpxmath}
%^^A\usepackage[cal=boondoxo]{mathalfa} 
%^^A\useosf
%^^A\useproportional
%^^A\end{verbatim}
%^^A\end{latexcode}
%^^A
%^^A\paragraph{defaultfont=libertine}
%^^A
%^^A\begin{latexcode}
%^^A\begin{verbatim}
%^^A\usepackage{textcomp}
%^^A\usepackage[sb]{libertine}
%^^A\usepackage[varqu,varl,scaled=0.94]{inconsolata}
%^^A\usepackage{amsthm}
%^^A\usepackage[libertine,bigdelims,vvarbb]{newtxmath}
%^^A\usepackage[cal=boondoxo]{mathalfa} 
%^^A\useosf
%^^A\end{verbatim}
%^^A\end{latexcode}
% 
%
%
%
% \subsubsection[Greek]{Typesetting Greek}
% 
% The Cochineal default font is complete of Greek and Cyrillic. Anyway the class provides the \opt{greekfont} option (see section \ref{sec:fonts}) to select four different Greek fonts for use with the roman typeface
% set by the \opt{defaultfont} option (except for the \opt{defaultfont=standard} option, i.e. with Latin Modern font family which uses the CB Greek by default). These fonts are: GFS Bodoni, GFS Artemisia, GFS Porson (by the Greek Font Society) and CB Greek (by Claudio Beccari). You may, of course, choose a different font but in this case, as previously cleared, remember to load the \opt{defaultfont=none} option.
% 
%
% \greekexample{Cochineal-LF}{Cochineal-LF}{Cochineal/Cochineal}{10}
% \greekexample{Cochineal-LF}{bodoni}{Cochineal/Bodoni}{10}
% \greekexample{Cochineal-LF}{artemisia}{Cochineal/Artemisia}{10}
% \greekexample{Cochineal-LF}{porson}{Cochineal/Porson}{10}
% \greekexample{Cochineal-LF}{lmr}{Cochineal/CB Greek}{10}
% \greekexample{LinuxLibertineT-OsF}{LinuxLibertineT-OsF}{Libertine/Libertine}{10}
% \greekexample{LinuxLibertineT-OsF}{bodoni}{Libertine/Bodoni}{10}
% \greekexample{LinuxLibertineT-OsF}{artemisia}{Libertine/Artemisia}{10}
% \greekexample{LinuxLibertineT-OsF}{porson}{Libertine/Porson}{10}
% \greekexample{LinuxLibertineT-OsF}{lmr}{Libertine/CB Greek}{10}
% \greekexample{zplosf}{bodoni}{New PX/Bodoni}{10}
% \greekexample{zplosf}{artemisia}{New PX/Artemisia}{10}
% \greekexample{zplosf}{porson}{New PX/Porson}{10}
% \greekexample{zplosf}{lmr}{New PX/CB Greek}{10.5}
% \greekexample{lmr}{cbgreek}{Latin Modern/CB Greek}{10}
%
%
%\begin{figure}
% \centering
%\begin{tikzpicture}
% \draw[draw=gray,fill=white,drop shadow] (-.5\textwidth,0) rectangle (.5\textwidth,\textheight);
%\draw[line width=.5mm,fill=white,text=black,draw=black!60] (-4.2,17.1) circle (9mm) node {\parbox{12mm}{\centering\tiny * * * \\  St. Anford\\ University\\[1mm] * * * }};
%\node[anchor=north west] at (-2.5cm,18cm) {%
%  \parbox{8cm}{%
%    \small University of St. Anford\par
%      \normalsize Department of Typography
%      \vskip1ex\hrule\vskip1.2ex
%        \large Ph.D. degree in \TeX{} and \LaTeX
%
% \vspace{3cm}
%
%    {\LARGE\sffamily\color{sufred} How to prepare\par
%      a formal frontispiece\par}
%      \vspace{2ex}
%      
%      {\normalsize\sffamily Theory and practice\par}
%
% \vspace{2cm}
%
%    {\normalsize\sffamily Candidate:\par}
%    \small Enrico Gregorio
%      \vspace{5ex}
%      
%    {\normalsize\sffamily Thesis advisor:\par}
%    \small Prof. R. J. Drofnats
%    \vspace{2ex}
%
%    {\normalsize\sffamily Research supervisors:\par}
%    \small J. H. Quick\par 
%     B. L. User\vspace{4cm}
%
% Thesis submitted in 2010
% }
%      };
%
%\end{tikzpicture}
%\caption{The frontispiece of \sty{suftesi}}\label{fig:frontispiece}
%\end{figure}
%\begin{figure}
% \centering
%\begin{tikzpicture}
% \draw[draw=gray,fill=gray!5,drop shadow] (-.5\textwidth,0) rectangle (.5\textwidth,\textheight);
% \node at (0,.7\textheight) (c) {};
% \draw[ball color=DarkBlue,draw=none] (c) circle (4.5cm);
% \node[anchor=center,text=white] at (c) {\parbox{6cm}{%
%  \centering
%       {\scriptsize\scshape Bruce Lee\\}
%       \vspace*{\baselineskip}
%
%     {\Large\bfseries The Art of Kicking\\}
%
%       {\smallskip\normalsize How to survive in the modern societies\\}
%
%       {\vspace*{2\baselineskip}\scriptsize edited by\\ Walker Texas Ranger\\}}};
%\node[align=center] at (0,1) {\scriptsize \textcolor{black}{Punch Press}\\[1ex]\textcolor{black}{\fbox{§§§§}}\\};
%\end{tikzpicture}
%\caption{An example of the cover page of \sty{suftesi}}\label{fig:cover}
%\end{figure}
%
% \subsection{The frontispiece} \enlargethispage{\baselineskip}
%
% The class has an original frontispiece (see figure \ref{fig:frontispiece}) now directly available 
% loading the \sty{frontespizio} package  
% with the \opt{suftesi} option. It is meant only for Italian users
% \parencite[see][version 1.4 or later]{Gregorio:frontespizio}:
%
%\begin{latexcode}
%\begin{verbatim}
%\usepackage[suftesi]{frontespizio}
%\end{verbatim}
%\end{latexcode}
% Note that this frontispiece requires the use of a logo which could be restricted by some institutions. So before using it you have to be sure that you have the required permissions.
%
% The \sty{frontespizio} package produces a frontispiece in the standard Computer Modern typeface. If you prefer a consistent font remember to load the needed packages or commands in the \cmd{Preambolo} command:
%
%\begin{latexcode}
%\begin{verbatim}
%\begin{frontespizio} 
%  \Preambolo{\usepackage[osf]{cochineal}}%  <===
%  \Universita{Paperopoli} 
%  \Dipartimento{Filosofia, Pedagogia e Psicologia} 
%  \Corso[Laurea]{Filosofia} 
%  \Annoaccademico{2030--2031} 
%  \Titolo{La mia tesi:\\ una lunga serie di risultati\\
%      difficilissimi e complicatissimi} 
%  \Sottotitolo{Alcune considerazioni mutevoli} 
%  \Candidato[vr12301231]{Paolo Rossi} 
%  \Relatore{Guido Bianchi} 
%\end{frontespizio}
%\end{verbatim}
%\end{latexcode}
%
%
%
%^^A\subsection{Title page}
% 
%^^A The class provides a very simple title page through the \cmd{maketitle} command. A large collection of stylish title pages is provided by the \sty{titlepages} package by Peter Wilson. This package is part of  the \sty{memoir} documentation but you can copy the code examples and adapt them to work with \sty{suftesi} with quite simple changes.
%
% \subsection{The cover page}
% Since version 2.6 \sty{suftesi} provides a command to print a simple cover page (figure \ref{fig:cover}), inspired by the Italian designer Bruno \textcite[85-89]{munari:arte-come-mestiere}. The color used for the text and the circle as well as other graphical elements are partially customizable (see section \ref{sec:options} for details). The cover shown in figure \ref{fig:cover} is obtained with the following code:
%\begin{latexcode}
%\begin{verbatim}
%\Cauthor{Bruce Lee}
%\Ctitle{The Art of Kicking}
%\Csubtitle{How to survive in the modern societies}
%\Ceditor{edited by\\ Walker Texas Ranger}
%\Cfoot{Punch Press\\\fbox{\bfseries P\&P}}
%\Cpagecolor{white!90!black}
%\Ctextcolor{white}
%\Cfootcolor{black}
%
%\makecover[ball color=DarkBlue,draw=none]
%\end{verbatim}
%\end{latexcode}
%
%To produce a complete book cover, with spine and dust jacket, you should take a look at the \sty{bookcover} package by Tibor Tómács.
%
%
% \section{Options}\label{sec:options}
%
% \subsection{Layout}\label{sec:page-styles}
%
%\begin{optionlist}
%\optitem[book]{structure}{\opt{book}, \opt{article}, \opt{collection}}
% \changes{v0.9b}{2012/09/23}{New option \opt{structure}} 
% \begin{valuelist}
% \item[book] For typesetting texts with chapters.
% \item[article] For typesetting articles or short theses 
% (without chapters).
% \item[collection] For typesetting journals or collections of articles (see section \ref{sec:collection}).
% \end{valuelist}
%\optitem[standard]{pagelayout}{\opt{standard}, \opt{standardaureo},  \opt{periodical}, \opt{periodicalaureo}, \opt{compact}, \opt{compactaureo},  \opt{supercompact}, \opt{supercompactaureo}}
% \changes{v0.9b}{2012/09/23}{New option \opt{pagestyle}}
%\begin{valuelist}
%\item[standard]
% Prints an A4 page with a 
% typeblock of $\SI{110}{mm} \times \SI{220}{mm}$.
%\end{valuelist}
% With the following values the option prints the typeblock 
% on a an A4 paper showing the crop marks which can be controlled 
% by the \opt{version} option.
%\begin{valuelist}
%\item[periodical]
% Prints a page  of $\SI{17}{cm} \times \SI{24}{cm}$ with a typeblock of  $\SI{11}{cm} \times \SI{16,5}{cm}$.
%\item[periodicalaureo] The same of \opt{periodical} but with a 
% \emph{golden ratio} typeblock of $\SI{120}{cm} \times \SI{194}{mm}$. 
%\item[compact]
% Prints a page  of $\SI{16}{cm} \times \SI{24}{cm}$ with a typeblock of $\SI{11}{cm} \times \SI{16,5}{cm}$. 
%\item[compactaureo]
% The same as \opt{compact} but with 
% a \emph{golden ratio} typeblock of $\SI{11,8}{cm} \times \SI{19,1}{cm}$.%\item[supercompact]
% Prints a page  of  $\SI{14}{cm} \times \SI{21}{cm}$  with a typeblock of $\SI{10}{cm} \times \SI{15,5}{cm}$.
%\item[supercompactaureo]
% The same as \opt{supercompact} but with 
% a \emph{golden ratio} typeblock of $\SI{10,8}{cm} \times \SI{17,5}{cm}$.
%\end{valuelist}
% The details of this option are shown in table \ref{tab:layouts}.
% \optitem[final]{version}{\opt{screen}, \opt{cscreen}, \opt{draft}, \opt{final}}
% \begin{valuelist}
% \item[screen] Prints the \textsc{pdf} with its real dimensions.
% \item[cscreen] The same as \opt{version=screen} but with a centered typeblock.
% \end{valuelist}
% These previous two options are provided to have a better view when you are 
% typesetting and reading the \textsc{pdf} on the screen or for online publications. The following options meant for printed documents.
% \begin{valuelist}
% \item[draft] 
% Prints the output on a A4 paper, showing the crop marks. Useless with \opt{standard} and \opt{standardaureo} layouts.
% \item[final] Prints the output on a A4 paper, hiding the crop marks. Useless with \opt{standard} and \opt{standardaureo} layouts.
% \end{valuelist}
% Sometimes it is not desirable to have the crop marks on every page. In such a case you can use the \cmd{crop}|[off]| command after the first page of the document.
%\end{optionlist}
%
%\subsection{Font}\label{sec:fonts}
%
% The first three options are available only with \LaTeX.
% Using \XeLaTeX{} or \LuaLaTeX{} 
% the default font is the Computer Modern but you can change it through 
% the \sty{fontspec} or \sty{mathspec} (\XeLaTeX{} only) packages. If you do not need to typeset mathematics, with these engines I do suggest the EB Garamond font family by Georg Duffner.
% 
%\begin{optionlist}
% \optitem[cochineal]{defaultfont}{\opt{none}, \opt{cochineal}, \opt{libertine}, \opt{palatino}, \opt{standard}}
%   \begin{valuelist}
%     \item[none] Does not load any font. Use this option 
%		if you want full control over the font selection.
%     \item[cochineal] Loads the Cochineal serif, the Linux Biolinum O sans serif and the Inconsolata typewriter.
%     \item[libertine] Loads the Linux Libertine O serif, the Linux Biolinum O sans serif and the Inconsolata typewriter.
%     \item[palatino] Loads the New PX serif, the Linux Biolinum O sans serif and the Inconsolata typewriter. Note that the New PX font does not provide support for Greek. The \opt{greekfont=artemisa} option offers a very good solution.
%^^A and {\fontfamily{artemisia}\selectfont Artemisia} by the {\fontfamily{artemisia}\selectfont Greek Font Society} for the greek text.
%     \item[standard] Loads the \sty{lmodern} package: Latin Modern font family and CB Greek.
%   \end{valuelist}
% \optitem[none]{greekfont}{\opt{none}, \opt{artemisia}, \opt{porson}, \opt{bodoni}, \opt{cbgreek}}
% Actually useless  with \opt{defaultfont=none} and not available with \opt{defaultfont=standard} options.
% \changes{v0.9b}{2012/09/23}{New option \opt{greekfont}}
%   \begin{valuelist}
%     \item[none] Does not load any Greek font. 
%     \item[artemisia] Loads the Artemisia font by the Greek Font Society.
%     \item[porson] Loads the Porson font by the Greek Font Society.\footnote{The first code for the Porson font (\sty{suftesi} <v.2.4) has been written by Enrico 
% Gregorio. Claudio Beccari successively added the code to improve the 
% scale factor when using this greek font in combination with Palatino.}
%     \item[bodoni] Loads the Bodoni font by the Greek Font Society.
%     \item[cbgreek] Loads the standard CB~Greek font by Claudio Beccari.
%   \end{valuelist}
%
% \optitem[minimal]{mathfont}{\opt{none},\opt{minimal},\opt{extended}}
% 
% Available only with Cochineal, Libertine and New PX fonts.
%   \begin{valuelist}
%     \item[none] Do not load any mathematical support. Use this option if you need some packages that have to be loaded before \sty{newtxmath} and \sty{newpxmath}.
%     \item[minimal]  Loads \sty{newtxmath} (for Cochineal and Linux Libertine O) or \sty{newpxmath} (for PX Fonts).
%     \item[extended] Loads the previous option plus the \sty{amsthm} and \sty{mathalfa} packages. 
%   \end{valuelist}
% \end{optionlist}
%
% \begin{optionlist}
% \optitem[low]{smallcapsstyle}{\opt{low}, \opt{upper}}
%Active only with \opt{\meta{level}font=smallcaps} and \opt{toc\meta{level}font=smallcaps} options (see sections \ref{sec:titles} and \ref{sec:toc}).
% With \XeLaTeX{} or \LuaLaTeX{} this option is effective only if a font selection command (such as \cmd{setmainfont}) is given.
%   \begin{valuelist}
%     \item[low] Prints lowercase spaced \textsc{\lsstyle small capitals}.
%     \item[upper] Prints spaced \textsc{\lsstyle Small Capitals} with uppercase initials.
%   \end{valuelist}
% \end{optionlist}
%
% 	\subsection{Titles}\label{sec:titles}
%
% \begin{optionlist}
% \optitem[roman]{\meta{level}font}{\opt{roman}, \opt{italic}, 
% \opt{smallcaps}}
%   \begin{valuelist}
%     \item[roman] Prints the \meta{level} title in roman.
%     \item[italic] Prints the \meta{level} title in \emph{italic}.
%     \item[smallcaps] Prints the \meta{level} title 
%      in \textsc{\lsstyle spaced small caps}
%   \end{valuelist}
%   Where \meta{level} can be \opt{part}, \opt{chap}, \opt{sec},
%   \opt{subsec}, \opt{subsubsec}.
% \optitem[left]{\meta{level}style}{\opt{left}, \opt{center}, 
% \opt{right}, \opt{parleft}, \opt{parcenter}, 
% \opt{parright}}
%   \begin{valuelist}
%     \item[left] Aligns the \meta{level} title on the left.
%     \item[center] Centers the \meta{level} title. 
%     \item[right] Aligns the \meta{level} title on the right.
%   \end{valuelist}
%   Where \meta{level} can be \opt{part}, \opt{chap}, \opt{sec}
%   \opt{subsec}, \opt{subsubsec}.
%   \begin{valuelist}
%     \item[parleft] As \opt{left} but with the title below the number.
%     \item[parcenter] As \opt{center} but with the title below the number.
%     \item[parright] As \opt{right} but with the title below the number.
%   \end{valuelist}
%   Where \meta{level} can be \opt{part}, \opt{chap}, \opt{sec}.
% \optitem[Roman/arabic/arabic]{\meta{level}numstyle}{\opt{arabic}, 
% \opt{roman}, \opt{Roman}, \opt{dotarabic}, \opt{dotroman}, 
% \opt{dotRoman}}
%   \begin{valuelist}
%     \item[arabic] Arabic \meta{level} number.
%     \item[roman] Small caps lowercase roman \meta{level} number.
%     \item[Roman] Uppercase roman \meta{level} number.
%     \item[dotarabic] As \opt{arabic} but followed by a dot.
%     \item[dotroman] As \opt{roman} but followed by a dot.
%     \item[dotRoman] As \opt{Roman} but followed by a dot.
%   \end{valuelist}
%   Where \meta{level} can be \opt{part}, \opt{chap}, \opt{sec}.
% \end{optionlist}
%
%
% 	\subsection{Contents lists}\label{sec:toc}
%
%
% \begin{optionlist}
% \optitem[standard]{tocstyle}{\opt{standard}, \opt{dotted}, 
% \opt{ragged}, \opt{leftpage}}
% \changes{v0.9b}{2012/09/23}{New option \opt{tocstyle}}
%	\begin{valuelist}
%		\item[standard] Prints a standard table of contents with
%   page numbers on the right margin.
%		\item[dotted] As the previous one but with dotted lines.
%		\item[ragged] Aligns the table of contents on the left 
%			as suggested by \textcite{Bringhurst:1992}.
%		\item[leftpage] Prints a table of contents with page numbers 
%     on the left margin.
%	\end{valuelist}
% \optitem[roman]{toc\meta{level}font}{\opt{roman}, \opt{italic}, 
% \opt{smallcaps}}
%	\begin{valuelist}
%     \item[roman] Prints the \meta{level} TOC entry in roman.
%     \item[italic] Prints the \meta{level} TOC entry in \emph{italic}.
%     \item[smallcaps] Prints the \meta{level} TOC entry 
%      in \textsc{\lsstyle spaced small caps}
%	\end{valuelist}
%   Where \meta{level} can be \opt{chap}, \opt{sec}, \opt{subsec}, \opt{subsubsec}.
% \optitem[all]{twocolcontents}{\opt{toc}, \opt{lof}, 
% \opt{lot}, \opt{toclof}, \opt{toclot}, \opt{loflot}, \opt{all}}
% This option activates the \opt{tocstyle=ragged} option 
% and disables the other values of \opt{tocstyle}.
%	\begin{valuelist}
%		\item[toc] Prints the table of contents in two columns.
%		\item[lof] Prints the list of figures in two columns.
%		\item[lot] Prints the list of tables in two columns.
%		\item[toclof] Prints the table of contents 
%    and the list of figures in two columns.
%		\item[toclot] Prints the table of contents 
%    and the list of tables in two columns.
%		\item[loflot] Prints the list of figures 
%    and the list of tables in two columns.
%		\item[all] Prints all the contents lists in two columns.
%	\end{valuelist}
% \end{optionlist}
%
% \subsection{Headings}
%
% \begin{optionlist}
% \optitem[inner]{headerstyle}{\opt{inner}, \opt{center}, 
% \opt{plain}, \opt{authortitleinner}, \opt{authortitlecenter}}
%   \begin{valuelist}
%     \item[inner] Prints the chapter title and the string ``Chapter~ 
%     \meta{n}'' in the inner side 
%      respectively of even and odd 
%     headers, and the page number in the outer side.
%     \item[center] Centers the header and footer and puts 
%       the page number in the center of the footer.
%     \item[plain] Prints no headers and puts the page number 
%       in the center of the footer.
%     \item[authortitleinner]
%       Prints the author's 
%       name on the even pages and the title on the odd ones. 
%       In order to use this option the \cmd{title} and 
%       \cmd{author} commands are required.
%     \item[authortitlecenter]
%       As the previous one but with centered headers and footers.
%   \end{valuelist}
% \optitem[roman]{headerfont}{\opt{roman}, \opt{italic}, 
% \opt{smallcaps}}
%   \begin{valuelist}
%     \item[roman] Prints the headers in roman.
%     \item[italic] Prints the headers in \emph{italic}.
%     \item[smallcaps] Prints the headers in \textsc{\lsstyle spaced small caps}.
%   \end{valuelist}
% \end{optionlist}
%
% \subsection{Notes, lists, quotations}
%
% \begin{optionlist}
% \optitem[center]{quotestyle}{\opt{center}, \opt{right}}
%  \changes{v0.9b}{2012/09/23}{New option \opt{quotestyle}} 
% \begin{valuelist}
% \item[center] 
% Indents the block quotations
% on both the left and right margins.
% \item[right] 
% Indents the block quotations
% only on the left margin.
% \end{valuelist}
% \optitem[footnotesize]{quotesize}{\opt{footnotesize}, \opt{small}}
%  \changes{v0.9b}{2012/09/23}{New option \opt{quotationfont}} 
% \begin{valuelist}
% \item[footnotesize] 
% Prints the block quotations 
% in \cmd{footnotesize} size.
% \item[small] 
% Prints the block quotations 
% in \cmd{small} size.
% \end{valuelist}
% \optitem[bulged]{footnotestyle}{\opt{bulged}, \opt{hung}, 
% \opt{dotted}, \opt{superscript}}
% \changes{v0.9b}{2012/09/23}{Changed \opt{standardfootnote} option 
% and renamed to \opt{footnotestyle}}
% \begin{valuelist}
% \item[bulged] The footnote number protrudes beyond the left margin.
% \item[hung] 
% Indents the footnote text, so it will hang 
% under the first line of the text. 
% \item[dotted] 
% The footnote number is aligned to the left margin
% and followed by a dot. 
% \item[superscript] 
% Prints a superscript footnote number aligned to the left margin. 
% \end{valuelist}
% \boolitem[false]{fewfootnotes}
% \changes{v0.9b}{2012/09/23}{New option \opt{fewfootnotes}}
% Uses symbols instead of numbers to mark footnotes.
% It is active only in documents with 
% three footnotes per page maximum. 
% The symbol sequence is: *, **, ***. 
% With more footnotes you should not need this option.
% \optitem[bulged]{liststyle}{\opt{bulged}, \opt{aligned}, \opt{indented}}
% \begin{valuelist}
% \item[bulged]
% The item number or symbol protrudes beyond the left margin.
% \item[aligned]
%  Aligns the lists to the left margin.
% \item[indented]
%  Prints standard indented lists.
% \end{valuelist}
% \optitem[standard]{captionstyle}{\opt{standard}, \opt{sanserif}, \opt{italic}, \opt{smallcaps}}
% All these values print the caption in small size, changing the shape:
% \begin{valuelist}
% \item[standard] Prints the caption text and label in normal font.
% \item[sanserif] Prints the caption text and label in \textsf{sanserif}.
% \item[italic] Prints the caption text and label in \emph{italic}. 
% \item[smallcaps] Prints only the caption label in \textsc{\lsstyle spaced small caps}.
% \end{valuelist}
% \end{optionlist}
%
%
% \subsection{Miscellaneous}
%
%\begin{optionlist}
% \optitem[compact]{parindent}{\opt{compact}, \opt{wide}}
%  \changes{v0.9b}{2012/09/23}{New option \opt{parindent}} 
% \begin{valuelist}
% Sets the indentation of the first line of each paragraph except those following a section title.
% \item[compact] 
%  Sets indentation to 1\,em.
% \item[wide] 
% Sets indentation to 1.5\,em.
% \end{valuelist}
%\boolitem[true]{marginpar}
% \begin{valuelist}
% \item[true] Prints the marginal notes.
% \item[false] Hide the marginal notes.
% \end{valuelist}
%\boolitem[false]{partpage}
% \changes{v0.9b}{2012/09/23}{New option \opt{partpage}}
%  Active only with \opt{structure=article}.
% \begin{valuelist}
% \item[true] Prints the part title in a separate page as in 
% \opt{structure=book}
% \item[false] Prints a part title similar to the one used in the standard \opt{article} class.
% \end{valuelist}
% \optitem[false]{draftdate}{\opt{true}, \opt{false}}
%  \changes{v1.0}{2012/10/14}{First stable release. Renamed option \opt{bozza} to \opt{draftdate}} 
% If \opt{true} it prints the string ``Version of \meta{date}'' in the footer. It requires the \cmd{date}\ar{\meta{date}} command.
% \end{optionlist}
%
% \subsection{Pre-defined styles}\label{sec:predefined-styles}
%
% These pre-defined styles are intended as shotrcuts to some groups of 
% class options. Combining them
% with \opt{chapnumstyle} options 
% you can get up to 120 variants. 
%
% \begin{optionlist}
% \optitem[roman1]{style}{\opt{roman(1-6)}, \opt{italic(1-12)}, 
% \opt{smallcaps(1-6)}}
% \begin{valuelist}
% \item[roman(1-6)] The titles of chapters and headers are printed 
% in roman. The number of the chapter is on the same line in styles 
% 1-3 and above the title in styles 4-6.
% The title can be printed on the left (styles 1 and 4), 
% in the center (styles 2 and 5) or on the right (styles 3 and 6).
% \item[italic(1-12)]
% The titles of chapters and headers are in \emph{italic}. The section title is in \emph{italic}  in styles 1-6 and in \textsc{\lsstyle spaced small caps}  in styles 7-12.
% There are three position for the title and two positions for the number as above.
% \item[smallcaps(1-12)]
% The titles of chapters and headers are in \textsc{\lsstyle spaced small caps}. The  title of the section is in \emph{italic}  in styles 1-6 and in \textsc{\lsstyle spaced small caps}  in styles 7-12.
% There are three position for the title and two positions for the chapter as above.
% \end{valuelist}
% \end{optionlist}
%
%
% \section{New commands}\label{sec:commands}
%
% \subsection{Printing the cover page}
%
% \begin{ltxsyntax}
% \cmditem{makecover}
% \cmditem{makecover}[tikz options]
%
% Prints the cover page (figure \ref{fig:cover}). This command requires \sty{tikz} and the optional arguments accepts the same options of the \cmd{draw} command of that package (see the examples below). Commands available:
%
% \cmditem{Cauthor}{text} 
%
%Printed at the top of the circle.
%
% \cmditem{Ctitle}{text}  
%
% Printed below the author.
%
% \cmditem{Csubtitle}{text}  
%
%Printed below the title.
%
% \cmditem{Ceditor}{text} 
% 
%Printed below the subtitle.
%
% \cmditem{Cfoot}{text}  
%
%Printed in the footer.
%
% The color of the circle can be customized using the optional argument of the \cmd{makecover} command. For the other elements of the cover page the following commands are available. (You can load \sty{xcolor} with your favorite option to access to many beautiful colors.)
%
% \cmditem{Cpagecolor}{color} 
%
%The color of the cover page. 
%
% \cmditem{Ctextcolor}{color} 
%
%The color of the text inside the circle.
%
% \cmditem{Cfootcolor}{color} 
%
%The color of the text in the footer.
%
% \end{ltxsyntax}
% 
% \subsubsection*{Some examples of cover pages}
%
% First you have to declare the informations you want to put in the cover page. It is better to give these informations in the preamble:
%
%\begin{latexcode}
%\begin{verbatim}
%\usepackage{tikz}
%
%\Cauthor{Bruce Lee}
%\Ctitle{The Art of Kicking}
%\Csubtitle{How to survive in the modern societies}
%\Ceditor{edited by\\ Walker Texas Ranger}
%\Cfoot{Punch Press\\\includegraphics[width=1cm]{logo}}
%\end{verbatim}
%\end{latexcode}
%
%Then you can print the cover page with:
%
%\begin{ttquote}
%\cmd{makecover}\phantom{\oar{\meta{tikz options}}}
%\end{ttquote}
%
% or
%
%\begin{ttquote}
%\cmd{makecover}\oar{\meta{tikz options}}
%\end{ttquote}

% You can customize the circle using the optional arguments of the \cmd{makecover} command. Some of these \meta{tikz options} require special \sty{tikz} libraries. For example, you can add a shadow to the circle loading the \texttt{shadows} library and using the \texttt{circular drop shadow} otpion:
%
%\begin{latexcode}
%\begin{verbatim}
%\usepackage{tikz}
%\usetikzlibrary{shadows}
%...
%\begin{document}
%\makecover[circular drop shadow]
%\end{document}
%\end{verbatim}
%\end{latexcode}
%
% There are thousands of possibilities. Here is another (not necessarily good) example:
%\begin{latexcode}
%\begin{verbatim}
%\usepackage{tikz}
%
%\Cpagecolor{gray!30}
%\Ctextcolor{blue!50}
%\Cfootcolor{black}
%
%\begin{document}
%\makecover[fill=blue!30!black,draw=teal,line width=2mm,dashed]
%\end{document}
%\end{verbatim}
%\end{latexcode}
% 
%
% \subsection{Printing the colophon} 
% 
% \begin{ltxsyntax}
%
% \cmditem{colophon}[OS]{name and surname}{additional info}
% 
% This command is provided only for Italian documents. It prints a page with 
% a copyright notice and the colophon in the bottom of the page. For different languages 
% use \cmd{bookcolophon} instead (see below).
% 
% If you don't need the copyright notice, leave the second argument of 
% the command empty:
% \begin{ttquote}
% \cmd{colophon\oarm{OS}\ar{}\arm{additional info}}
% \end{ttquote}
% With the \opt{article} document structure, you can use the \cmd{artcolophon} command 
% (see below) 
% as well as  the \cmd{thanks} command:
%
%\begin{latexcode}
%\begin{verbatim}
%\author{Name Surname
%  \thanks{This work has been typeset with \LaTeX, using the 
%  \textsf{suftesi} class by Ivan Valbusa}.}
%\end{verbatim}
%\end{latexcode}
%
% \cmditem{bookcolophon}{copyright notice}{attribution notice and 
% other informations}
% \changes{v0.6}{2011/10/21}{New command \cmd{bookcolophon}}
%
% Similar to \cmd{colophon} but fully customizable. 
% The first argument prints its content (usually the copyright notice) in the center of the page. 
% The second one prints its content at the bottom. For example:
%
%\begin{latexcode}
%\begin{verbatim}
% \bookcolophon{%
%    Copyright © 2007 by Ivan Valbusa}{%
%    This work has been typeset with \LaTeX, using the  \textsf{suftesi} 
%    class by Ivan Valbusa\index{Valbusa, Ivan}. The serif font is 
%    Cochineal by Michael Sharpe and the sans serif font is Linux 
%    Biolinum O by Philipp H. Poll.}
%\end{verbatim}
%\end{latexcode}
%
% \cmditem{artcolophon}{copyright notice, attribution and other informations}
% \changes{v0.6}{2011/10/21}{New command \cmd{artcolophon}}
%
% This command only has one argument. It simply prints its content at the 
% bottom of the page. 
% Here is an example:
%
%\begin{latexcode}
%\begin{verbatim}
% \artcolophon{%
%    This work is licensed under the Creative Commons 
%    Attribution-NonCommercial-NoDerivs 3.0 Unported 
%    License. To view a copy of this license, visit
%     \begin{center}
%       http://creativecommons.org/licenses/by-nc-nd/3.0
%     \end{center}
%    or send a letter to Creative Commons, 444 
%    Castro Street, Suite 900, Mountain View, 
%    California, 94041, USA.\\[1ex]
%
%    Typeset with \LaTeX,  using the \textsf{suftesi} 
%    class by Ivan Valbusa.}
%\end{verbatim}
%\end{latexcode}
%
% \cmditem{finalcolophon}{colophon content}
%
% The same as \cmd{artcolophon}, but centering its content at the 
% top of the page. It is aimed at typesetting a classical \emph{colophon} 
% at the end of the work.
%
% \end{ltxsyntax}
%
% \subsection[Breaking titles]{Breaking titles}
%
% \begin{ltxsyntax}
% \cmditem{headbreak}
%
% A manual break which is active for the table of contents but not in the text or in the headers.
%
% \begin{latexcode}
%\begin{verbatim}
% \section{This title will be break here \headbreak{} 
%     inside the table of contents}
%\end{verbatim}
% \end{latexcode}
% \cmditem{xheadbreak}
%
% A manual break which is active in the text but not in the headers and in the table of contents.
%
% \begin{latexcode}
%\begin{verbatim}
% \section{This title will be break here \xheadbreak{} 
%     inside the text}
%\end{verbatim}
% \end{latexcode}
%
% \end{ltxsyntax}
%
% \subsection{Miscellaneous}
%
% \begin{ltxsyntax}
% \cmditem{xfootnote}[symbol]{footnote text}
% \changes{v0.9b}{2012/09/23}{New command \cmd{xfootnote}}
%
% A command to print a footnote with a discretionary
% symbols given in the optional argument (default=*).  
% It does not increment the footnote counter.
% \begin{ttquote}
% \cmd{xfootnote}\oar{\em\textdollar\cmd{dagger}\textdollar}\arm{Footnote text}\\
% \cmd{xfootnote}\oar{\em\cmd{textdagger}}\arm{Footnote text}
% \end{ttquote}
%
% \cmditem{title}[short title for headers]{complete title for titlepage}
% \changes{v0.8}{2012/03/19}{Renewed \cmd{title} command}
% 
% Useful with \opt{headerstyle=authortitle} option if the title is too long
% or has some breaks. 
%
% \end{ltxsyntax}
%
% \begin{ltxsyntax}
% \cmditem{toclabelwidth}{level}{dim}
%  \changes{v0.9a}{2012/08/31}{New command \cmd{toclabelspace}}
%
% Adds the \meta{dim} to the \meta{level} label in the table of contents, where \meta{level} can be \opt{part}, \opt{chap},  \opt{sec},  \opt{subsec},  \opt{subsubsec},  \opt{par},  \opt{subpar}, \opt{fig}, \opt{tab}. For example, when using \opt{chapnumstyle=Roman} you would probably need to adjust the width of the chapter label with:
%
%\begin{latexcode}
%\begin{verbatim}
% \toclabelwidth{chap}{1em}
%\end{verbatim}
%\end{latexcode}
% \end{ltxsyntax}
%
%
%
% \begin{ltxsyntax}
%
% \cmditem{chapterintro}
%
% Prints an unnumbered introduction at the beginning of the chapter, 
% with the correct hyperlink. In order to use this command the \sty{hyperref} package must be loaded.
%
% \cmditem{chapterintroname}{name}\hfill(default=\texttt{Introduzione})
%
% Changes in \meta{name} the title printed by the \cmd{chapterintro} command.
%
% \cmditem{appendixpage}
% 
% Prints a page with the argument of \cmd{appendicesname} (default=\texttt{Appendici}) at the center. Particularly useful if you have two or more appendices.
%
% \cmditem{appendicesname}{name}\hfill(default=\texttt{Appendici})
%
% Changes in \meta{name} the title printed by the \cmd{appendixpage} command.
%
%\end{ltxsyntax}
%
% \changes{v0.5}{2011/10/21}{New command \opt{chapnumfont}}
%
% \section{Collections}\label{sec:collection}
%
% The \opt{collection} document structure is thought to create a collection of papers. Each paper has to be typeset in a separate \file{.tex} file inside the \env{article} environment:
%
%\begin{latexcode}
%\begin{verbatim}
% \begin{article}
% \author{Author}
% \title{Title of the paper}
%  % The abstract is optional.
%  % \begin{abstract}
%  %  The abstract
%  % \end{abstract}
%
% \maketitle
%
%   Text of the paper
% \end{article}
%\end{verbatim}
% \end{latexcode}
% If the names of the papers are \file{article1.tex}, \file{article2.tex}, \file{article3.tex}, etc., then a minimal main file should be similar to this:
%
%\begin{latexcode}
%\begin{verbatim}
% \documentclass[structure=collection]{suftesi}
%
% \begin{document}
%  \input{article1}
%  \input{article2}
%  \input{article3}
% \end{document}
%\end{verbatim}
%\end{latexcode}
%
% \subsection{Options}
%
% In addition to the following options you can use the other options of the class too. In particular, with the \opt{chapstyle} and \opt{chapfont} options you can customize all the section titles which are treated as a normal unnumbered chapters in the \opt{book} document structure, such as ``Table of Contents'', ``Index'', ``Bibliography'', etc.
%
% \begin{optionlist}
% \optitem[left]{papertitlestyle}{\opt{left}, \opt{center}, \opt{right}}
%   \begin{valuelist}
%     \item[left] Aligns the author-title block on the left.
%     \item[center] Centers the author-title block. 
%     \item[right] Aligns the author-title block on the right.
%   \end{valuelist}
% \optitem[false]{revauthortitle}{\opt{true}, \opt{false}}
%   \begin{valuelist}
%     \item[true] Prints the author’s name below the title.
%     \item[false] Prints the author’s name above the title.
%   \end{valuelist}
% \optitem[italic]{titlefont}{\opt{roman}, \opt{italic}, \opt{smallcaps}}
%   \begin{valuelist}
%     \item[roman] Prints the title of the articles in roman.
%     \item[italic] Prints the  title of the articles in \emph{italic}.
%     \item[smallcaps] Prints the  title of the articles 
%      in \textsc{\lsstyle spaced small caps}.
%   \end{valuelist}
% \optitem[roman]{authorfont}{\opt{roman}, \opt{italic}, \opt{smallcaps}}
%   \begin{valuelist}
%     \item[roman] Prints the author's name in roman.
%     \item[italic] Prints the author's name in \emph{italic}.
%     \item[smallcaps] Prints the author's name in \textsc{\lsstyle spaced small caps}. 
%
%        Note that if you use this last option you need 
%         to protect the \cmd{thanks} command:
%
% \begin{latexcode}
%\begin{verbatim}
%\author{The Author\protect\thanks{...}}
%\end{verbatim}
% \end{latexcode}
%\end{valuelist}
% \optitem[italic]{toctitlefont}{\opt{roman}, \opt{italic}, \opt{smallcaps}}
%   \begin{valuelist}
%     \item[roman] Prints the title TOC entry in roman.
%     \item[italic] Prints the  title TOC entry in \emph{italic}.
%     \item[smallcaps] Prints the  title TOC entry  
%      in \textsc{\lsstyle spaced small caps}.
%   \end{valuelist}
% \optitem[roman]{tocauthorfont}{\opt{roman}, \opt{italic}, \opt{smallcaps}}
%   \begin{valuelist}
%     \item[roman] Prints the author’s name TOC entry in roman.
%     \item[italic] Prints the  author’s name TOC entry in \emph{italic}.
%     \item[smallcaps] Prints the  author’s name TOC entry  
%      in \textsc{\lsstyle spaced small caps}. 
%\end{valuelist}
% \end{optionlist}
%
% \subsection{Commands} 
% \begin{ltxsyntax} 
%
% \cmditem{frontispiece} 
% 
% Typeset the frontispiece of the collection.
%
% It requires the \cmd{collectiontitle} and \cmd{collectioneditor} commands in the preamble of your document.
%
% \cmditem{collectiontitle}{The Title of the Collection} 
%
% Sets the title of the collection. 
%
% \cmditem{collectioneditor}{The Editor(s)} 
%
% Sets the editor(s) of the collection.
%\end{ltxsyntax}
%
% \section{Known issues}
%
% A problem occurs with the article document structure. The \cmd{part} command resets the headers so if it is the first sectioning command of the page you will get no headers in that page. In this case you have to add the header manually with something like this:
%\begin{latexcode}
%\begin{verbatim}
%\markboth{Section title}{Section title}
%\part{The title of the part}
% %
%\section{Section title}
%\end{verbatim}
%\end{latexcode}
%If the title of the part is at the beginning of the page you will need the \sty{afterpage} package and the \cmd{afterapge} command:
%\begin{latexcode}
%\begin{verbatim}
%\afterpage{
%  \markboth{Section title}{Section title}
%    \part{The title of the part}
%   }
% %
%\section{Section title}
%\end{verbatim}
%\end{latexcode}
%
%
%
% \section{Backward compatibility}
%
% \begin{optionlist}
% \optitem{defaultfont}{\opt{compatibility}}
% Loads the fonts of \sty{suftesi} v2.3 (and previous): Palatino (\sty{mathpazo}), Iwona, Bera Mono. To get the default Greek font of those versions (i.e. Artemisia) add the \opt{greekfont=artemisia} option.
% \optitem[book]{documentstructure}{\opt{book}, \opt{article}, \opt{collection}}
% An alias for \opt{structure}
%\boolitem[true]{crop}
% \begin{valuelist}
% \item[true] An alias for \opt{version=draft} option.
% \item[false] An alias for \opt{version=center} option.
% \end{valuelist}
%
%
% \optitem{style}{\opt{FSPLa}, \opt{FSPLb}, \opt{FSPLc}}
% \end{optionlist}
% These styles are only meant to typeset a doctoral thesis respecting the features required by the Joint Project \emph{Formal Style for PhD Theses with LaTeX} of the Verona University  (Italy).
%
% 
%
% \changes{v1.3}{2013/03/05}{New command \cmd{FSPLcolophon}}
%\noindent The \cmd{FSPLcolophon}\ar{\meta{Name Surname}} command is provided to typeset the colophon according to these styles:
%\medskip
%
%\noindent\rule{\textwidth}{.4pt}
%
%\noindent\textcircled{\raisebox{1pt}{\scalebox{.7}{cc}}} \the\year{} \meta{Name Surname}%
%\vskip1ex
% 
%
%\small\noindent This work is licensed under the Creative Commons 
%Attribution-NonCommercial-NoDerivs 3.0 Unported License. 
%
%\noindent To view a copy of this license, 
%visit http://creativecommons.org/licenses/by-nc-nd/3.0/.
%
%\footnotesize
%
%\bigskip
%
%\noindent Typeset with \LaTeX{} in collaboration with the Joint Project 
%\emph{Formal Style for PhD Theses with \LaTeX{}} (University of Verona, 
%Italy) using the \textsf{suftesi} class by Ivan Valbusa. The text face 
%is Palatino, designed by Hermann Zapf. The sans serif font is Iwona by Janus M. Nowacki.
%
%
%\noindent\rule{\textwidth}{.4pt}
%
% ^^A\section*{Obsolete options}
%
% ^^A \begin{multicols}{2}
% ^^A \begin{ltxsyntax}
% ^^A \setlength{\parskip}{.5ex}
% ^^A \setlength{\itemindent}{1.5cm}
% ^^A \setlength{\labelwidth}{2.5cm}
% ^^A \optitem{defaultparindent}
% \changes{v0.9a}{2012/08/31}{New option \opt{defaultparindent}} 
% ^^A $\rightarrow$ \opt{parindent=compact}
% ^^A \optitem{ralignquotation}
% \changes{v0.9a}{2012/08/31}{New option \opt{ralignquotation}}
% ^^A$\rightarrow$ \opt{quotestyle=right}
% ^^A \optitem{smallquotation}
% \changes{v0.9a}{2012/08/31}{New option \opt{smallquotation}} 
% ^^A $\rightarrow$ \opt{quotesize=small}
% ^^A \optitem{dottedfootnote}
% \changes{v0.9a}{2012/08/31}{New option \opt{dottedfootnote}}
% ^^A $\rightarrow$ \opt{footnotes=dot}
% ^^A \optitem{indentlist} 
% \changes{v0.9a}{2012/08/31}{New option \opt{indentlist}} 
% ^^A $\rightarrow$ \opt{liststyle=indented}
% ^^A \optitem{alignlist} 
% \changes{v0.9a}{2012/08/31}{New option \opt{alignlist}} 
% ^^A $\rightarrow$ \opt{liststyle=aligned}
% ^^A \optitem{artemisia} $\rightarrow$ \opt{greekfont=artemisia}
% ^^A \optitem{porson} $\rightarrow$ \opt{greekfont=porson}
% ^^A \optitem{defaultgreek} $\rightarrow$ \opt{greekfont=cbgreek}
% ^^A \optitem{defaultfont} 
% ^^A $\rightarrow$ \opt{defaultfont=standard}
% ^^A \optitem{centerpart}
% \changes{v0.9a}{2012/08/31}{New option \opt{centerpart}} 
% ^^A $\rightarrow$ \opt{partstyle=center}
% ^^A \optitem{centerchap}
% \changes{v0.9a}{2012/08/31}{New option \opt{centerchap}} 
% ^^A $\rightarrow$ \opt{chapstyle=center}
% ^^A \optitem{centersec}
% \changes{v0.9a}{2012/08/31}{New option \opt{centersec}} 
% ^^A $\rightarrow$ \opt{secstyle=center}
% ^^A \optitem{rightpart}
% \changes{v0.9a}{2012/08/31}{New option \opt{rightpart}} 
% ^^A $\rightarrow$ \opt{partstyle=right}
% ^^A \optitem{rightchap}
% \changes{v0.9a}{2012/08/31}{New option \opt{rightchap}} 
% ^^A $\rightarrow$ \opt{chapstyle=right}
% ^^A \optitem{rightsec}
% \changes{v0.9a}{2012/08/31}{New option \opt{rightsec}} 
% ^^A $\rightarrow$ \opt{secstyle=right}
% ^^A \optitem{numparpart}
% \changes{v0.9a}{2012/08/31}{New option \opt{numparpart}} 
% ^^A $\rightarrow$ \opt{partnumposition=above}
% ^^A \optitem{numparchap}
% \changes{v0.9a}{2012/08/31}{New option \opt{numparchap}} 
% ^^A $\rightarrow$ \opt{chapnumposition=above}
% ^^A \optitem{numparsec}
% \changes{v0.9a}{2012/08/31}{New option \opt{numparsec}} 
% ^^A $\rightarrow$ \opt{secnumposition=above}
% ^^A \optitem{smallcapspart}
% \changes{v0.9a}{2012/08/31}{New option \opt{smallcapspart}} 
% ^^A $\rightarrow$ \opt{partstyle=smallcaps}
% ^^A \optitem{smallcapschap}
% \changes{v0.9a}{2012/08/31}{New option \opt{smallcapschap}} 
% ^^A $\rightarrow$ \opt{chapstyle=smallcaps}
% ^^A \optitem{smallcapssec}
% \changes{v0.9a}{2012/08/31}{New option \opt{smallcapssec}} 
% ^^A $\rightarrow$ \opt{subsecstyle=smallcaps}
% ^^A \optitem{italicpart}
% \changes{v0.9a}{2012/08/31}{New option \opt{italicpart}} 
% ^^A $\rightarrow$ \opt{partstyle=italic}
% ^^A \optitem{italicchap}
% \changes{v0.9a}{2012/08/31}{New option \opt{italicchap}} 
% ^^A $\rightarrow$ \opt{chapsecstyle=italic}
% ^^A \optitem{italicsec}
% \changes{v0.9a}{2012/08/31}{New option \opt{italicsec}} 
% ^^A $\rightarrow$ \opt{secstyle=italic}
% ^^A \optitem{italicsubsec}
% \changes{v0.9a}{2012/08/31}{New option \opt{italicsubsec}}
% ^^A $\rightarrow$ \opt{subsecstyle=italic}
% ^^A \optitem{romanchapnum}
% \changes{v0.9a}{2012/08/31}{New option \opt{romanchap}} 
% ^^A $\rightarrow$ \opt{chapnumstyle=roman}
% ^^A \optitem{dottedchap}
% \changes{v0.9a}{2012/08/31}{New option \opt{dottedchap}}
% ^^A $\rightarrow$ \opt{chapnumstyle}
% ^^A \optitem{italicheader} 
% \changes{v0.9a}{2012/08/31}{New option \opt{italicheader}} 
% ^^A $\rightarrow$ \opt{headerfont=italic}
% ^^A \optitem{centerheader}
% ^^A $\rightarrow$ \opt{headerstyle=center}
% ^^A \optitem{sufplain}
% ^^A $\rightarrow$ \opt{headerstyle=plain}
% ^^A \optitem{authortitle}
% \changes{v0.6}{2011/10/21}{New option \opt{authortitle}}
% ^^A $\rightarrow$ \opt{headerstyle}
% ^^A \optitem{periodical} 
% \changes{v0.8}{2012/03/19}{New option \opt{periodical}}
% ^^A $\rightarrow$ \opt{pagestyle=periodical}
% ^^A \optitem{compact} 
% \changes{v0.5}{2011/10/21}{New option \opt{compact}}
% ^^A $\rightarrow$ \opt{pagestyle=compact}
% ^^A \optitem{supercompact} 
% \changes{v0.5}{2011/10/21}{New option \opt{supercompact}}
% ^^A $\rightarrow$ \opt{pagestyle=supercompact}
% ^^A \optitem{dottedtoc} 
% \changes{v0.9}{2012/04/22}{New option \opt{dottedtoc}}
% ^^A $\rightarrow$ \opt{tocstyle=dotted}
% ^^A \optitem{raggedtoc} 
% \changes{v0.9}{2012/04/22}{New option \opt{raggedtoc}}
% ^^A $\rightarrow$ \opt{tocstyle=ragged}
% ^^A \optitem{tocpageleft}  
% \changes{v0.9a}{2012/08/31}{New option \opt{tocpageleft}}
% ^^A $\rightarrow$ \opt{tocstyle=leftpage}
% ^^A \optitem{dottedpart}
% \changes{v0.9a}{2012/08/31}{New option \opt{dottedpart}} 
% ^^A $\rightarrow$ \opt{partnumstyle=dotarabic}
% ^^A \optitem{elements} 
% \changes{v0.5}{2011/10/21}{New option \opt{elements}}
% ^^A $\rightarrow$ \opt{style=elements}
% ^^A \optitem{nomarginpar}
% \changes{v0.5}{2011/10/21}{New option \opt{nomarginpar}}
% ^^A $\rightarrow$ \opt{marginpar=false}
% ^^A \optitem{nocrop} 
% \changes{v0.5}{2011/10/21}{New option \opt{nocrop}}
% ^^A $\rightarrow$ \opt{crop=false}
% ^^A \optitem{rmstyle(1-6)} 
% \changes{v0.9a}{2012/08/31}{New option \opt{rmstyle(1-6)}} 
% ^^A $\rightarrow$ \opt{style=roman(1-6)}
% ^^A \optitem{itstyle(1-12)}
% \changes{v0.9a}{2012/08/31}{New option \opt{itstyle(1-12)}} 
% ^^A $\rightarrow$ \opt{style=italic(1-6)}
% ^^A \optitem{scstyle(1-12)}
% \changes{v0.9a}{2012/08/31}{New option \opt{scstyle(1-12)}}
% ^^A $\rightarrow$ \opt{style=smallcaps(1-6)}
% ^^A \optitem{sufelements} 
% \changes{v0.5}{2011/10/21}{New option \opt{sufelements}}
% ^^A $\rightarrow$ \opt{style=sufelements}
% ^^A \optitem{standardfootnote} 
% \changes{v0.9a}{2012/08/31}{New option \opt{standardfootnote}}
% ^^A $\rightarrow$ \opt{footnotes=superscript}
% ^^A \optitem{smallcapsheader}
% \changes{v0.9a}{2012/08/31}{New option \opt{smallcapsheader}} 
% ^^A $\rightarrow$ \opt{headerfont=smallcaps} 
% ^^A \optitem{viewmode}
% ^^A $\rightarrow$ \opt{version} 
% ^^A \end{ltxsyntax}
% ^^A \end{multicols}
%
% \defbibnote{note}{\sffamily This bibliography has been typeset with 
% the \sty{biblatex-philosophy} package, created by the same author of this class.}
%
% \addcontentsline{toc}{section}{\refname}
% \printbibliography[prenote=note]
%
% \StopEventually{\PrintChanges\PrintIndex}
%
% \section*{The Code}
% \addcontentsline{toc}{section}{The Code}
% \iffalse
%<*class>
% \fi
%    \begin{macrocode}
\ClassWarningNoLine{suftesi}{%
  ******************************************\MessageBreak
  * DO NOT MODIFY THE STYLES OF THIS CLASS\MessageBreak 
  * WITH PACKAGES AND/OR COMMANDS WHICH\MessageBreak 
  * MAY CHANGE THE LAYOUT OF THE DOCUMENT.\MessageBreak 
  * SEE DOCUMENTATION FOR DETAILS.\MessageBreak
  * ANYWAY, DON'T WORRY!\MessageBreak
  * THIS IS A HARMLESS MESSAGE :-)\MessageBreak
  ******************************************}
\RequirePackage{xkeyval}
\newif\ifsuftesi@compatibility
\newif\ifsuftesi@nofont 
\newif\ifsuftesi@greekfont 
\newif\ifsuftesi@standard
\newif\ifsuftesi@cochineal
\newif\ifsuftesi@libertine 
\newif\ifsuftesi@palatino 
\newif\ifsuftesi@porson 
\newif\ifsuftesi@artemisia
\newif\ifsuftesi@bodoni
\newif\ifsuftesi@cbgreek
\newif\ifsuftesi@mathminimal
\newif\ifsuftesi@mathextended
\newif\ifsuftesi@centerheader
\newif\ifsuftesi@sufplain
\newif\ifsuftesi@article
\newif\ifsuftesi@authortitle
\newif\ifsuftesi@periodical
\newif\ifsuftesi@periodicalaureo
\newif\ifsuftesi@compact
\newif\ifsuftesi@compactaureo
\newif\ifsuftesi@supercompact
\newif\ifsuftesi@supercompactaureo
\newif\ifsuftesi@screen
\newif\ifsuftesi@screencentered
\newif\ifsuftesi@dottedtoc
\newif\ifsuftesi@raggedtoc
\newif\ifsuftesi@numparpart
\newif\ifsuftesi@numparchap
\newif\ifsuftesi@numparsec
\newif\ifsuftesi@numparsubsec
\newif\ifsuftesi@numparsubsubsec
\newif\ifsuftesi@smallcapspart
\newif\ifsuftesi@smallcapschap
\newif\ifsuftesi@smallcapssec
\newif\ifsuftesi@draftdate
\newif\ifsuftesi@fewfootnotes
\newif\ifsuftesi@partpage
\newif\ifsuftesi@FSPL
\newif\ifsuftesi@pagelefttoc
\newif\ifsuftesi@twocolumntoc
\newif\ifsuftesi@twocolumnlof
\newif\ifsuftesi@twocolumnlot
\newif\ifsuftesi@reverseauthortitle
\newif\ifsuftesi@collection
\DeclareOption{a4paper}{%
  \ClassWarningNoLine{suftesi}{Option 'a4paper' not available}{}}
\DeclareOption{a5paper}{%
  \ClassWarningNoLine{suftesi}{Option 'a5paper' not available}{}}
\DeclareOption{b5paper}{%
  \ClassWarningNoLine{suftesi}{Option 'b5paper' not available}{}}
\DeclareOption{legalpaper}{%
  \ClassWarningNoLine{suftesi}{Option 'legalpaper' not available}{}}
\DeclareOption{executivepaper}{%
  \ClassWarningNoLine{suftesi}{Option 'executivepaper' not available}{}}
\DeclareOption{landscape}{%
  \ClassWarningNoLine{suftesi}{Option 'landscape' not available}{}}
%    \end{macrocode}
% The \sty{suftesi} class is based on the standard \sty{book} class but the previous options are disabled as they contrast with the layouts provided by the class.
%    \begin{macrocode}
\DeclareOption*{\PassOptionsToClass{\CurrentOption}{book}}   
\ProcessOptions
\relax
\LoadClass{book} 
%    \end{macrocode}
% \subsection*{Document structure}
%    \begin{macrocode}
\define@choicekey{}{structure}[\val\nr]
    {book,article,collection}[book]{%
\ifcase\nr\relax
\disable@keys{}{secnumstyle}
\or
\suftesi@articletrue
\@titlepagefalse
\or
\suftesi@collectiontrue
  \setkeys{}{headerstyle=authortitleinner}
\fi}
\define@choicekey{}{documentstructure}[\val\nr]
    {book,article,collection}[book]{%
\ClassWarningNoLine{suftesi}{%
  'documentstructure' option is deprecated.\MessageBreak
  Use 'structure' option instead}
\ifcase\nr\relax
  \setkeys{}{structure=book}
\or
  \setkeys{}{structure=article}
\or
  \setkeys{}{structure=collection}
\fi}
%    \end{macrocode}
% \subsection*{Page layout}
%    \begin{macrocode}
\RequirePackage{geometry}
\define@choicekey{}{pagelayout}[\val\nr]
    {standard,standardaureo,periodical,compact,compactaureo,supercompact,
    supercompactaureo,periodicalaureo}[standard]{%
\ifcase\nr\relax
  \DeclareRobustCommand{\SUF@chaptersize}{\Large}
  \DeclareRobustCommand{\SUF@sectionsize}{\large}
  \DeclareRobustCommand{\SUF@subsectionsize}{\normalsize}
  \DeclareRobustCommand{\SUF@subsubsectionsize}{\normalsize}
  \geometry{%
    heightrounded,
    a4paper,
    includeheadfoot=true,
    textwidth=      110mm,
    textheight=     220mm,
    marginratio=    2:3,
    marginparwidth= 30mm,
    marginparsep=   12pt}
\or
  \DeclareRobustCommand{\SUF@chaptersize}{\Large}
  \DeclareRobustCommand{\SUF@sectionsize}{\large}
  \DeclareRobustCommand{\SUF@subsectionsize}{\normalsize}
  \DeclareRobustCommand{\SUF@subsubsectionsize}{\normalsize}
  \geometry{%
    heightrounded,
    a4paper,
    includeheadfoot=true,
    textwidth=      136mm,
    textheight=     220mm,
    marginratio=    2:3,
    marginparwidth= 30mm,
    marginparsep=   12pt}
\or
\suftesi@periodicaltrue
  \DeclareRobustCommand{\SUF@chaptersize}{\large}
  \DeclareRobustCommand{\SUF@sectionsize}{\normalsize}
  \DeclareRobustCommand{\SUF@subsectionsize}{\normalsize}
  \DeclareRobustCommand{\SUF@subsubsectionsize}{\normalsize}
  \geometry{
    heightrounded,
    includeheadfoot=false,
    textheight=     165mm,
    textwidth=      110mm,
    paperwidth=     170mm,
    paperheight=    240mm,
    marginratio=    2:3,
    marginparwidth= 26mm,
    marginparsep=   10pt}
\or
\suftesi@compacttrue
  \DeclareRobustCommand{\SUF@chaptersize}{\large}
  \DeclareRobustCommand{\SUF@sectionsize}{\normalsize}
  \DeclareRobustCommand{\SUF@subsectionsize}{\normalsize}
  \DeclareRobustCommand{\SUF@subsubsectionsize}{\normalsize}
  \geometry{
    heightrounded,
    includeheadfoot=false,
    textheight=     165mm,
    textwidth=      110mm,
    paperwidth=     160mm,
    paperheight=    240mm,
    marginratio=    2:3,
    marginparwidth= 22mm,
    marginparsep=   9pt}
\or
\suftesi@compactaureotrue
  \DeclareRobustCommand{\SUF@chaptersize}{\large}
  \DeclareRobustCommand{\SUF@sectionsize}{\normalsize}
  \DeclareRobustCommand{\SUF@subsectionsize}{\normalsize}
  \DeclareRobustCommand{\SUF@subsubsectionsize}{\normalsize}
  \geometry{
    heightrounded,
    includeheadfoot=false,
    textheight=     191mm,
    textwidth=      118mm,
    paperwidth=     160mm,
    paperheight=    240mm,
    marginratio=    2:3,
    marginparwidth= 19mm,
    marginparsep=   9pt}
\or
\suftesi@supercompacttrue
  \DeclareRobustCommand{\SUF@chaptersize}{\large}
  \DeclareRobustCommand{\SUF@sectionsize}{\normalsize}
  \DeclareRobustCommand{\SUF@subsectionsize}{\normalsize}
  \DeclareRobustCommand{\SUF@subsubsectionsize}{\normalsize}
  \geometry{
    heightrounded,
    includeheadfoot=false,
    textheight=     150mm,
    textwidth=      100mm,
    paperwidth=     140mm,
    paperheight=    210mm,
    marginratio=    2:3,
    marginparwidth= 18mm,
    marginparsep=   8pt}
\or
\suftesi@supercompactaureotrue
  \DeclareRobustCommand{\SUF@chaptersize}{\large}
  \DeclareRobustCommand{\SUF@sectionsize}{\normalsize}
  \DeclareRobustCommand{\SUF@subsectionsize}{\normalsize}
  \DeclareRobustCommand{\SUF@subsubsectionsize}{\normalsize}
  \geometry{
    heightrounded,
    includeheadfoot=false,
    textheight=     175mm,
    textwidth=      108mm,
    paperwidth=     140mm,
    paperheight=    210mm,
    marginratio=    1:1,
    marginparwidth= 11mm,
    marginparsep=   7pt}
\or
\suftesi@periodicalaureotrue
  \DeclareRobustCommand{\SUF@chaptersize}{\large}
  \DeclareRobustCommand{\SUF@sectionsize}{\normalsize}
  \DeclareRobustCommand{\SUF@subsectionsize}{\normalsize}
  \DeclareRobustCommand{\SUF@subsubsectionsize}{\normalsize}
  \geometry{%
    heightrounded,
    includeheadfoot=true,
    textwidth=      120mm,
    textheight=     194mm,
    paperwidth=     17cm,
    paperheight=    24cm,
    marginratio=    2:3,
    marginparwidth= 62pt,
    marginparsep=   10pt}
\or
\fi}
%    \end{macrocode}
% \subsection*{Sections style}
%    \begin{macrocode}
\define@choicekey{}{partstyle}[\val\nr]{%
    left,center,right,parleft,parcenter,parright}[left]{%
\ifcase\nr\relax
  \def\SUF@lr@PARTSwitch{\filright}
  \DeclareRobustCommand{\xheadbreak}{\xheadbreakNL}
\or
  \def\SUF@lr@PARTSwitch{\filcenter}
  \DeclareRobustCommand{\xheadbreak}{\xheadbreakBB}
\or
  \def\SUF@lr@PARTSwitch{\filleft}
  \DeclareRobustCommand{\xheadbreak}{\xheadbreakBB}
\or%numpar
  \def\SUF@lr@PARTSwitch{\filright}
  \suftesi@numparparttrue
  \DeclareRobustCommand{\xheadbreak}{\xheadbreakNL}
\or
  \def\SUF@lr@PARTSwitch{\filcenter}
  \suftesi@numparparttrue
  \DeclareRobustCommand{\xheadbreak}{\xheadbreakBB}
\or
  \def\SUF@lr@PARTSwitch{\filleft}
  \suftesi@numparparttrue
  \DeclareRobustCommand{\xheadbreak}{\xheadbreakBB}
\fi}
\define@choicekey{}{chapstyle}[\val\nr]{%
    left,center,right,parleft,parcenter,parright}[left]{%
\ifcase\nr\relax
  \def\SUF@lr@CHAPSwitch{\filright}
  \DeclareRobustCommand{\xheadbreak}{\xheadbreakNL}
\or
  \def\SUF@lr@CHAPSwitch{\filcenter}
  \DeclareRobustCommand{\xheadbreak}{\xheadbreakBB}
\or
  \def\SUF@lr@CHAPSwitch{\filleft}
  \DeclareRobustCommand{\xheadbreak}{\xheadbreakBB}
\or%numparchap
  \def\SUF@lr@CHAPSwitch{\filright}
  \suftesi@numparchaptrue
  \DeclareRobustCommand{\xheadbreak}{\xheadbreakNL}
\or
  \def\SUF@lr@CHAPSwitch{\filcenter}
  \suftesi@numparchaptrue
  \DeclareRobustCommand{\xheadbreak}{\xheadbreakBB}
\or
  \def\SUF@lr@CHAPSwitch{\filleft}
  \suftesi@numparchaptrue
  \DeclareRobustCommand{\xheadbreak}{\xheadbreakBB}
\fi}
\define@choicekey{}{secstyle}[\val\nr]{%
    left,center,right,parleft,parcenter,parright}[left]{%
\ifcase\nr\relax
  \def\SUF@lr@SECSwitch{\filright}
  \DeclareRobustCommand{\xheadbreak}{\xheadbreakNL}
\or
  \def\SUF@lr@SECSwitch{\filcenter}
  \DeclareRobustCommand{\xheadbreak}{\xheadbreakBB}
\or
  \def\SUF@lr@SECSwitch{\filleft}
  \DeclareRobustCommand{\xheadbreak}{\xheadbreakBB}
\or%numparsec
  \def\SUF@lr@SECSwitch{\filright}
  \suftesi@numparsectrue
  \DeclareRobustCommand{\xheadbreak}{\xheadbreakNL}
\or
  \def\SUF@lr@SECSwitch{\filcenter}
  \suftesi@numparsectrue
  \DeclareRobustCommand{\xheadbreak}{\xheadbreakBB}
\or
  \def\SUF@lr@SECSwitch{\filleft}
  \suftesi@numparsectrue
  \DeclareRobustCommand{\xheadbreak}{\xheadbreakBB}
\fi}
\define@choicekey{}{subsecstyle}[\val\nr]{left,center,right}[left]{%
\ifcase\nr\relax
  \def\SUF@lr@SUBSECSwitch{\filright}
  \DeclareRobustCommand{\xheadbreak}{\xheadbreakNL}
\or
  \def\SUF@lr@SUBSECSwitch{\filcenter}
  \DeclareRobustCommand{\xheadbreak}{\xheadbreakBB}
\or
  \def\SUF@lr@SUBSECSwitch{\filleft}
  \DeclareRobustCommand{\xheadbreak}{\xheadbreakBB}
\fi}
\define@choicekey{}{subsubsecstyle}[\val\nr]{left,center,right}[left]{%
\ifcase\nr\relax
  \def\SUF@lr@SUBSUBSECSwitch{\filright}
  \DeclareRobustCommand{\xheadbreak}{\xheadbreakNL}
\or
  \def\SUF@lr@SUBSUBSECSwitch{\filcenter}
  \DeclareRobustCommand{\xheadbreak}{\xheadbreakBB}
\or
  \def\SUF@lr@SUBSUBSECSwitch{\filleft}
  \DeclareRobustCommand{\xheadbreak}{\xheadbreakBB}
\fi}
%    \end{macrocode}
% \subsection*{Sections font}
%    \begin{macrocode}
\define@choicekey{}{partfont}[\val\nr]{roman,italic,smallcaps}[roman]{%
\ifcase\nr\relax
\def\SUF@PART@StyleSwitch{\relax}
\or
\def\SUF@PART@StyleSwitch{\itshape}
\or
\def\SUF@PART@StyleSwitch{\expandafter\SUF@titlesmallcaps}
\fi}
\define@choicekey{}{chapfont}[\val\nr]{roman,italic,smallcaps}[roman]{%
\ifcase\nr\relax
\def\SUF@CHAP@StyleSwitch{\relax}
\or
\def\SUF@CHAP@StyleSwitch{\itshape}
\or
\def\SUF@CHAP@StyleSwitch{\expandafter\SUF@titlesmallcaps}
\fi}
\define@choicekey{}{secfont}[\val\nr]{roman,italic,smallcaps}[italic]{%
\ifcase\nr\relax
\def\SUF@SEC@StyleSwitch{\relax}
\or
\def\SUF@SEC@StyleSwitch{\itshape}
\or
\def\SUF@SEC@StyleSwitch{\expandafter\SUF@titlesmallcaps}
\fi}
\define@choicekey{}{subsecfont}[\val\nr]{roman,italic,smallcaps}[roman]{%
\ifcase\nr\relax
\def\SUF@SUBSEC@StyleSwitch{\relax}
\or
\def\SUF@SUBSEC@StyleSwitch{\itshape}
\or
\def\SUF@SUBSEC@StyleSwitch{\expandafter\SUF@titlesmallcaps}
\fi}
\define@choicekey{}{subsubsecfont}[\val\nr]{roman,italic,smallcaps}[roman]{%
\ifcase\nr\relax
\def\SUF@SUBSUBSEC@StyleSwitch{\relax}
\or
\def\SUF@SUBSUBSEC@StyleSwitch{\itshape}
\or
\def\SUF@SUBSUBSEC@StyleSwitch{\expandafter\SUF@titlesmallcaps}
\fi}
%    \end{macrocode}
% \subsection*{TOC font}
%    \begin{macrocode}
\define@choicekey{}{tocchapfont}[\val\nr]{roman,italic,smallcaps}[roman]{%
\ifcase\nr\relax
\def\SUF@tocCHAP@font{\relax}
\or
\def\SUF@tocCHAP@font{\itshape}
\or
\def\SUF@tocCHAP@font{\expandafter\SUF@TOCtitlesmallcaps}
\fi}
\define@choicekey{}{tocsecfont}[\val\nr]{roman,italic,smallcaps}[italic]{%
\ifcase\nr\relax
\def\SUF@tocSEC@font{\relax}
\or
\def\SUF@tocSEC@font{\itshape}
\or
\def\SUF@tocSEC@font{\expandafter\SUF@TOCtitlesmallcaps}
\fi}
\define@choicekey{}{tocsubsecfont}[\val\nr]{roman,italic,smallcaps}[roman]{%
\ifcase\nr\relax
\def\SUF@tocSUBSEC@font{\relax}
\or
\def\SUF@tocSUBSEC@font{\itshape}
\or
\def\SUF@tocSUBSEC@font{\expandafter\SUF@TOCtitlesmallcaps}
\fi}
\define@choicekey{}{tocsubsubsecfont}[\val\nr]{roman,italic,smallcaps}[roman]{%
\ifcase\nr\relax
\def\SUF@tocSUBSUBSEC@font{\relax}
\or
\def\SUF@tocSUBSUBSEC@font{\itshape}
\or
\def\SUF@tocSUBSUBSEC@font{\expandafter\SUF@TOCtitlesmallcaps}
\fi}
\define@choicekey{}{tocauthorfont}[\val\nr]{roman,italic,smallcaps}[roman]{%
\ifcase\nr\relax
\def\SUF@tocAUT@font{\relax}
\or
\def\SUF@tocAUT@font{\itshape}
\or
\def\SUF@tocAUT@font{\expandafter\SUF@TOCtitlesmallcaps}
\fi}
\define@choicekey{}{toctitlefont}[\val\nr]{roman,italic,smallcaps}[italic]{%
\ifcase\nr\relax
\def\SUF@tocTIT@font{\relax}
\or
\def\SUF@tocTIT@font{\itshape}
\or
\def\SUF@tocTIT@font{\expandafter\SUF@TOCtitlesmallcaps}
\fi}
%    \end{macrocode}
% \subsection*{Sections number style}
% \subsubsection*{Part number style}
%    \begin{macrocode}
\define@choicekey{}{partnumstyle}[\val\nr]
    {arabic,roman,Roman,dotarabic,dotroman,dotRoman}[arabic]{%
\ifcase\nr\relax
  \def\SUF@thepart{\arabic{part}}
  \def\SUF@dotpart{}
  \def\SUF@toclabelnum{}
\or
  \def\SUF@thepart{\textsc{\roman{part}}}
  \def\SUF@dotpart{}
  \def\SUF@toclabelnum{\scshape\@roman}
\or
  \def\SUF@thepart{\Roman{part}}
  \def\SUF@dotpart{}
  \def\SUF@toclabelnum{\@Roman}
\or
  \def\SUF@thepart{\arabic{part}}
  \def\SUF@dotpart{.}
  \def\SUF@toclabelnum{}
\or
  \def\SUF@thepart{\textsc{\roman{part}}}
  \def\SUF@dotpart{.}
  \def\SUF@toclabelnum{\scshape\@roman}
\or
  \def\SUF@thepart{\Roman{part}}
  \def\SUF@dotpart{.}
  \def\SUF@toclabelnum{\@Roman}
\fi}
%    \end{macrocode}
% \subsubsection*{Chapter number style}
%    \begin{macrocode}
\define@choicekey{}{chapnumstyle}[\val\nr]{arabic,roman,Roman,
dotarabic,dotroman,dotRoman}[arabic]{%
\ifcase\nr\relax
  \def\SUF@thechapter{\arabic{chapter}}
  \def\SUF@dotchap{}
  \def\SUF@toclabelnum{}
\or
  \def\SUF@thechapter{\textsc{\roman{chapter}}}
  \def\SUF@dotchap{}
  \def\SUF@toclabelnum{\scshape\@roman}
\or
  \def\SUF@thechapter{\Roman{chapter}}
  \def\SUF@dotchap{}
  \def\SUF@toclabelnum{\@Roman}
\or
  \def\SUF@thechapter{\arabic{chapter}}
  \def\SUF@dotchap{.}
  \def\SUF@toclabelnum{}
\or
  \def\SUF@thechapter{\textsc{\roman{chapter}}}
  \def\SUF@dotchap{.}
  \def\SUF@toclabelnum{\scshape\@roman}
\or
  \def\SUF@thechapter{\Roman{chapter}}
  \def\SUF@dotchap{.}
  \def\SUF@toclabelnum{\@Roman}
\fi}
%    \end{macrocode}
% \subsubsection*{Section number style}
%    \begin{macrocode}
\define@choicekey{}{secnumstyle}[\val\nr]
    {arabic,roman,Roman,dotarabic,dotroman,dotRoman}[arabic]{%
\ifcase\nr\relax
  \def\SUF@thesection{\arabic{section}}
  \def\SUF@dotsec{}
  \def\SUF@toclabelnum{}
\or
  \def\SUF@thesection{\textsc{\roman{section}}}
  \def\SUF@dotsec{}
  \def\SUF@toclabelnum{\scshape\@roman}
\or
  \def\SUF@thesection{\Roman{section}}
  \def\SUF@dotsec{}
  \def\SUF@toclabelnum{\@Roman}
\or
  \def\SUF@thesection{\arabic{section}}
  \def\SUF@dotsec{.}
  \def\SUF@toclabelnum{}
\or
  \def\SUF@thesection{\textsc{\roman{section}}}
  \def\SUF@dotsec{.}
  \def\SUF@toclabelnum{\scshape\@roman}
\or
  \def\SUF@thesection{\Roman{section}}
  \def\SUF@dotsec{.}
  \def\SUF@toclabelnum{\@Roman}
\fi}
%    \end{macrocode}
% This option controls the style of small capitals used in the 
% title of chapters and sections using \opt{\meta{level}font=smallcaps} option:
%    \begin{macrocode}
\define@choicekey{}{smallcapsstyle}[\val\nr]
    {low,upper}[low]{%
\ifcase\nr\relax
  \def\suftesi@MakeTextLowercase{\MakeLowercase}
  \def\suftesi@MakeTextTOCLowercase{\lowercase}
\or
  \def\suftesi@MakeTextLowercase{\relax}
  \def\suftesi@MakeTextTOCLowercase{\relax}
\fi}
%    \end{macrocode}
% \subsection*{Headers}
%
% \subsubsection*{Header style}
%    \begin{macrocode}
\define@choicekey{}{headerstyle}[\val\nr]
    {inner,center,plain,authortitleinner,
     authortitlecenter}[inner]{%
\ifcase\nr\relax
  \def\SUF@rightmark{\SUF@Rheadstyle{\rightmark}}
  \def\SUF@leftmark{\SUF@Lheadstyle{\SUF@leftrightmark}}
\or
  \suftesi@centerheadertrue
  \def\SUF@rightmark{\SUF@Rheadstyle{\rightmark}}
  \def\SUF@leftmark{\SUF@Lheadstyle{\SUF@leftrightmark}}
\or
  \suftesi@sufplaintrue
\or
  \suftesi@authortitletrue
  \def\SUF@rightmark{\let\thanks\@gobble\SUF@Rheadstyle{\@headtitle}}
  \def\SUF@leftmark{\let\thanks\@gobble\SUF@Lheadstyle{\@author}}
\or
  \suftesi@authortitletrue
  \def\SUF@rightmark{\let\thanks\@gobble\SUF@Rheadstyle{\@headtitle}}
  \def\SUF@leftmark{\let\thanks\@gobble\SUF@Lheadstyle{\@author}}
  \suftesi@centerheadertrue
\fi}
%    \end{macrocode}
% \subsubsection*{Header font}
%    \begin{macrocode}
\define@choicekey{}{headerfont}[\val\nr]
    {roman,italic,smallcaps}[roman]{%
\ifcase\nr\relax
  \def\SUF@Rheadstyle{}
  \def\SUF@Lheadstyle{}
  \def\SUF@thepage{\thepage}
\or
  \def\SUF@Rheadstyle{\itshape}
  \def\SUF@Lheadstyle{\itshape}
  \def\SUF@thepage{\thepage}
\or
  \def\SUF@Rheadstyle{\SUF@headingsmallcaps}
  \def\SUF@Lheadstyle{\SUF@headingsmallcaps}
  \def\SUF@thepage{\SUF@headingsmallcaps{\thepage}}
\fi}
%    \end{macrocode}
% \subsection*{Text elements}
% \subsubsection*{Lists}
%    \begin{macrocode}
\RequirePackage[inline]{enumitem}
\renewcommand\labelitemi{\color{sufgray}\textbullet}
\setlist{itemsep=.5ex,parsep=0pt,listparindent=\parindent}
\setlist[description]{font=\normalfont\itshape}
\define@choicekey{}{liststyle}[\val\nr]
    {bulged,aligned,indented}[bulged]{%
\ifcase\nr\relax
    \setlist[enumerate,1]{leftmargin=0pt,label=\arabic*.}
    \setlist[enumerate,2]{leftmargin= 1.3\parindent,label=\alph*.}
    \setlist[enumerate,3]{leftmargin= 1.3\parindent,label=\roman*.}
    \setlist[itemize,1]{leftmargin=0pt}
    \setlist[itemize,2]{leftmargin=1.3\parindent}
    \setlist[itemize,3]{leftmargin=1.3\parindent}
\or
    \setlist[enumerate,1]{leftmargin=1\parindent,label=\arabic*.}
    \setlist[enumerate,2]{leftmargin= 1.5\parindent,label=\alph*.}
    \setlist[enumerate,3]{leftmargin= 1.5\parindent,label=\roman*.}
    \setlist[itemize,1]{leftmargin=1\parindent}
    \setlist[itemize,2]{leftmargin=1.5\parindent}
    \setlist[itemize,3]{leftmargin=1.5\parindent}
\or
    \setlist[enumerate,1]{leftmargin=2\parindent,label=\arabic*.}
    \setlist[enumerate,2]{leftmargin= 2.5\parindent,label=\alph*.}
    \setlist[enumerate,3]{leftmargin= 2.5\parindent,label=\roman*.}
    \setlist[itemize,1]{leftmargin=2\parindent}
    \setlist[itemize,2]{leftmargin=2.5\parindent}
    \setlist[itemize,3]{leftmargin=2.5\parindent}
\fi}
%    \end{macrocode}
% \subsubsection*{Quotations}
%    \begin{macrocode}
\define@choicekey{}{quotestyle}[\val\nr]
    {center,right}[center]{%
\ifcase\nr\relax
\def\SUF@quote@style{\rightmargin=\parindent}
\or
\def\SUF@quote@style{}
\fi}
\define@choicekey{}{quotesize}[\val\nr]
    {footnotesize,small}[footnotesize]{%
\ifcase\nr\relax
\def\SUF@quotation@size{\footnotesize}
\or
\def\SUF@quotation@size{\small}
\fi}
%    \end{macrocode}
% \subsubsection*{Footnotes}
%    \begin{macrocode}
\define@choicekey{}{footnotestyle}[\val\nr]
    {bulged,hung,dotted,superscript}[bulged]{%
\ifcase\nr\relax
    \renewcommand\@makefntext{%
      \hskip-2.5em\makebox[2em][r]{\@thefnmark}\hskip.5em}
\or
    \renewcommand\@makefntext{%
      \leftskip=1em\hskip-1.5em%
        \makebox[1em][r]{\@thefnmark}\hskip.5em}
\or
    \renewcommand\@makefntext{%
         \@thefnmark.\hskip.5em}
\or
  \renewcommand\@makefntext{%
    \textsuperscript{\@thefnmark}\hskip.3em}
\fi}
\define@choicekey{}{fewfootnotes}[\val\nr]
    {true,false}[true]{%
\ifcase\nr\relax
  \suftesi@fewfootnotestrue
\or
  \relax
\fi}
%    \end{macrocode}
% \subsubsection*{Captions}
%    \begin{macrocode}
\RequirePackage{caption}
\define@choicekey{}{captionstyle}[\val\nr]
    {standard,sanserif,italic,smallcaps}[standard]{%
\ifcase\nr\relax
   \captionsetup{labelsep=period,font=small}
\or
   \captionsetup{labelsep=period,font=small}
   \captionsetup{font+=sf}
\or
   \captionsetup{labelsep=period,font=small}
   \captionsetup{font+=it}
\or
   \captionsetup{labelsep=period,font=small,labelfont=sc}
\fi
}
%    \end{macrocode}
% \subsubsection*{Marginal notes}
%    \begin{macrocode}
\define@choicekey{}{marginpar}[\val\nr]
    {true,false}[true]{%
\ifcase\nr\relax
\or   
  \renewcommand\marginpar[2][]{}
\fi}
%    \end{macrocode}
% \subsubsection*{Table of contents}
%    \begin{macrocode}
\define@choicekey{}{tocstyle}[\val\nr]{%
    standard,dotted,ragged,leftpage}[standard]{%
\ifcase\nr\relax
  \def\SUF@titlerule{\titlerule*{}}
  \def\SUF@chaptitlerule{\titlerule*{}}
\or
\suftesi@dottedtoctrue
  \def\SUF@titlerule{\titlerule*{\footnotesize .\ }}
  \def\SUF@chaptitlerule{\titlerule*{}}
\or
\suftesi@raggedtoctrue
  \def\SUF@titlerule{\hspace{1em}}
  \def\SUF@chaptitlerule{\hspace{1em}}
\or
\suftesi@pagelefttoctrue
\fi}
\define@choicekey{}{twocolcontents}[\val\nr]{%
    toc,lof,lot,toclof,toclot,loflot,all}[all]{%
\ifcase\nr\relax
  \suftesi@twocolumntoctrue
  \setkeys{}{tocstyle=ragged}
\or
  \suftesi@twocolumnloftrue
  \setkeys{}{tocstyle=ragged}
\or
  \suftesi@twocolumnlottrue
  \setkeys{}{tocstyle=ragged}
\or
  \suftesi@twocolumntoctrue
  \suftesi@twocolumnloftrue
  \setkeys{}{tocstyle=ragged}
\or
  \suftesi@twocolumntoctrue
  \suftesi@twocolumnlottrue
  \setkeys{}{tocstyle=ragged}
\or
  \suftesi@twocolumnloftrue
  \suftesi@twocolumnlottrue
  \setkeys{}{tocstyle=ragged}
\or
  \suftesi@twocolumntoctrue
  \suftesi@twocolumnloftrue
  \suftesi@twocolumnlottrue
  \setkeys{}{tocstyle=ragged}
\fi
}
%    \end{macrocode}
% \subsection*{Fonts}
% \subsubsection*{Roman fonts}
%    \begin{macrocode}
\define@choicekey{}{defaultfont}[\val\nr]{%
    none,cochineal,libertine,palatino,standard,compatibility}[cochineal]{%
\ifcase\nr\relax
  \suftesi@nofonttrue
\or
  \suftesi@cochinealtrue
\or
  \suftesi@libertinetrue
\or
  \suftesi@palatinotrue
\or
  \suftesi@standardtrue
\or
  \suftesi@compatibilitytrue
\fi}
\define@choicekey{}{mathfont}[\val\nr]{%
    none,minimal,extended}[minimal]{%
\ifcase\nr\relax
  \suftesi@mathminimalfalse
  \suftesi@mathextendedfalse 
\or
  \suftesi@mathminimaltrue
\or
  \suftesi@mathextendedtrue
\fi}
%    \end{macrocode}
% \subsubsection*{Greek fonts}
%    \begin{macrocode}
\define@choicekey{}{greekfont}[\val\nr]{%
    none,artemisia,porson,bodoni,cbgreek}[none]{%
\ifcase\nr\relax
\suftesi@greekfontfalse
\or
\suftesi@greekfonttrue\suftesi@artemisiatrue
\or
\suftesi@greekfonttrue\suftesi@porsontrue
\or
\suftesi@greekfonttrue\suftesi@bodonitrue
\or
\suftesi@greekfonttrue\suftesi@cbgreektrue
\fi}
%    \end{macrocode}
% \subsection*{Other options}
% \subsubsection*{Indentation}
%    \begin{macrocode}
\define@choicekey{}{parindent}[\val\nr]
    {compact,wide}[compact]{%
\ifcase\nr\relax
\setlength\parindent{1em}
\or
\setlength\parindent{1.5em}
\fi}
%    \end{macrocode}
% \subsubsection*{Part page}
%    \begin{macrocode}
\define@choicekey{}{partpage}[\val\nr]{true,false}[true]{%
\ifcase\nr\relax
  \suftesi@partpagetrue
\or
  \relax
\fi}
%    \end{macrocode}
% \subsubsection*{Draftdate}
%    \begin{macrocode}
\define@choicekey{}{draftdate}[\val\nr]
    {true,false}[true]{%
\ifcase\nr\relax
  \suftesi@draftdatetrue
\or
\fi}
%    \end{macrocode}
% \subsubsection*{View mode}
%    \begin{macrocode}
\define@choicekey{}{version}[\val\nr]
    {screen,cscreen,draft,final}[draft]{%
\ifcase\nr\relax
  \suftesi@screentrue
\or
  \suftesi@screentrue
  \suftesi@screencenteredtrue
\or
\or
  \AtBeginDocument{\crop[off]}
\fi}
\define@choicekey{}{crop}[\val\nr]
    {true,false}[true]{%
\ClassWarningNoLine{suftesi}{%
  'crop' option is deprecated.\MessageBreak
  Use 'version' option instead}
\ifcase\nr\relax
  \setkeys{}{version=draft}
\or
  \setkeys{}{version=screen}
\fi}
%    \end{macrocode}
% \subsubsection*{Titlepage}
%    \begin{macrocode}
\define@choicekey{}{titlepage}[\val\nr]
    {true,false}[true]{%
\ifcase\nr\relax
  \@titlepagetrue
\or
  \@titlepagefalse
\fi}
%    \end{macrocode}
% \subsubsection*{Pre-defined styles}
%    \begin{macrocode}
\define@choicekey{}{style}[\val\nr]
  {roman1,roman2,roman3,roman4,roman5,roman6,
  italic1,italic2,italic3,italic4,italic5,italic6,
  italic7,italic8,italic9,italic10,italic11,italic12,
  smallcaps1,smallcaps2,smallcaps3,smallcaps4,smallcaps5,smallcaps6,
  smallcaps7,smallcaps8,smallcaps9,smallcaps10,smallcaps11,smallcaps12,
  FSPLa,FSPLb,FSPLc}  
  [roman1]{%
\ifcase\nr\relax
%    \end{macrocode}
% \paragraph{The `roman' styles}
%    \begin{macrocode}
  \setkeys{}{chapstyle=left}
\or
  \setkeys{}{chapstyle=center}
\or
  \setkeys{}{chapstyle=right}
\or
  \setkeys{}{chapstyle=parleft}
\or
  \setkeys{}{chapstyle=parcenter}
\or
  \setkeys{}{chapstyle=parright}
\or
%    \end{macrocode}
% \paragraph{The `italic' styles}
%    \begin{macrocode}
  \setkeys{}{
   chapstyle=left,
   chapfont=italic,
   tocchapfont=italic,
   headerfont=italic}
\or
  \setkeys{}{
   chapstyle=center,
   chapfont=italic,
   tocchapfont=italic,
   headerfont=italic}
\or
  \setkeys{}{
   chapstyle=right,
   chapfont=italic,
   tocchapfont=italic,
   headerfont=italic}
\or
  \setkeys{}{
   chapstyle=parleft,
   chapfont=italic,
   tocchapfont=italic,
   headerfont=italic}
\or
  \setkeys{}{
   chapstyle=parcenter,
   chapfont=italic,
   tocchapfont=italic,
   headerfont=italic}
\or
  \setkeys{}{
   chapstyle=parright,
   chapfont=italic,
   tocchapfont=italic,
   headerfont=italic}
\or
  \setkeys{}{
   chapstyle=left,
   chapfont=italic,
   tocchapfont=italic,
   secfont=smallcaps,
   headerfont=italic}
\or
  \setkeys{}{
   chapstyle=center,
   chapfont=italic,
   tocchapfont=italic,
   secfont=smallcaps,
   headerfont=italic}
\or
  \setkeys{}{
   chapstyle=right,
   chapfont=italic,
   tocchapfont=italic,
   secfont=smallcaps,
   headerfont=italic}
\or
  \setkeys{}{
   chapstyle=parleft,
   chapfont=italic,
   tocchapfont=italic,
   secfont=smallcaps,
   headerfont=italic}
\or
  \setkeys{}{
   chapstyle=parcenter,
   chapfont=italic,
   tocchapfont=italic,
   secfont=smallcaps,
   headerfont=italic}
\or
  \setkeys{}{
   chapstyle=parright,
   chapfont=italic,
   tocchapfont=italic,
   secfont=smallcaps,
   headerfont=italic}
\or
%    \end{macrocode}
% \paragraph{The smallcaps' styles}
%    \begin{macrocode}
  \setkeys{}{
  chapstyle=left,
  chapfont=smallcaps,
  tocchapfont=smallcaps,
  headerfont=smallcaps}
\or
  \setkeys{}{
  chapstyle=center,
  chapfont=smallcaps,
  tocchapfont=smallcaps,
  headerfont=smallcaps}
\or
  \setkeys{}{
  chapstyle=right,
  chapfont=smallcaps,
  tocchapfont=smallcaps,
  headerfont=smallcaps}
\or
  \setkeys{}{
  chapstyle=parleft,
  chapfont=smallcaps,
  tocchapfont=smallcaps,
  headerfont=smallcaps}
\or
  \setkeys{}{
  chapstyle=parcenter,
  chapfont=smallcaps,
  tocchapfont=smallcaps,
  headerfont=smallcaps}
\or
  \setkeys{}{
  chapstyle=parright,
  chapfont=smallcaps,
  tocchapfont=smallcaps,
  headerfont=smallcaps}
\or
  \setkeys{}{
  chapstyle=left,
  chapfont=smallcaps,
  tocchapfont=smallcaps,
  secfont=smallcaps,
  headerfont=smallcaps}
\or
  \setkeys{}{
  chapstyle=center,
  chapfont=smallcaps,
  tocchapfont=smallcaps,
  secfont=smallcaps,
  headerfont=smallcaps}
\or
  \setkeys{}{
  chapstyle=right,
  chapfont=smallcaps,
  tocchapfont=smallcaps,
  secfont=smallcaps,
  headerfont=smallcaps}
\or
  \setkeys{}{
  chapstyle=parleft,
  chapfont=smallcaps,
  tocchapfont=smallcaps,
  secfont=smallcaps,
  headerfont=smallcaps}
\or
  \setkeys{}{
  chapstyle=parcenter,
  chapfont=smallcaps,
  tocchapfont=smallcaps,
  secfont=smallcaps,
  headerfont=smallcaps}
\or
  \setkeys{}{
  chapstyle=parright,
  chapfont=smallcaps,
  tocchapfont=smallcaps,
  secfont=smallcaps,
  headerfont=smallcaps}
\or
%    \end{macrocode}
% \paragraph{The \opt{FSPL} styles}
%    \begin{macrocode}
\suftesi@periodicalaureotrue
\suftesi@FSPLtrue
  \setkeys{}{%
    pagelayout=periodicalaureo,
    style=roman5,
    chapnumstyle=roman,
    headerstyle=inner,
    footnotestyle=hung,
    liststyle=indented,
    tocstyle=leftpage}
\disable@keys{}
{structure,documentstructure,pagelayout,partfont,chapfont,secfont,%
subsecfont,subsubsecfont,partstyle,chapstyle,secstyle,%
subsecstyle,subsubsecstyle,partnumstyle,chapnumstyle,%
secnumstyle,tocstyle,headerstyle,headerfont,quotestyle,%
quotesize,footnotestyle,liststyle,captionstyle,parindent}
\or
\suftesi@periodicalaureotrue
\suftesi@FSPLtrue
  \setkeys{}{%  
    pagelayout=periodicalaureo,
    style=italic5,
    chapnumstyle=roman,
    headerstyle=inner,
    headerfont=italic,
    footnotestyle=hung,
    liststyle=indented,
    tocstyle=leftpage}
\disable@keys{}
{structure,documentstructure,pagelayout,partfont,chapfont,secfont,%
subsecfont,subsubsecfont,partstyle,chapstyle,secstyle,%
subsecstyle,subsubsecstyle,partnumstyle,chapnumstyle,%
secnumstyle,tocstyle,headerstyle,headerfont,quotestyle,%
quotesize,footnotestyle,liststyle,captionstyle,parindent}
\or
\suftesi@periodicalaureotrue
\suftesi@FSPLtrue
  \setkeys{}{%
    pagelayout=periodicalaureo,
    style=smallcaps5,
    chapnumstyle=roman,
    headerstyle=inner,
    headerfont=smallcaps,
    secfont=smallcaps,
    footnotestyle=hung,
    liststyle=indented,
    tocstyle=leftpage}
\disable@keys{}
{structure,documentstructure,pagelayout,partfont,chapfont,secfont,%
subsecfont,subsubsecfont,partstyle,chapstyle,secstyle,%
subsecstyle,subsubsecstyle,partnumstyle,chapnumstyle,%
secnumstyle,tocstyle,headerstyle,headerfont,quotestyle,%
quotesize,footnotestyle,liststyle,captionstyle,parindent}
\fi}
%    \end{macrocode}
% \subsubsection*{Options for \opt{collection} document structure}
%    \begin{macrocode}
\define@choicekey{}{papertitlestyle}[\val\nr]{%
    left,center,right}[left]{%
\ifcase\nr\relax
  \def\SUF@lr@coll@titleSwitch{\filright}
\or
  \def\SUF@lr@coll@titleSwitch{\filcenter}
\or
  \def\SUF@lr@coll@titleSwitch{\filleft}
\fi}
\define@choicekey{}{revauthortitle}[\val\nr]
    {true,false}[false]{%
\ifcase\nr\relax
\suftesi@reverseauthortitletrue
\or\relax
\fi}
\define@choicekey{}{titlefont}[\val\nr]{roman,italic,smallcaps}[roman]{%
\ifcase\nr\relax
\def\SUF@titlefont@Switch{\relax}
\or
\def\SUF@titlefont@Switch{\itshape}
\or
\def\SUF@titlefont@Switch{\expandafter\SUF@titlesmallcaps}
\fi}

\define@choicekey{}{authorfont}[\val\nr]{roman,italic,smallcaps}[roman]{%
\ifcase\nr\relax
\def\SUF@authorfont@Switch{\relax}
\or
\def\SUF@authorfont@Switch{\itshape}
\or
\def\SUF@authorfont@Switch{\expandafter\SUF@titlesmallcaps}
\fi}
%    \end{macrocode}
% \subsection*{Execute and process options}
%    \begin{macrocode}
\ExecuteOptionsX<>{
  captionstyle=standard,
  marginpar=true,
  parindent=compact,
  tocstyle=standard,
  defaultfont=cochineal,
  mathfont=minimal,
  greekfont=none,
  pagelayout=standard,
  headerstyle=inner,
  liststyle=bulged,
  footnotestyle=bulged,
  quotesize=footnotesize,
  quotestyle=center,
  partstyle=left,
  chapstyle=left,
  secstyle=left,
  subsecstyle=left,
  subsubsecstyle=left,
  partfont=roman,
  chapfont=roman,
  secfont=italic,
  subsecfont=roman,
  subsubsecfont=roman,
  headerfont=roman,
  secnumstyle=arabic,
  partnumstyle=Roman,
  chapnumstyle=arabic,
  smallcapsstyle=low,
  titlefont=italic,
  authorfont=roman,
  toctitlefont=italic,
  tocauthorfont=roman,
  revauthortitle=false,
  papertitlestyle=left,
  authorfont=roman,
  titlefont=italic,
  tocchapfont=roman,
  tocsecfont=roman,
  tocsubsecfont=roman,
  tocsubsubsecfont=roman}
\ProcessOptionsX<>\relax
%    \end{macrocode}
% The use of the \cmd{XKV@useoption} command, provided by 
% the \sty{xkeyval}, is a trick to delete the option given as 
% its argument from the list stored in \cmd{@unusedoptionlist} 
% so that the option will not produce the ``Unused global options''
% message:
%    \begin{macrocode}
\XKV@useoption{a4paper}   \XKV@useoption{10pt}      
\XKV@useoption{11pt}      \XKV@useoption{12pt}
\XKV@useoption{oneside}   \XKV@useoption{twoside}
\XKV@useoption{onecolumn} \XKV@useoption{twocolumn}
\XKV@useoption{titlepage} \XKV@useoption{notitlepage}
\XKV@useoption{openright} \XKV@useoption{openany}
\XKV@useoption{leqno}     \XKV@useoption{fleqn}
\XKV@useoption{a5paper}   \XKV@useoption{b5paper}
\XKV@useoption{legalpaper}\XKV@useoption{executivepaper}
\XKV@useoption{landscape}
%    \end{macrocode}
% A trick to delete the class options from \cmd{XKV@classoptionslist},
% in order to avoid incompatibility with packages using options 
% defined by \sty{suftesi} too. Thanks to Enrico Gregorio!
%    \begin{macrocode}
\def\XKV@classoptionslist{}
%    \end{macrocode}
% \subsection*{Basic packages}
%    \begin{macrocode}
\RequirePackage{color}
    \definecolor{sufred}{rgb}{0.5,0,0}
    \definecolor{sufgray}{rgb}{0.5,0.5,0.5}
\RequirePackage{multicol}
\RequirePackage{emptypage}
\RequirePackage{iftex}
\RequirePackage{microtype}
%    \end{macrocode}
% The FSPL style requires a verbose colophon which uses specific icons:
%    \begin{macrocode}
\ifsuftesi@FSPL
  \RequirePackage{cclicenses}
 \else
\fi
%    \end{macrocode}
% \subsection*{Page layout}
% The \cmd{geometry} command must be declare before the \sty{fontenc} package. If declared after it, the \opt{heightrounded} option becomes ineffective and many underfull vertical boxes may occur. 
%    \begin{macrocode}
\ifsuftesi@screen
  \newcommand*\crop[1][]{}
    \ifsuftesi@screencentered
      \geometry{hmarginratio=1:1}
    \else
  \fi
\else
  \RequirePackage[a4,cam,center]{crop}
\fi
%    \end{macrocode}
%
% \subsection*{Fonts}
%
% With \XeTeX{} we do not load any fonts. Anyway the \sty{fontspec}
% package is required because some commands of the class are base on it. 
%    \begin{macrocode}
\unless\ifPDFTeX%if xelatex or lualatex
\AtBeginDocument{%
\@ifpackageloaded{fontspec}
  {\relax}
  {\ClassError{suftesi}
    {***********************************\MessageBreak
    * For using suftesi with XeLaTeX\MessageBreak
    * load either 'fontspec' or 'mathspec'\MessageBreak
    * For using suftesi with LuaLaTeX\MessageBreak
    * load 'fontspec' \MessageBreak
    ************************************}
    {***********************************\MessageBreak
    * For using suftesi with XeLaTeX\MessageBreak
    * load either 'fontspec' or 'mathspec'\MessageBreak
    * For using suftesi with LuaLaTeX\MessageBreak
    * load 'fontspec' \MessageBreak
    ************************************}}
  }
\else% if pdftex
\RequirePackage[LGR,T1]{fontenc}
\RequirePackage{substitutefont}
\AtBeginDocument{\fontencoding{T1}\selectfont}
%    \end{macrocode}
% Now we load the macros for the \sty{defaultfont} option. 
% The greek fonts by the Greek Font Society are available 
% only with \opt{palatino}, \opt{libertine} and \opt{cochineal} options.
% A warning appears if the \opt{cbgreek} option is unused.
%    \begin{macrocode}
\ifsuftesi@nofont
    \ifsuftesi@greekfont
      \ClassWarningNoLine{suftesi}{%
        Unused 'greekfont' option}%
        \suftesi@greekfontfalse  
          \else\suftesi@greekfontfalse\fi     
\else
\ifsuftesi@standard
  \RequirePackage{lmodern}
    \ifsuftesi@greekfont
      \ClassWarningNoLine{suftesi}{%
        Unused 'greekfont' option}%
        \suftesi@greekfontfalse  
          \else\suftesi@greekfontfalse\fi     
\else
%    \end{macrocode}
% Previous versions of the class were based on 
% Palatino (\sty{mathpazo}), Iwona and Bera Mono. 
% Now this combination is provided only 
% for backward compatibility.
%    \begin{macrocode}
  \ifsuftesi@compatibility
      \RequirePackage[osf,sc]{mathpazo}
      \RequirePackage[scaled=0.8]{beramono}
        \renewcommand{\sfdefault}{iwona}
      \ifsuftesi@greekfont
      \ifsuftesi@bodoni
      \gdef\SUFfntscale{0.96}
      \else
      \ifsuftesi@artemisia
      \gdef\SUFfntscale{0.97}
      \else
      \ifsuftesi@porson
      \gdef\SUFfntscale{1.18}
      \else
      \ifsuftesi@cbgreek
        \def\lmfntscale{1.06}
      \else
      \fi 
      \fi 
      \fi 
      \fi 
      \else
      \ClassWarningNoLine{suftesi}{\MessageBreak
      If you need the Greek font remember\MessageBreak
      to set one of the following options:\MessageBreak
      greekfont=artemisia,\MessageBreak
      greekfont=porson,\MessageBreak
      greekfont=cbgreek}
      \fi
\else
  \ifsuftesi@palatino
    \RequirePackage[full]{textcomp}
         \RequirePackage{newpxtext}
         \RequirePackage[scaled=1.06]{biolinum}
         \RequirePackage[varqu,varl]{inconsolata}
         \ifsuftesi@mathextended
         \RequirePackage{amsthm}
         \RequirePackage[bigdelims,vvarbb]{newpxmath}
         \RequirePackage[cal=boondoxo]{mathalfa} 
         \else
         \ifsuftesi@mathminimal
         \RequirePackage[bigdelims,vvarbb]{newpxmath}
         \else
         \fi
         \fi
         \useosf
         \useproportional
      \ifsuftesi@greekfont
      \ifsuftesi@bodoni
      \gdef\SUFfntscale{0.96}
      \else
      \ifsuftesi@artemisia
      \gdef\SUFfntscale{0.97}
      \else
      \ifsuftesi@porson
      \gdef\SUFfntscale{1.18}
      \else
      \ifsuftesi@cbgreek
        \def\lmfntscale{1.06}
      \else
      \fi 
      \fi 
      \fi 
      \fi 
      \else
      \ClassWarningNoLine{suftesi}{\MessageBreak
      If you need the Greek font remember\MessageBreak
      to set one of the following options:\MessageBreak
      greekfont=artemisia,\MessageBreak
      greekfont=porson,\MessageBreak
      greekfont=cbgreek}
      \fi
\else
  \ifsuftesi@libertine
    \RequirePackage{textcomp}
         \RequirePackage[sb]{libertine}
         \RequirePackage[varqu,varl,scaled=0.94]{inconsolata}
         \ifsuftesi@mathextended
         \RequirePackage{amsthm}
         \RequirePackage[libertine,bigdelims,vvarbb]{newtxmath}
         \RequirePackage[cal=boondoxo]{mathalfa} 
         \else
         \ifsuftesi@mathminimal
         \RequirePackage[libertine,bigdelims,vvarbb]{newtxmath}
         \else
         \fi
         \fi
         \useosf
      \ifsuftesi@bodoni
      \gdef\SUFfntscale{0.9}
      \else
      \ifsuftesi@artemisia
      \gdef\SUFfntscale{0.91}
      \else
      \ifsuftesi@porson
      \gdef\SUFfntscale{1.1}
      \else
      \fi 
      \fi 
      \fi 
\else
  \ifsuftesi@cochineal
    \RequirePackage{textcomp}
        \RequirePackage{cochineal}
        \RequirePackage[varqu,varl,var0]{inconsolata}
        \RequirePackage{biolinum}
        \ifsuftesi@mathextended
        \RequirePackage{amsthm}
        \RequirePackage[cochineal,bigdelims,cmintegrals,vvarbb]{newtxmath}
        \RequirePackage[cal=boondoxo]{mathalfa}
        \else
        \ifsuftesi@mathminimal
        \RequirePackage[cochineal,bigdelims,cmintegrals,vvarbb]{newtxmath}
        \else
        \fi
        \fi
        \useosf
        \useproportional
      \ifsuftesi@bodoni
      \gdef\SUFfntscale{0.9}
      \else
      \ifsuftesi@artemisia
      \gdef\SUFfntscale{0.905}
      \else
      \ifsuftesi@porson
      \gdef\SUFfntscale{1.1}
      \else
      \fi 
      \fi 
      \fi 
    \else
    \fi
    \fi
    \fi
    \fi
    \fi
    \fi
\fi
%    \end{macrocode}
%
%    \begin{macrocode}
\unless\ifPDFTeX\else
\ifsuftesi@greekfont
    \ifsuftesi@artemisia
      \def\SUF@greekfamily{artemisia}
        \else
          \ifsuftesi@porson
            \def\SUF@greekfamily{porson}
                 \else
                \ifsuftesi@bodoni
              \def\SUF@greekfamily{bodoni}
              \else
                \ifsuftesi@cbgreek
               \def\SUF@greekfamily{lmr}
            \else
          \fi
        \fi
      \fi
    \fi
%    \end{macrocode}
% Thanks to Claudio Beccari for the following macro.
%    \begin{macrocode}
  \AtBeginDocument{
    \substitutefont{LGR}{\rmdefault}{\SUF@greekfamily}
        \DeclareRobustCommand{\greektext}{%
           \usefont{LGR}{\SUF@greekfamily}{\f@series}{\f@shape}
           \def\encodingdefault{LGR}}
        \DeclareTextFontCommand{\textgreek}{\greektext}}
\ifsuftesi@cbgreek
   \else
%    \end{macrocode}
% We redefine the font definitions of the GFS fonts 
% in order to scale the fonts according to the roman default.
%    \begin{macrocode}
\DeclareFontFamily{LGR}{bodoni}{}
\DeclareFontShape{LGR}{bodoni}{m}{n}{<-> s * [\SUFfntscale] gbodonirg6a}{}
\DeclareFontShape{LGR}{bodoni}{m}{it}{<-> s * [\SUFfntscale] gbodonii6a}{}
\DeclareFontShape{LGR}{bodoni}{b}{n}{<-> s * [\SUFfntscale] gbodonib6a}{}
\DeclareFontShape{LGR}{bodoni}{b}{it}{<-> s * [\SUFfntscale] gbodonibi6a}{}
\DeclareFontShape{LGR}{bodoni}{m}{sl}{<-> s * [\SUFfntscale] gbodonio6a}{}
\DeclareFontShape{LGR}{bodoni}{b}{sl}{<-> s * [\SUFfntscale] gbodonibo6a}{}
\DeclareFontShape{LGR}{bodoni}{m}{sc}{<-> s * [\SUFfntscale] gbodonisc6a}{}
\DeclareFontShape{LGR}{bodoni}{m}{sco}{<-> s * [\SUFfntscale] gbodonisco6a}{}

\DeclareFontShape{LGR}{bodoni}{bx}{n}{<-> s * [\SUFfntscale] gbodonib6a}{}
\DeclareFontShape{LGR}{bodoni}{bx}{it}{<-> s * [\SUFfntscale] gbodonibi6a}{}
\DeclareFontShape{LGR}{bodoni}{bx}{sl}{<-> s * [\SUFfntscale] gbodonibo6a}{}

\DeclareFontFamily{LGR}{artemisia}{}
\DeclareFontShape{LGR}{artemisia}{m}{n}{<-> s * [\SUFfntscale] gartemisiarg6a}{}
\DeclareFontShape{LGR}{artemisia}{m}{it}{<-> s * [\SUFfntscale] gartemisiai6a}{}
\DeclareFontShape{LGR}{artemisia}{b}{n}{<-> s * [\SUFfntscale] gartemisiab6a}{}
\DeclareFontShape{LGR}{artemisia}{b}{it}{<-> s * [\SUFfntscale] gartemisiabi6a}{}
\DeclareFontShape{LGR}{artemisia}{m}{sl}{<-> s * [\SUFfntscale] gartemisiao6a}{}
\DeclareFontShape{LGR}{artemisia}{b}{sl}{<-> s * [\SUFfntscale] gartemisiabo6a}{}
\DeclareFontShape{LGR}{artemisia}{m}{sc}{<-> s * [\SUFfntscale] gartemisiasc6a}{}
\DeclareFontShape{LGR}{artemisia}{m}{sco}{<-> s * [\SUFfntscale] gartemisiasco6a}{}

\DeclareFontShape{LGR}{artemisia}{bx}{n}{<-> s * [\SUFfntscale] gartemisiab6a}{}
\DeclareFontShape{LGR}{artemisia}{bx}{it}{<-> s * [\SUFfntscale] gartemisiabi6a}{}
\DeclareFontShape{LGR}{artemisia}{bx}{sl}{<-> s * [\SUFfntscale] gartemisiabo6a}{}

\DeclareFontFamily{LGR}{porson}{}
\DeclareFontShape{LGR}{porson}{m}{n}{<-> s * [\SUFfntscale] gporsonrg6a}{}
\DeclareFontShape{LGR}{porson}{m}{it}{<-> s * [\SUFfntscale] gporsonrg6a}{}
\DeclareFontShape{LGR}{porson}{b}{n}{<-> s * [\SUFfntscale] gporsonrg6a}{}
\DeclareFontShape{LGR}{porson}{b}{it}{<-> s * [\SUFfntscale] gporsonrg6a}{}
\DeclareFontShape{LGR}{porson}{m}{sl}{<-> s * [\SUFfntscale] gporsonrg6a}{}
\DeclareFontShape{LGR}{porson}{b}{sl}{<-> s * [\SUFfntscale] gporsonrg6a}{}
\DeclareFontShape{LGR}{porson}{m}{sc}{<-> s * [\SUFfntscale] gporsonrg6a}{}

\DeclareFontShape{LGR}{porson}{bx}{n}{<-> s * [\SUFfntscale] gporsonrg6a}{}
\DeclareFontShape{LGR}{porson}{bx}{it}{<-> s * [\SUFfntscale] gporsonrg6a}{}
\DeclareFontShape{LGR}{porson}{bx}{sl}{<-> s * [\SUFfntscale] gporsonrg6a}{}
\fi
\else
\fi
\fi
%    \end{macrocode}
% \subsection*{Section titles}
%    \begin{macrocode}
\RequirePackage{titlesec}
    \newlength{\sectionsep}
    \setlength{\sectionsep}{\dimexpr(\baselineskip) plus 1pt minus 1pt}
\unless\ifPDFTeX% if xetex or luatex
  \DeclareRobustCommand{\SUF@TOCtitlesmallcaps}[1]{%
    \addfontfeature{LetterSpace=10}\scshape\suftesi@MakeTextTOCLowercase{#1}}
  \DeclareRobustCommand{\SUF@titlesmallcaps}[1]{%
    \addfontfeature{LetterSpace=10}\scshape\suftesi@MakeTextLowercase{#1}}
  \DeclareRobustCommand{\SUF@headingsmallcaps}[1]{%
    \addfontfeature{LetterSpace=10}\scshape\suftesi@MakeTextLowercase{#1}}
\else% if pdftex
  \DeclareRobustCommand{\SUF@TOCtitlesmallcaps}[1]{%
    \scshape\suftesi@MakeTextTOCLowercase{\textls*{#1}}}%
  \DeclareRobustCommand{\SUF@titlesmallcaps}[1]{%
    \scshape\suftesi@MakeTextLowercase{\textls*{#1}}}%
  \DeclareRobustCommand{\SUF@headingsmallcaps}[1]{%
    \scshape\suftesi@MakeTextLowercase{\textls*{#1}}}%
\fi
%    \end{macrocode}
% Default styles:
%    \begin{macrocode}
\newlength\SUF@ADJnumparchap
\def\SUF@default@CHAPTER{
\ifsuftesi@numparchap
  \titleformat{\chapter}[display]
    {\SUF@chaptersize\SUF@lr@CHAPSwitch}
    {\SUF@thechapter\SUF@dotchap}
    {2ex}
    {\SUF@CHAP@StyleSwitch}
  \else
  \titleformat{\chapter}[hang]
    {\SUF@chaptersize\SUF@lr@CHAPSwitch}
    {\SUF@thechapter\SUF@dotchap}
    {3em}
    {\SUF@CHAP@StyleSwitch}
  \fi
\ifsuftesi@numparchap
  \setlength{\SUF@ADJnumparchap}{-2.5\baselineskip}
    \else
  \setlength{\SUF@ADJnumparchap}{0pt}
\fi}
%    \end{macrocode}
%    \begin{macrocode}
\def\SUF@default@SECTIONS{
\ifsuftesi@numparsec
\titleformat{\section}[display]
	  {\SUF@sectionsize\SUF@lr@SECSwitch}
	  {\ifsuftesi@article\SUF@thesection
	    \else\thesection\fi\SUF@dotsec}
	  {0ex}
	  {\SUF@SEC@StyleSwitch}
	\titlespacing*{\section}{0em}{\sectionsep}{\sectionsep}[0em]	 
\else
\titleformat{\section}[hang]
	  {\SUF@sectionsize\SUF@lr@SECSwitch}
	  {\ifsuftesi@article\SUF@thesection
	    \else\thesection\fi\SUF@dotsec}
	  {1em}
	  {\SUF@SEC@StyleSwitch}
	\titlespacing*{\section}{0ex}{\sectionsep}{\sectionsep}[0ex]	  
\fi
%    \end{macrocode}
%    \begin{macrocode}
\ifsuftesi@numparsubsec
\titleformat{\subsection}[display]
	  {\SUF@subsectionsize\SUF@lr@SUBSECSwitch}
	  {\textnormal\thesubsection}
	  {0ex}
	  {\SUF@SUBSEC@StyleSwitch}
	\titlespacing*{\subsection}{0em}{\sectionsep}{\sectionsep}[0em]
\else
\titleformat{\subsection}[hang]
	  {\SUF@subsectionsize\SUF@lr@SUBSECSwitch}
	  {\textnormal\thesubsection}
	  {1em}
	  {\SUF@SUBSEC@StyleSwitch}
	\titlespacing*{\subsection}{\parindent}{\sectionsep}{\sectionsep}[0ex]
\fi
%    \end{macrocode}
%    \begin{macrocode}
\ifsuftesi@numparsubsubsec
\titleformat{\subsubsection}[display]
	  {\SUF@subsectionsize\SUF@lr@SUBSUBSECSwitch}
	  {\textnormal\thesubsubsection}
	  {0ex}
	  {\SUF@SUBSUBSEC@StyleSwitch}
	\titlespacing*{\subsubsection}{0em}{\sectionsep}{\sectionsep}[0em]  
\else
\titleformat{\subsubsection}[hang]
	  {\SUF@subsectionsize\SUF@lr@SUBSUBSECSwitch}
	  {\textnormal\thesubsubsection}
	  {1em}
	  {\SUF@SUBSUBSEC@StyleSwitch}
	\titlespacing*{\subsubsection}{\parindent}{\sectionsep}{\sectionsep}[0ex]  
  \fi
}
\SUF@default@CHAPTER
\SUF@default@SECTIONS
\ifsuftesi@periodical
  \titlespacing*{\chapter}{0em}{0\SUF@ADJnumparchap}{18ex}
  \titlespacing*{name=\chapter,numberless}{0em}{0pt}{18ex}
\else
\ifsuftesi@periodicalaureo
  \titlespacing*{\chapter}{0em}{\SUF@ADJnumparchap}{18ex}
  \titlespacing*{name=\chapter,numberless}{0em}{0pt}{18ex}
\else
\ifsuftesi@compact
  \titlespacing*{\chapter}{0em}{\SUF@ADJnumparchap}{18ex}
  \titlespacing*{name=\chapter,numberless}{0em}{0pt}{18ex}
\else
\ifsuftesi@compactaureo
  \titlespacing*{\chapter}{0em}{\SUF@ADJnumparchap}{18ex}
  \titlespacing*{name=\chapter,numberless}{0em}{0pt}{18ex}
\else
\ifsuftesi@supercompact
  \titlespacing*{\chapter}{0em}{\SUF@ADJnumparchap}{18ex}
  \titlespacing*{name=\chapter,numberless}{0em}{0pt}{18ex}
\else
\ifsuftesi@supercompactaureo
  \titlespacing*{\chapter}{0em}{\SUF@ADJnumparchap}{18ex}
  \titlespacing*{name=\chapter,numberless}{0em}{0pt}{18ex}
\else%standard/standardaureo
  \titlespacing*{\chapter}{0em}{%
     \dimexpr(6ex+\SUF@ADJnumparchap)}{18ex}	
  \titlespacing*{name=\chapter,numberless}{0em}{6ex}{18ex}
         \fi
       \fi
     \fi
   \fi
 \fi
\fi
%    \end{macrocode}
% \paragraph{Redefinitions for \opt{`article'} mode}
%    \begin{macrocode}
\ifsuftesi@article
\def\chapter#1{\ClassError{suftesi}
  {\noexpand\chapter level is undefined 
     using 'structure=article'}
  {\noexpand\chapter level is undefined 
     using 'structure=article'}}
  \setcounter{tocdepth}{3}
    \setcounter{secnumdepth}{3}
 		  \renewcommand\thesection{%
		          \@arabic\c@section}
		  \renewcommand\thesubsection{%
		          \thesection.\@arabic\c@subsection} 
		  \renewcommand\thesubsubsection{%
		          \thesubsection.\@arabic\c@subsubsection}
		  \renewcommand\theparagraph{%
		          \thesubsubsection.\@arabic\c@paragraph}
		  \renewcommand\thesubparagraph{%
		          \theparagraph.\@arabic\c@subparagraph}
%    \end{macrocode}
% The \opt{partpage} options allows you to print a standard part page
% in \opt{article} mode.
%    \begin{macrocode}
\ifsuftesi@partpage
  \relax
    \else
%    \end{macrocode}
% In \opt{article} mode the \cmd{part} command is similar to a 
% \cmd{section} but with more vertical space before and after. 
%    \begin{macrocode}  
  \titleclass{\part}{straight}
  \titlespacing*{\part}{0ex}{2\sectionsep}{2\sectionsep}[0ex]
\fi
\ifsuftesi@numparpart
\titleformat{\part}[display]
  {\SUF@chaptersize\SUF@lr@PARTSwitch}
  {\SUF@PART@StyleSwitch\partname\hskip.5em\SUF@thepart\SUF@dotpart}
  {2ex}
  {\SUF@PART@StyleSwitch}
\else
\titleformat{\part}[hang]
  {\SUF@chaptersize\SUF@lr@PARTSwitch}
  {\SUF@PART@StyleSwitch\partname\hskip.5em\SUF@thepart\SUF@dotpart}
  {1em}
  {\SUF@PART@StyleSwitch}
\fi
\else
%    \end{macrocode}
% \paragraph{The default \opt{`book'} mode}
%    \begin{macrocode}
\ifsuftesi@numparpart
\titleformat{\part}[display]
  {\SUF@chaptersize\SUF@lr@PARTSwitch}
  {\SUF@PART@StyleSwitch\partname\hskip.5em\SUF@thepart\SUF@dotpart}
  {2ex}
  {\SUF@PART@StyleSwitch}
\else
\titleformat{\part}[hang]
  {\SUF@chaptersize\SUF@lr@PARTSwitch}
  {\SUF@PART@StyleSwitch\partname\hskip.5em\SUF@thepart\SUF@dotpart}
  {1em}
  {\SUF@PART@StyleSwitch}
\fi
\fi

%    \end{macrocode}
%    \begin{macrocode}
\titleformat{\paragraph}[runin]
  {}
  {\theparagraph}
  {.5em}
  {\itshape}
  [{.}\hspace*{1em}]
\titlespacing*{\paragraph}{\parindent}{.5\sectionsep}{.5\sectionsep}
%    \end{macrocode}
%    \begin{macrocode}
\titleformat{\subparagraph}[runin]
  {}
  {\thesubparagraph}
  {.5em}
  {}
  [{.}\hspace*{1em}]
\titlespacing*{\subparagraph}{\parindent}{.5\sectionsep}{.5\sectionsep}
%    \end{macrocode}
% The \sty{biblatex} package uses the \sty{book} class
% definitions of bibliography and list of shorthands,
% so we must redefine them according to the styles of \sty{suftesi},
% which does not use uppercase letters in the headings.
%    \begin{macrocode}
\ifsuftesi@article			
\AtBeginDocument{%
\@ifpackageloaded{biblatex}{%
  \defbibheading{bibliography}[\refname]{%
    \section*{#1}%
    \markboth{#1}{#1}}
  \defbibheading{shorthands}[\losname]{%
    \section*{#1}%
    \markboth{#1}{#1}}
  \defbibheading{bibintoc}[\refname]{%
    \section*{#1}%
    \addcontentsline{toc}{section}{#1}%
    \markboth{#1}{#1}}
  \defbibheading{losintoc}[\losname]{%
    \section*{#1}%
    \addcontentsline{toc}{section}{#1}%
    \markboth{#1}{#1}}
  \defbibheading{bibnumbered}[\refname]{%
    \section{#1}%
    \if@twoside\markright{#1}\fi}
  \defbibheading{losnumbered}[\losname]{%
    \section{#1}%
    \if@twoside\markright{#1}\fi}
  \defbibheading{subbibliography}[\refname]{%
    \subsection*{#1}}
  \defbibheading{subbibintoc}[\refname]{%
    \subsection*{#1}%
    \addcontentsline{toc}{subsection}{#1}}
  \defbibheading{subbibnumbered}[\refname]{%
    \subsection{#1}}}%
    {\relax}%
}%
\else							
\AtBeginDocument{%
\@ifpackageloaded{biblatex}{%
  \defbibheading{bibliography}[\bibname]{%
    \chapter*{#1}%
    \markboth{#1}{#1}}
  \defbibheading{shorthands}[\losname]{%
    \chapter*{#1}%
    \markboth{#1}{#1}}
  \defbibheading{bibintoc}[\bibname]{%
    \chapter*{#1}%
    \addcontentsline{toc}{chapter}{#1}%
    \markboth{#1}{#1}}
  \defbibheading{losintoc}[\losname]{%
    \chapter*{#1}%
    \addcontentsline{toc}{chapter}{#1}%
    \markboth{#1}{#1}}
  \defbibheading{bibnumbered}[\bibname]{%
    \chapter{#1}%
    \if@twoside\markright{#1}\fi}
  \defbibheading{losnumbered}[\losname]{%
    \chapter{#1}%
    \if@twoside\markright{#1}\fi}
  \defbibheading{subbibliography}[\refname]{%
    \section*{#1}%
    \if@twoside\markright{#1}\fi}
  \defbibheading{subbibintoc}[\refname]{%
    \section*{#1}%
    \addcontentsline{toc}{section}{#1}%
    \if@twoside\markright{#1}\fi}
  \defbibheading{subbibnumbered}[\refname]{%
    \section{#1}}}
    {\relax}%
}%
\fi
%    \end{macrocode}
% \subsection*{Cover page}
%    \begin{macrocode}
\newcommand{\Ctitle}[1]{\def\@Ctitle{#1}}
\newcommand{\Csubtitle}[1]{\def\@Csubtitle{#1}}
\newcommand{\Cauthor}[1]{\def\@Cauthor{#1}}
\newcommand{\Ceditor}[1]{\def\@Ceditor{#1}}
\newcommand{\Cfoot}[1]{\def\@Cfoot{#1}}
\newcommand{\Cpagecolor}[1]{\def\@Cpagecolor{#1}}
\newcommand{\Ccirclecolor}[1]{\def\@Ccirclecolor{#1}}
\newcommand{\Ctextcolor}[1]{\def\@Ctextcolor{#1}}
\newcommand{\Cfootcolor}[1]{\def\@Cfootcolor{#1}}
\Cauthor{}
\Ctitle{}
\Csubtitle{}
\Ceditor{}
\Cfoot{}
\Cpagecolor{gray!30}
\Ctextcolor{white}
\Cfootcolor{black}
\AtBeginDocument{
\@ifpackageloaded{tikz}{%
\newcommand\makecover[1][]{%
\begin{titlepage}
\begin{tikzpicture}[overlay,remember picture]
  \draw[draw=none,fill=\@Cpagecolor]
    (current page.north west) rectangle (current page.south east);
  \node[anchor=center,yshift=.22\paperwidth] at (current page.center) (c) {};
  \draw[draw=none,fill=gray,#1]
    (c)  circle (.38\paperwidth) ;
  \node[anchor=center] at (c) (author) {%
\parbox{.7\paperwidth}{%
  \centering
    \ifx\@Cauthor\@empty
     \else
       {\scshape\color{\@Ctextcolor}\@Cauthor\\}
       \vspace*{\baselineskip}
     \fi

     \ifx\@Ctitle\@empty
     \else
     {\Huge\bfseries\color{\@Ctextcolor}\@Ctitle\\[1ex]}
     \fi

    \ifx\@Csubtitle\@empty
     \else
       {\smallskip\Large\color{\@Ctextcolor}\@Csubtitle\\}
     \fi

    \ifx\@Ceditor\@empty
     \else
       {\vspace*{2\baselineskip}\color{\@Ctextcolor}\@Ceditor\\}
     \fi}
};
\ifx\@Cfoot\@empty\else
  \node[xshift=.5\paperwidth,yshift=1cm,
    align=center,text=\@Cfootcolor,anchor=south]
  at (current page.south west) {\@Cfoot};
\fi
\end{tikzpicture}
\end{titlepage}
}
}
{\def\makecover{\ClassError{suftesi}{\MessageBreak%
 ***********************************\MessageBreak
 * To use the \noexpand\makecover command\MessageBreak
 * load the 'tikz' package.\MessageBreak
 ************************************}{\MessageBreak%
 ***********************************\MessageBreak
 * To use the \noexpand\makecover command\MessageBreak
 * load the 'tikz' package.\MessageBreak
 ************************************}}}
}
%    \end{macrocode}
% \subsection*{Title page}
% The new \cmd{title} command has an optional argument 
% which can be used in the headers.
%    \begin{macrocode}
\def\isbn#1{\gdef\@isbn{#1}}
  \def\@issn{\@latex@warning@no@line{%
    No \noexpand\isbn given}}
\def\doi#1{\gdef\@doi{#1}}
  \def\@doi{\@latex@warning@no@line{%
    No \noexpand\doi given}}
\def\isbn#1{\gdef\@issn{#1}}
  \def\@issn{\@latex@warning@no@line{%
    No \noexpand\isbn given}}

\renewcommand*{\title}[2][]{\gdef\@headtitle{#1}\gdef\@title{#2}}
    \edef\title{\noexpand\@dblarg
  \expandafter\noexpand\csname\string\title\endcsname}
    \def\@headtitle{--missing title--%
        \protect\ClassWarningNoLine{suftesi}{%
            No \string\title\space given \MessageBreak%
            See the class documentation for explanation}}
    \def\@title{--missing title--%
        \protect\ClassWarningNoLine{suftesi}{%
            No \string\title\space given\MessageBreak%
            See the class documentation for explanation}}
    \def\@author{--missing author--%
        \protect\ClassWarningNoLine{suftesi}{%
            No \string\author\space given\MessageBreak%
            See the class documentation for explanation}}
%    \end{macrocode}
% For |titlepage| (default) option:
%    \begin{macrocode}
\if@titlepage% titlepage
  \renewcommand\maketitle{\begin{titlepage}%
  \let\footnotesize\small
  \let\footnoterule\relax
  \let \footnote \thanks
  \renewcommand\thefootnote{\@fnsymbol\c@footnote}%
  \null\vfil
  \vskip 60\p@
  \begin{center}%
    {\SUF@chaptersize\color{sufred}\sffamily%
    \ifsuftesi@smallcapschap%
     \SUF@titlesmallcaps{\@title}
      \else
      \ifsuftesi@article
       \ifsuftesi@smallcapssec
        \SUF@titlesmallcaps{\@title}
         \else
          \@title
        \fi
       \else
      \@title
     \fi
    \fi\par}%
    \vskip 3em%
    {\small\lineskip .75em%
      \begin{tabular}[t]{c}%
        \@author
      \end{tabular}\par}%
      \vskip 1.5em%
    {\small\@date\par}%       
  \end{center}\par
  \@thanks
  \vfil\null
  \end{titlepage}%
  \setcounter{footnote}{0}%
  \global\let\thanks\relax
  \global\let\maketitle\relax
  \global\let\@thanks\@empty
     \global\let\@date\@empty
  \global\let\date\relax
  \global\let\and\relax}
%    \end{macrocode}
%  Reproduces the standard |\maketitle| style:
%    \begin{macrocode}
\newcommand\standardtitle{\begin{titlepage}%
  \let\footnotesize\small
  \let\footnoterule\relax
  \let \footnote \thanks
  \null\vfil
  \vskip 60\p@
  \begin{center}%
    {\LARGE \@title \par}%
    \vskip 3em%
    {\large
     \lineskip .75em%
      \begin{tabular}[t]{c}%
        \@author
      \end{tabular}\par}%
      \vskip 1.5em%
    {\large \@date \par}%       
  \end{center}\par
  \@thanks
  \vfil\null
  \end{titlepage}%
  \setcounter{footnote}{0}%
  \global\let\thanks\relax
  \global\let\maketitle\relax
  \global\let\@thanks\@empty
     \global\let\@date\@empty
  \global\let\date\relax
  \global\let\and\relax}
\else
%    \end{macrocode}
%  For |notitlepage| option:
%    \begin{macrocode}
\renewcommand\maketitle{\par
   \begingroup
     \renewcommand\thefootnote{\@fnsymbol\c@footnote}%
     \def\@makefnmark{\rlap{\@textsuperscript{\normalfont\@thefnmark}}}%
     \long\def\@makefntext##1{\parindent 1em\noindent
             \hb@xt@1.8em{%
                 \hss\@textsuperscript{\normalfont\@thefnmark}}##1}%
     \if@twocolumn
       \ifnum \col@number=\@ne
         \@maketitle
       \else
         \twocolumn[\@maketitle]%
       \fi
     \else
       \newpage
       \global\@topnum\z@% Prevents figures from going at top of page.
       \@maketitle
     \fi
     \thispagestyle{plain}\@thanks
   \endgroup
   \setcounter{footnote}{0}%
   \global\let\thanks\relax
   \global\let\maketitle\relax
   \global\let\@maketitle\relax
   \global\let\@thanks\@empty
   \global\let\@date\@empty
   \global\let\date\relax
   \global\let\and\relax}
     \def\@maketitle{%
   \newpage
   \null
   \vskip 2em%
   \begin{center}%
   \let \footnote \thanks
     {\SUF@chaptersize\color{sufred}\sffamily%
    \ifsuftesi@smallcapschap%
     \SUF@titlesmallcaps{\@title}
      \else
      \ifsuftesi@article
       \ifsuftesi@smallcapssec
        \SUF@titlesmallcaps{\@title}
         \else
          \@title
        \fi
       \else
      \@title
     \fi
  \fi\par}%
     \vskip 1.5em%
     {\small\lineskip .5em%
       \begin{tabular}[t]{c}%
        \@author\par
       \end{tabular}\par}%
     \vskip 1em%
     {\small\@date\par}%
   \end{center}%
   \par
   \vskip 1.5em}
%    \end{macrocode}
%  Reproduces the standard |\maketitle| style:
%    \begin{macrocode}
\newcommand\standardtitle{\par
  \begingroup
    \renewcommand\thefootnote{\@fnsymbol\c@footnote}%
    \def\@makefnmark{\rlap{\@textsuperscript{\normalfont\@thefnmark}}}%
    \long\def\@makefntext##1{\parindent 1em\noindent
            \hb@xt@1.8em{%
                \hss\@textsuperscript{\normalfont\@thefnmark}}##1}%
    \if@twocolumn
      \ifnum \col@number=\@ne
        \@standardmaketitle
      \else
        \twocolumn[\@standardmaketitle]%
      \fi
    \else
      \newpage
      \global\@topnum\z@   
      \@standardmaketitle
    \fi
    \thispagestyle{plain}\@thanks
  \endgroup
   \setcounter{footnote}{0}%
   \global\let\thanks\relax
   \global\let\maketitle\relax
   \global\let\@standardmaketitle\relax
   \global\let\@thanks\@empty
   \global\let\@date\@empty
   \global\let\date\relax
   \global\let\and\relax}
\def\@standardmaketitle{%
  \newpage
  \null
  \vskip 2em%
  \begin{center}%
  \let \footnote \thanks
    {\LARGE \@title \par}%
    \vskip 1.5em%
    {\large
      \lineskip .5em%
      \begin{tabular}[t]{c}%
        \@author
      \end{tabular}\par}%
    \vskip 1em%
    {\large \@date}%
  \end{center}%
  \par
  \vskip 1.5em}
\fi
%    \end{macrocode}
% \subsection*{The \opt{collection} document structure}
%    \begin{macrocode}
\ifsuftesi@collection
\newcounter{journalnumber}
\newcounter{journalvolume}
\newcounter{issue}
\newcounter{title}
\setcounter{title}{1}
\newcounter{article}
\setcounter{article}{0}
\setcounter{journalnumber}{0}
\setcounter{tocdepth}{0}
\def\journalname#1{\gdef\@journalname{#1}}
  \def\@journalname{\@latex@warning@no@line{%
    No \noexpand\journalname given}}
\def\journalvolume#1{\gdef\@journalvolume{#1}}
  \def\@journalvolume{\@latex@warning@no@line{%
    No \noexpand\journalvolume given}}
\def\journalnumber#1{\gdef\@journalnumber{#1}}
  \def\@journalnumber{\@latex@warning@no@line{%
    No \noexpand\journalnumber given}}
\def\issue#1{\gdef\@issue{#1}}
  \def\@issue{\@latex@warning@no@line{%
    No \noexpand\issue given}}
\def\journalyear#1{\gdef\@journalyear{#1}}
  \def\@journalyear{\@latex@warning@no@line{%
    No \noexpand\journalyear given}}
\def\journalwebsite#1{\gdef\@journalwebsite{\url{#1}}}
  \def\@journalwebsite{\@latex@warning@no@line{%
    No \noexpand\journalwebsite given}}
\def\thanks#1{\footnotemark\ \protected@xdef\@thanks{%
  \@thanks\protect\footnotetext[\the\c@footnote]{#1}}}
\def\fulljournal{\emph{\@journalname} \@journalnumber, %
  \@issue{} \@journalyear}
\def\issuename#1{\gdef\@issuename{#1}}
\def\collectiontitle#1{\gdef\@collectiontitle{#1}}
  \def\@collectiontitle{\@latex@warning@no@line{%
    No \noexpand\collectiontitle given}}
\def\collectioneditor#1{\gdef\@collectioneditor{#1}}
  \def\@collectioneditor{\@latex@warning@no@line{%
    No \noexpand\collectioneditor given}}
%    \end{macrocode}
%    \begin{macrocode}
\renewcommand*{\title}[2][]{%
  \gdef\@headtitle{#1}\gdef\@title{#2}\markright{#1}}
    \edef\title{\noexpand\@dblarg
  \expandafter\noexpand\csname\string\title\endcsname}
    \def\@headtitle{--missing title--%
        \protect\ClassWarningNoLine{suftesi}{%
            No \string\title\space given \MessageBreak%
            See the class documentation for explanation}}
    \def\@title{--missing title--%
        \protect\ClassWarningNoLine{suftesi}{%
            No \string\title\space given\MessageBreak%
            See the class documentation for explanation}}
    \def\@author{--missing author--%
        \protect\ClassWarningNoLine{suftesi}{%
            No \string\author\space given\MessageBreak%
            See the class documentation for explanation}}

\newcommand*\l@title[2]{%
  \ifnum \c@tocdepth >\m@ne
    \addpenalty{-\@highpenalty}%
    \vskip 1.0ex \@plus\p@
    \begingroup
      \parindent \z@ \rightskip \@pnumwidth
      \parfillskip -\@pnumwidth
      \advance\leftskip1em
      \hskip -\leftskip
      #1\nobreak%
    \ifsuftesi@dottedtoc\dotfill%
      \nobreak\hb@xt@\@pnumwidth{\hss #2}\par
        \else
          \ifsuftesi@raggedtoc%
            \nobreak\hskip1em #2 \hfill\null\par
              \else
                \ifsuftesi@pagelefttoc
              \ClassError{suftesi}
                {\MessageBreak
                You can not use tocstyle=leftpage\MessageBreak
                  with structure=collection}
                {You can not use tocstyle=leftpage\MessageBreak
                  with structure=collection}
            \else
          \nobreak\hfill #2\par
        \fi
      \fi
    \fi
      \penalty\@highpenalty
    \endgroup
  \fi}
%    \end{macrocode}
%    \begin{macrocode}
\renewcommand\maketitle{\par
  \begingroup
    \renewcommand\thefootnote{\@fnsymbol\c@footnote}%
    \def\@makefnmark{\rlap{\@textsuperscript{\normalfont\@thefnmark}}}%
    \long\def\@makefntext##1{\parindent 1em\noindent
            \hb@xt@1.8em{%
                \hss\@textsuperscript{\normalfont\@thefnmark}}##1}%
    \if@twocolumn
      \ifnum \col@number=\@ne
        \@maketitle
      \else
        \twocolumn[\@maketitle]%
      \fi
    \else
      \newpage
      \global\@topnum\z@   % Prevents figures from going at top of page.
      \@maketitle
    \fi
    \thispagestyle{plain}\@thanks%
  \endgroup
%  \setcounter{footnote}{0}%
  \setcounter{section}{0}%
%  \global\let\thanks\relax
%  \global\let\maketitle\relax
%  \global\let\@maketitle\relax
%  \global\let\@thanks\@empty
%  \global\let\@author\@empty
%  \global\let\@date\@empty
%  \global\let\@title\@empty
%  \global\let\title\relax
%  \global\let\author\relax
%  \global\let\date\relax
  \global\let\and\relax
  \let\thanks\@gobble}
\AtBeginDocument{\def\@maketitle{%
  \refstepcounter{article}
  \SUF@chaptersize
   \SUF@lr@coll@titleSwitch
    \let\footnote\thanks
     \parindent=0pt
    {\ifsuftesi@reverseauthortitle
      \SUF@titlefont@Switch{\@title}%
        \else\SUF@authorfont@Switch{\@author}\fi}%
   \label{begin:\thearticle}
     \xdef\@currentHref{title.\thearticle}%
  \Hy@raisedlink{%
  \hyper@anchorstart{\@currentHref}\hyper@anchorend}%
  \csname toc@entry@\endcsname
    \begingroup%
    \let\thanks\@gobble
    \addcontentsline{toc}{title}{%
    {\SUF@tocAUT@font{\@author}}\texorpdfstring{\newline}{, }%
    {\SUF@tocTIT@font{\@headtitle}}}
    \endgroup%
    \par\nobreak\vspace{2ex}
    {\ifsuftesi@reverseauthortitle
      \SUF@authorfont@Switch{\@author}%
        \else
      \SUF@titlefont@Switch{\@title}%
    \fi\vskip1.5cm}}%
    }
%\newenvironment{article}
%  {\begingroup
%  \global\let\@thanks\@empty
%  \setcounter{footnote}{0}
%  \refstepcounter{article}
%  \label{begin:\thearticle}
%}
%  {\label{end:\thearticle}\endgroup
%  }
\newenvironment{article}
          {\begingroup
          \setcounter{section}{0}
          \setcounter{footnote}{0}
          \setcounter{figure}{0}
          \setcounter{table}{0}}
          {\label{end:\thearticle}
            \cleardoublepage
              \global\let\@thanks\@empty
            \endgroup}
%    \end{macrocode}
%  A command to typeset 
% the frontispiece of the collection.
%    \begin{macrocode}
\newcommand{\frontispiece}{%
  \thispagestyle{empty}%
    \begingroup
     \centering
       \vspace*{\stretch{1}}
     
       {\SUF@chaptersize\@collectiontitle\par}
         \vskip5ex
     
       \@collectioneditor
       \vspace*{\stretch{3}}
            
  \endgroup
  \clearpage}
%    \end{macrocode}
% In the collection document structure the articles are treated as
% chapters but you would not need to print in
% the table of contents all the sections of every article. So first of all we include in the table of contents only the author and the title of each paper:
%    \begin{macrocode}
  \setcounter{tocdepth}{0}
%    \end{macrocode}
% Anyway the sections inside each paper are numbered 
% as in standard articles:
%    \begin{macrocode}
  \renewcommand\thesection{%
          \@arabic\c@section}
  \renewcommand\thesubsection{%
          \thesection.\@arabic\c@subsection}
  \renewcommand\thesubsubsection{%
          \thesubsection.\@arabic\c@subsubsection}
  \renewcommand\theparagraph{%
          \thesubsubsection.\@arabic\c@paragraph}
  \renewcommand\thesubparagraph{%
          \theparagraph.\@arabic\c@subparagraph}
  \else
\fi
%    \end{macrocode}
% \subsection*{Frontispiece}
%    \begin{macrocode}
\AtBeginDocument{%
\@ifpackagewith{frontespizio}{suftesi}{%
\ifsuftesi@periodical
  \Margini {5.5cm}{7cm}{4.5cm}{0cm}
    \else
\ifsuftesi@compact
  \Margini {4.5cm}{7cm}{4.5cm}{0cm}
    \else
\ifsuftesi@supercompact
  \Margini {4.5cm}{10cm}{6cm}{1cm}
    \else
\ifsuftesi@compactaureo
  \Margini {4.5cm}{7cm}{4.5cm}{0cm}
    \else
\ifsuftesi@supercompactaureo
  \Margini {4.5cm}{10cm}{6cm}{1cm}
    \else
\ifsuftesi@periodicalaureo
  \Margini {5.5cm}{7cm}{4.5cm}{0cm}
    \else
\fi\fi\fi\fi\fi\fi}
{\@ifpackageloaded{frontespizio}{%
\ifsuftesi@periodical
  \Margini {1cm}{7cm}{5cm}{1cm}
  \Rientro{1cm}
    \else
\ifsuftesi@compact
  \Margini {1cm}{7cm}{6cm}{1cm}
  \Rientro{1cm}
    \else
\ifsuftesi@supercompact
  \Margini {1cm}{10cm}{8cm}{1cm}
  \Rientro{1cm}
    \else
\ifsuftesi@compactaureo
  \Margini {1cm}{7cm}{6cm}{1cm}
  \Rientro{1cm}
    \else
\ifsuftesi@supercompactaureo
  \Margini {1cm}{10cm}{8cm}{1cm}
  \Rientro{1cm}
    \else
\ifsuftesi@periodicalaureo
  \Margini {1cm}{7cm}{5cm}{1cm}
  \Rientro{1cm}
    \else
\fi\fi\fi\fi\fi\fi}
{\relax}}}
%    \end{macrocode}
%    \begin{macrocode}
\renewenvironment{theindex}
               {\if@twocolumn
                  \@restonecolfalse
                \else
                  \@restonecoltrue
                \fi
                \ifsuftesi@article
                \twocolumn[\section*{\indexname}]%
                \else
                \twocolumn[\@makeschapterhead{\indexname}]%
                \fi
                \@mkboth{\indexname}{\indexname}%
                \thispagestyle{plain}%
                \raggedright%
                \parindent\z@
                \parskip\z@ \@plus .3\p@\relax
                \columnseprule \z@
                \columnsep 35\p@
                \let\item\@idxitem}
               {\if@restonecol\onecolumn\else\clearpage\fi}
%    \end{macrocode}
% \subsection*{Appendix}
%    \begin{macrocode}
\newcommand{\appendicesname}[1]{\def\SUF@appendices{#1}}
    \appendicesname{Appendici}
    \newcommand{\appendixpage}{\SUF@appendixpage}
\def\SUF@appendixpage{%
  \@mainmattertrue
    \titlecontents{part}                                                  
      [0em]                                                                    
      {\addvspace{3ex}}
      {}
      {}
      {}
      [\addvspace{1ex}]
\let\contentspage\relax
    \cleardoublepage
       \thispagestyle{empty}
         \addcontentsline{toc}{part}{\SUF@appendices}
     \begingroup
       \centering
          \null\vfil
        {\LARGE\SUF@appendices\par}
          \vfil
   \endgroup
    \cleardoublepage
\titlecontents{part}                                                  
   [0em]                                                                    
  {\addvspace{3ex}\partname~}                                                 
  {\makebox[\SUF@label@part][l]{%
    \SUF@toclabelnum\thecontentslabel}\hspace*{1em}}                                                    
  {}                                                                      
  {}
  [\addvspace{1ex}]
}
\ifsuftesi@article
  \renewcommand\appendix{\par
    \setcounter{section}{0}%
    \setcounter{subsection}{0}%
    \gdef\SUF@thesection{\@Alph\c@section}}
\else
  \renewcommand\appendix{\par
    \setcounter{chapter}{0}%
    \setcounter{section}{0}%
    \gdef\@chapapp{\appendixname}%
    \gdef\SUF@thechapter{\@Alph\c@chapter}}
\fi
%    \end{macrocode}
% \subsection*{Headings}
%    \begin{macrocode}
\RequirePackage{fancyhdr}
\newcommand{\versionstring}[1]{\def\version@string{#1}}
  \versionstring{Version of}
\AtBeginDocument{%
 \pagestyle{fancy}
  \renewcommand{\headrulewidth}{0pt} 
  \renewcommand{\footnoterule}{}
\def\SUF@versionstring{\texttt{\version@string{} \today}}
%    \end{macrocode}
% \paragraph{The \opt{default} headers}
%    \begin{macrocode}
\renewcommand{\chaptermark}[1]{%
  \markboth{\chaptertitlename\ \SUF@thechapter}{#1}}
	\ifsuftesi@article
	  \renewcommand{\sectionmark}[1]{\markright{\SUF@thesection.\ #1}}
	    \else
	  \renewcommand{\sectionmark}[1]{}
	\fi
%    \end{macrocode}
% \paragraph{The \opt{plain} style}
%    \begin{macrocode}
\fancypagestyle{plain}{\fancyhf{}}
%    \end{macrocode}
% \paragraph{The \opt{sufplain} style}
%    \begin{macrocode}
\fancypagestyle{sufplain}{%
 \fancyhf{}%
 \fancyfoot[RE,LO]{%
   \ifsuftesi@draftdate\footnotesize\SUF@versionstring\else\fi}
 \fancyfoot[C]{\footnotesize\SUF@thepage}}
%    \end{macrocode}
% \paragraph{The \opt{centerheader} style}
%    \begin{macrocode}
\fancypagestyle{centerheader}{%
  \fancyhf{}%
  \fancyfoot[RE,LO]{%
    \ifsuftesi@draftdate\footnotesize\SUF@versionstring\else\fi}
  \fancyhead[CO]{\footnotesize\xheadbreakfalse\SUF@rightmark} 
  \fancyhead[CE]{\footnotesize\xheadbreakfalse%
    \SUF@LR@MarkSwitch} 
  \fancyfoot[C]{\footnotesize\SUF@thepage}%
  }%
%    \end{macrocode}
% \paragraph{The \opt{sufdefault} style}
%    \begin{macrocode}
\fancypagestyle{sufdefault}{%
  \fancyhf{}%
  \fancyfoot[RE,LO]{%
    \ifsuftesi@draftdate\footnotesize\SUF@versionstring\else\fi}
  \fancyhead[LE,RO]{\footnotesize\SUF@thepage} 
  \fancyhead[LO]{\footnotesize\xheadbreakfalse\SUF@rightmark}
  \fancyhead[RE]{\footnotesize\xheadbreakfalse%
    \SUF@LR@MarkSwitch}
  }%
%
\ifsuftesi@article
				\ifsuftesi@authortitle
				\def\SUF@LR@MarkSwitch{\SUF@leftmark}
				\else
				\def\SUF@LR@MarkSwitch{\SUF@rightmark}
				\fi
\else
        \def\SUF@LR@MarkSwitch{\SUF@leftmark}
\fi
\def\SUF@leftrightmark{%
 \if@mainmatter\leftmark\else\rightmark\fi} 
%    \end{macrocode}
% Setting the default page style:
%    \begin{macrocode}
\pagestyle{sufdefault}
  \ifsuftesi@centerheader\pagestyle{centerheader}\else\fi
  \ifsuftesi@sufplain\pagestyle{sufplain}\else\fi}
%    \end{macrocode}
% \subsection*{Text elements}
%
% \paragraph{Block Quotations}
%
% New environments for block quotations according to a popular Italian style. 
% The font size is the same of the footnotes and the margins are set to \cmd{parindent}.
%    \begin{macrocode}
\renewenvironment{quotation}
            {\list{}{\listparindent\parindent%
                     \itemindent    \listparindent
                     \leftmargin     \parindent
                     \SUF@quote@style                       
                     \parsep        \z@ \@plus\p@}%
               \item\relax%
                 \SUF@quotation@size%
                 \noindent\ignorespaces}
               {\endlist}
\renewenvironment{quote}
            {\list{}{\leftmargin \parindent
            \SUF@quote@style}%
               \item\relax%
                 \SUF@quotation@size}%\ignorespaces?
               {\endlist}
\renewenvironment{verse}
               {\let\\\@centercr
                \list{}{\itemsep      \z@
                        \itemindent   -1.5em%
                        \listparindent\itemindent
                        \rightmargin  \leftmargin
                        \advance\leftmargin 1.5em}%
                \item\relax
                 \SUF@quotation@size}
               {\endlist}  
%    \end{macrocode}
% \paragraph{The \opt{fewfootnotes} option}
% Enable only with three footnotes per page maximum.
%    \begin{macrocode}
\ifsuftesi@fewfootnotes
\AtBeginDocument{%
\def\@fnsymbol#1{\ensuremath{\ifcase#1\or*\or{*}{*}\or{*}{*}{*}\or%
  \ClassError{suftesi}%
  {Too many footnotes\MessageBreak
  Remove the class option 'fewfootnote'}
  {Too many footnotes\MessageBreak
  Remove the class option 'fewfootnote'}
  \else\@ctrerr\fi}}}
\def\thefootnote{\@fnsymbol\c@footnote}%
\else\fi
%    \end{macrocode}
% Prints a footnote with discretionary
% symbol give in the first argument.
%    \begin{macrocode}
\newcommand*\xfootnote[1][*]{%
    \xdef\@thefnmark{#1}%
    \@footnotemark\@footnotetext}
%    \end{macrocode}
% \paragraph{Marginal notes}
% The |\marginpar| command is redefined according to the look 
% of \emph{Classic Thesis} by 
% André \textcite{Miede:2011}\index{Miede, André}. 
%    \begin{macrocode}
\def\SUF@mpsetup{%
  \itshape
    \footnotesize%
    \parindent=0pt \lineskip=0pt \lineskiplimit=0pt %
    \tolerance=2000 \hyphenpenalty=300 \exhyphenpenalty=300%
    \doublehyphendemerits=100000%
    \finalhyphendemerits=\doublehyphendemerits}
  \let\oldmarginpar\marginpar
  \renewcommand{\marginpar}[1]{\oldmarginpar%
     [\SUF@mpsetup\raggedleft\hspace{0pt}{#1}]%
     {\SUF@mpsetup\raggedright\hspace{0pt}{#1}}}
%    \end{macrocode}
% Redefine an internal command of the \sty{todonotes} package in 
% order to use the class-specific marginal notes when this 
% package is loaded. This redefinition simply substitute 
% \cmd{oldmarginpar} to \cmd{marginpar}:
%    \begin{macrocode}
\AtBeginDocument{%
\@ifpackageloaded{todonotes}{%
\renewcommand{\@todonotes@drawMarginNoteWithLine}{%
\begin{tikzpicture}[remember picture, overlay, baseline=-0.75ex]%
    \node [coordinate] (inText) {};%
\end{tikzpicture}%
\oldmarginpar[{% Draw note in left margin
    \@todonotes@drawMarginNote%
    \@todonotes@drawLineToLeftMargin%
}]{% Draw note in right margin
    \@todonotes@drawMarginNote%
    \@todonotes@drawLineToRightMargin%
}%
}%
}
{\relax}}
%    \end{macrocode}
% \paragraph{Abstract}
%    \begin{macrocode}
\ifsuftesi@collection
  \newenvironment{abstract}{%
      \if@twocolumn
        \section*{\abstractname}%
      \else
        \small
        \begin{center}%
          {\abstractname\vspace{-.5em}\vspace{\z@}}%
        \end{center}%
        \quotation
      \fi}
      {\if@twocolumn\else\endquotation\fi\vspace{6ex}}
\else
\if@titlepage
  \newenvironment{abstract}{%
      \titlepage
      \null\vfil
      \@beginparpenalty\@lowpenalty
      \begin{center}%
         \abstractname
        \@endparpenalty\@M
      \end{center}}%
     {\par\vfil\null\endtitlepage}
\else
  \newenvironment{abstract}{%
      \if@twocolumn
        \section*{\abstractname}%
      \else
        \small
        \begin{center}%
          {\abstractname\vspace{-.5em}\vspace{\z@}}%
        \end{center}%
        \quotation
      \fi}
      {\if@twocolumn\else\endquotation\fi}
\fi
\fi
\newcommand\abstractname{Abstract}
%    \end{macrocode}
% \paragraph{Colophon or copyright notice} 
%    \begin{macrocode}
\newcommand{\colophon}[3][]{%
  \thispagestyle{empty}
  \null
    \vfill
     \def\next{#2}
         \ifx\next\@empty\else
            \noindent Copyright \copyright{} \the\year~#2\\[1ex]
          Tutti i diritti riservati
        \fi
  \vfill  
  {\small\noindent Questo lavoro \`e stato composto con \LaTeX{}%
     \def\next{#1}
        \ifx\next\@empty\else su #1 
      \fi usando la classe \textsf{suftesi} di 
      Ivan Valbusa\index{Valbusa, Ivan}. #3\par}
      \cleardoublepage}
%    \end{macrocode}
%    \begin{macrocode}
\newcommand{\bookcolophon}[2]{%
  \thispagestyle{empty}
  \null
    \vfill
            \noindent #1
  \vfill  
  {\small\noindent #2\par}
      \cleardoublepage}
%    \end{macrocode}
%    \begin{macrocode}
\newcommand{\artcolophon}[1]{%
\thispagestyle{empty}
  \null
    \vfill
  {\small\noindent #1\par}}
%    \end{macrocode}
%    \begin{macrocode}
\newcommand{\finalcolophon}[1]{%
\thispagestyle{empty}
  \null\vspace*{\stretch{1}}
  \begin{center}
  \begin{minipage}{.5\textwidth}
  \centering\small #1
  \end{minipage}
  \end{center}
    \vspace*{\stretch{6}}}
%    \end{macrocode}
% \paragraph{The \cmd{FSPL} colophon}
% This command is defined only for the \opt{style=FSPL*} options.
%    \begin{macrocode}
\ifsuftesi@FSPL
\newcommand{\FSPLcolophon}[1][\the\year]{%
\begingroup
\thispagestyle{empty}
\null\vspace{\stretch{1}}
\noindent \hskip-.5em\cc #1 \@author%
\vskip1ex

\small\noindent This work is licensed under the Creative Commons 
Attribution-NonCommercial-NoDerivs 3.0 Unported License. 
To view a copy of this license, 
visit http://creativecom mons.org/licenses/by-nc-nd/3.0/.

\endgroup

\begingroup
\footnotesize

\null\vspace{\stretch{1}}

\noindent Typeset with \LaTeX{} in collaboration with the Joint Project 
\emph{Formal Style for PhD Theses with \LaTeX{}} (University of Verona, 
Italy) using the \textsf{suftesi} class by Ivan Valbusa. The text face 
is Palatino, designed by Hermann Zapf. The sans serif font is Iwona by 
Janusz M. Nowacki.

\endgroup

\clearpage}
\else
\def\FSPLcolophon{%
 \ClassError{suftesi}
   {\noexpand\FSPLcolophon is defined\MessageBreak 
     only for the FSPL styles}
   {\noexpand\FSPLcolophon is defined\MessageBreak 
     only for the FSPL styles}}
\fi
%    \end{macrocode}
% \subsection*{Toc, lof, lot}
%    \begin{macrocode}
\RequirePackage{titletoc}
%    \end{macrocode}
% All the lengths depend on |\SUF@label@chap| 
% so we define this first.
%    \begin{macrocode}
\newlength\SUF@label@chap
\setlength\SUF@label@chap{.5em}
%    \end{macrocode}
% |\toclabelwidth| is provided to
% adjust the label width in the table of contents:
%    \begin{macrocode}
\newcommand*{\toclabelwidth}[2]{%
  \AtBeginDocument{
    \addtolength{\csname SUF@label@#1\endcsname}{#2}%
    \addtolength{\csname SUF@tocindent@#1\endcsname}{#2}%
  }
}
\newcommand{\toclabelspace}{%
  \ClassError{suftesi}
    {\MessageBreak
    \noexpand\toclabelspace is not more defined\MessageBreak
    Use \noexpand\toclabelwidth instead.\MessageBreak
    See package documentation for details}
    {\MessageBreak
    \noexpand\toclabelspace is not more defined\MessageBreak
    Use \noexpand\toclabelwidth instead.\MessageBreak
    See package documentation for details}}
%    \end{macrocode}
% This macro controls the space between page number and chapter
% label using the \opt{tocpageleft} option:  
%    \begin{macrocode}
\newlength{\SUF@tochang}
\setlength{\SUF@tochang}{3em}
\AtBeginDocument{
\newlength\SUF@label@part
\newlength\SUF@label@sec
\newlength\SUF@label@subsec
\newlength\SUF@label@subsubsec
\newlength\SUF@label@par
\newlength\SUF@label@subpar
\newlength\SUF@label@fig
\newlength\SUF@label@tab
\setlength\SUF@label@part         
    {\SUF@label@chap}
\setlength\SUF@label@sec      
    {\dimexpr(\SUF@label@chap+.5em)}
\setlength\SUF@label@subsec   
    {\dimexpr(\SUF@label@sec+.5em)}
\setlength\SUF@label@subsubsec
    {\dimexpr(\SUF@label@subsec+.5em)}
\setlength\SUF@label@par    
    {\dimexpr(\SUF@label@subsubsec+.5em)}
\setlength\SUF@label@subpar 
    {\dimexpr(\SUF@label@par+.5em)}
\setlength\SUF@label@fig       
    {\SUF@label@sec}
\setlength\SUF@label@tab        
    {\SUF@label@sec}
}
%    \end{macrocode}
% Part in article mode
%    \begin{macrocode}
\titlecontents{part}
  [0em]
  {\addvspace{3ex}\partname\hspace*{.5em}}
  {\makebox[\SUF@label@part][l]{%
    \SUF@toclabelnum\thecontentslabel}\hspace*{1em}}
  {}
  {}
  [\addvspace{1ex}]
%    \end{macrocode}
% \paragraph{`\opt{tocpageleft}' toc}
%    \begin{macrocode}
\ifsuftesi@pagelefttoc
%    \end{macrocode}
% First we reset the right margin to zero:
%    \begin{macrocode}
\contentsmargin{0pt}
\AtBeginDocument{
\newlength\SUF@tochang@chap
\newlength\SUF@tochang@sec
\newlength\SUF@tochang@subsec
\newlength\SUF@tochang@subsubsec
\newlength\SUF@tochang@par
\newlength\SUF@tochang@subpar
\newlength\SUF@tochang@fig
\newlength\SUF@tochang@tab
\newlength\SUF@addto@tochang@chap
\newlength\SUF@addto@tochang@sec
\newlength\SUF@addto@tochang@subsec
\newlength\SUF@addto@tochang@subsubsec
\newlength\SUF@addto@tochang@par
\newlength\SUF@addto@tochang@subpar
\newlength\SUF@addto@tochang@fig
\newlength\SUF@addto@tochang@tab
\setlength\SUF@tochang@chap     
    {\dimexpr(1em+\SUF@tochang+\SUF@label@chap+1em)}
\setlength\SUF@tochang@sec      
    {\dimexpr(\SUF@tochang@chap+\SUF@label@sec+1em)}
\setlength\SUF@tochang@subsec   
    {\dimexpr(\SUF@tochang@sec+\SUF@label@subsec+1em)}
\setlength\SUF@tochang@subsubsec
    {\dimexpr(\SUF@tochang@subsec+\SUF@label@subsubsec+1em)}
\setlength\SUF@tochang@par   
    {\dimexpr(\SUF@tochang@subsubsec+\SUF@label@par+1em)}
\setlength\SUF@tochang@subpar   
    {\dimexpr(\SUF@tochang@par+\SUF@label@subpar+1em)}
\setlength\SUF@tochang@fig       
    {\SUF@tochang@chap}
\setlength\SUF@tochang@tab       
    {\SUF@tochang@chap}
\setlength\SUF@addto@tochang@chap      
    {\SUF@tochang}
\setlength\SUF@addto@tochang@sec       
    {\dimexpr(\SUF@addto@tochang@chap+\SUF@label@sec+.5em)}
\setlength\SUF@addto@tochang@subsec    
    {\dimexpr(\SUF@addto@tochang@sec+\SUF@label@subsec+.5em)}
\setlength\SUF@addto@tochang@subsubsec 
    {\dimexpr(\SUF@addto@tochang@subsec+\SUF@label@subsubsec+.5em)}
\setlength\SUF@addto@tochang@par       
    {\dimexpr(\SUF@addto@tochang@subsubsec+\SUF@label@par+.5em)}
\setlength\SUF@addto@tochang@subpar       
    {\dimexpr(\SUF@addto@tochang@par+\SUF@label@subpar+.5em)}
\setlength\SUF@addto@tochang@fig       
    {\dimexpr(\SUF@addto@tochang@chap-\SUF@label@sec+\SUF@label@chap)}
\setlength\SUF@addto@tochang@tab       
    {\dimexpr(\SUF@addto@tochang@chap-\SUF@label@sec+\SUF@label@chap)}
}
%    \end{macrocode}
% \paragraph{TOC entries}
%    \begin{macrocode}
\titlecontents{chapter}
  [\SUF@tochang@chap]
  {\addvspace{2ex}}
  {\hskip-\SUF@tochang@chap%
   \makebox[1em][l]{\thecontentspage}%
    \hskip\SUF@addto@tochang@chap%
      \makebox[\SUF@label@chap][l]{%
        \SUF@toclabelnum\thecontentslabel}\hspace*{1em}%
          \SUF@tocCHAP@font}
  {\hskip-\SUF@tochang@chap%
     \makebox[1em][l]{\thecontentspage}%
       \hskip\SUF@addto@tochang@chap\SUF@tocCHAP@font}
  {}
  [\addvspace{1ex}]
\titlecontents{section}
  [\SUF@tochang@sec]
  {}
  {\hskip-\SUF@tochang@sec%
   \makebox[1em][l]{\thecontentspage}\hskip\SUF@addto@tochang@sec%
    \makebox[\SUF@label@sec][l]{%
    \ifsuftesi@article\SUF@toclabelnum%
     \else\fi\thecontentslabel}\hspace*{1em}%
       \SUF@tocSEC@font}
  {\hskip-\SUF@tochang@sec%
   \makebox[1em][l]{\thecontentspage}\hskip\SUF@addto@tochang@sec%
     \SUF@tocSEC@font}
  {}
\titlecontents{subsection}
  [\SUF@tochang@subsec]
  {}
  {\hskip-\SUF@tochang@subsec%
   \makebox[1em][l]{\thecontentspage}\hskip\SUF@addto@tochang@subsec%
    \makebox[\SUF@label@subsec][l]{\thecontentslabel}\hspace*{1em}%
      \SUF@tocSUBSEC@font}
  {\hskip-\SUF@tochang@subsec%
   \makebox[1em][l]{\thecontentspage}\hskip\SUF@addto@tochang@subsec%
     \SUF@tocSUBSEC@font}
  {}
\titlecontents{subsubsection}
  [\SUF@tochang@subsubsec]
  {}
  {\hskip-\SUF@tochang@subsubsec%
   \makebox[1em][l]{\thecontentspage}\hskip\SUF@addto@tochang@subsubsec%
    \makebox[\SUF@label@subsubsec][l]{\thecontentslabel}\hspace*{1em}%
      \SUF@tocSUBSUBSEC@font}
  {\hskip-\SUF@tochang@subsubsec%
   \makebox[1em][l]{\thecontentspage}\hskip\SUF@addto@tochang@subsubsec%
     \SUF@tocSUBSUBSEC@font}
  {}
\titlecontents{paragraph}
  [\SUF@tochang@par]
  {}
  {\hskip-\SUF@tochang@par%
   \makebox[1em][l]{\thecontentspage}\hskip\SUF@addto@tochang@par%
    \makebox[\SUF@label@par][l]{\thecontentslabel}\hspace*{1em}}
  {\hskip-\SUF@tochang@par%
   \makebox[1em][l]{\thecontentspage}\hskip\SUF@addto@tochang@par}
  {}
\titlecontents{subparagraph}
  [\SUF@tochang@subpar]
  {}
  {\hskip-\SUF@tochang@subpar%
   \makebox[1em][l]{\thecontentspage}\hskip\SUF@addto@tochang@subpar%
    \makebox[\SUF@label@subpar][l]{\thecontentslabel}\hspace*{1em}}
  {\hskip-\SUF@tochang@subpar%
   \makebox[1em][l]{\thecontentspage}\hskip\SUF@addto@tochang@subpar}
  {}
\titlecontents{figure}
  [\SUF@tochang@fig]
  {}
  {\hskip-\SUF@tochang@fig%
   \makebox[1em][l]{\thecontentspage}\hskip\SUF@addto@tochang@fig%
     \makebox[\SUF@label@tab][l]{\thecontentslabel}\hspace*{1em}}
  {}
  {}
\titlecontents{table}
  [\SUF@tochang@tab]
  {}
  {\hskip-\SUF@tochang@tab%
   \makebox[1em][l]{\thecontentspage}\hskip\SUF@addto@tochang@tab%
     \makebox[\SUF@label@tab][l]{\thecontentslabel}\hspace*{1em}}
  {}
  {}
\else
%    \end{macrocode}
% \paragraph{Default toc}
%    \begin{macrocode}
\AtBeginDocument{
\newlength\SUF@tocindent@chap
\newlength\SUF@tocindent@sec
\newlength\SUF@tocindent@subsec
\newlength\SUF@tocindent@subsubsec
\newlength\SUF@tocindent@par
\newlength\SUF@tocindent@subpar
\newlength\SUF@tocindent@fig
\newlength\SUF@tocindent@tab
\ifsuftesi@article
\setlength\SUF@tocindent@sec
    {\dimexpr(\SUF@label@chap+1.5em)}
\setlength\SUF@tocindent@subsec
    {\dimexpr(\SUF@tocindent@sec+\SUF@label@subsec+1em)}
\setlength\SUF@tocindent@subsubsec
    {\dimexpr(\SUF@tocindent@subsec+\SUF@label@subsubsec+1em)}
\setlength\SUF@tocindent@par
    {\dimexpr(\SUF@tocindent@subsubsec+\SUF@label@par+1em)}
\setlength\SUF@tocindent@subpar
    {\dimexpr(\SUF@tocindent@par+\SUF@label@subpar+1em)}
\setlength\SUF@tocindent@fig
    {\dimexpr(\SUF@label@chap+1.5em)}
\setlength\SUF@tocindent@tab
    {\dimexpr(\SUF@label@chap+1.5em)}
\else
\setlength\SUF@tocindent@chap
    {\dimexpr(\SUF@label@chap+1em)}
\setlength\SUF@tocindent@sec
    {\dimexpr(\SUF@tocindent@chap+\SUF@label@sec+1em)}
\setlength\SUF@tocindent@subsec
    {\dimexpr(\SUF@tocindent@sec+\SUF@label@subsec+1em)}
\setlength\SUF@tocindent@subsubsec
    {\dimexpr(\SUF@tocindent@subsec+\SUF@label@subsubsec+1em)}
\setlength\SUF@tocindent@par
    {\dimexpr(\SUF@tocindent@subsubsec+\SUF@label@par+1em)}
\setlength\SUF@tocindent@subpar
    {\dimexpr(\SUF@tocindent@par+\SUF@label@subpar+1em)}
\setlength\SUF@tocindent@fig
    {\dimexpr(\SUF@tocindent@chap+\SUF@label@sec-\SUF@label@chap)}
\setlength\SUF@tocindent@tab
    {\dimexpr(\SUF@tocindent@chap+\SUF@label@sec-\SUF@label@chap)}
\fi
}
%    \end{macrocode}
% \paragraph{TOC entries}
%    \begin{macrocode}
\titlecontents{chapter}
  [\SUF@tocindent@chap]
  {\addvspace{2ex}}
  {\hskip-\SUF@tocindent@chap%
    \makebox[\SUF@label@chap][l]{\SUF@toclabelnum\thecontentslabel}%
    \hspace*{1em}%
      \SUF@tocCHAP@font}
  {\hskip-\SUF@tocindent@chap%
      \SUF@tocCHAP@font}
  {\SUF@chaptitlerule\contentspage}
  [\addvspace{1ex}]
\titlecontents{section}
  [\SUF@tocindent@sec]
  {}
  {\hskip-\dimexpr(\SUF@label@sec+1em)%
    \makebox[\SUF@label@sec][l]{%
    \ifsuftesi@article\SUF@toclabelnum%
      \else\fi\thecontentslabel}\hspace*{1em}%
        \SUF@tocSEC@font}
  {\hskip-\dimexpr(\SUF@label@sec+1em)%
    \SUF@tocSEC@font}
  {\ifsuftesi@article\SUF@chaptitlerule%
       \else\SUF@titlerule\fi\contentspage}
\titlecontents{subsection}
  [\SUF@tocindent@subsec]
  {}
  {\hskip-\dimexpr(\SUF@label@subsec+1em)%
    \makebox[\SUF@label@subsec][l]{\thecontentslabel}\hspace*{1em}%
      \SUF@tocSUBSEC@font}
  {\hskip-\dimexpr(\SUF@label@subsec+1em)%
    \SUF@tocSUBSEC@font}
  {\SUF@titlerule\contentspage}
\titlecontents{subsubsection}
  [\SUF@tocindent@subsubsec]
  {}
  {\hskip-\dimexpr(\SUF@label@subsubsec+1em)%
    \makebox[\SUF@label@subsubsec][l]{\thecontentslabel}\hspace*{1em}%
      \SUF@tocSUBSUBSEC@font}
  {\hskip-\dimexpr(\SUF@label@subsubsec+1em)%
    \SUF@tocSUBSUBSEC@font}
  {\SUF@titlerule\contentspage}
\titlecontents{paragraph}
  [\SUF@tocindent@par]
  {}
  {\hskip-\dimexpr(\SUF@label@par+1em)%
    \makebox[\SUF@label@par][l]{\thecontentslabel}\hspace*{1em}}
  {\hskip-\dimexpr(\SUF@label@par+1em)}
  {\SUF@titlerule\contentspage}
\titlecontents{subparagraph}
  [\SUF@tocindent@subpar]
  {}
  {\hskip-\dimexpr(\SUF@label@subpar+1em)%
    \makebox[\SUF@label@subpar][l]{\thecontentslabel}\hspace*{1em}}
  {\hskip-\dimexpr(\SUF@label@subpar+1em)}
  {\SUF@titlerule\contentspage}
\titlecontents{figure}
  [\SUF@tocindent@fig]
  {}
  {\hskip-\SUF@tocindent@fig%
    \makebox[\SUF@label@fig][l]{\thecontentslabel}\hspace*{1em}}
  {}
  {\SUF@titlerule\contentspage}
\titlecontents{table}
  [\SUF@tocindent@tab]
  {}
  {\hskip-\SUF@tocindent@tab%
    \makebox[\SUF@label@tab][l]{\thecontentslabel}\hspace*{1em}}
  {}
  {\SUF@titlerule\contentspage}
\fi
%    \end{macrocode}
% With \opt{article} option the toc, lof and lot 
% are printed as sections.
%    \begin{macrocode}
\ifsuftesi@article 
\renewcommand\tableofcontents{%
    \vspace{2ex}%
	    \section*{\contentsname}%
	      \@mkboth{\contentsname}{\contentsname}%
        \thispagestyle{empty}
        \ifsuftesi@twocolumntoc
          \begin{multicols}{2}
	          \@starttoc{toc}%
	        \end{multicols}
	      \else
	         \@starttoc{toc}%
	      \fi
    \vspace{2ex}%
   }
\renewcommand\listoffigures{%
    \vspace{2ex}%
    \section*{\listfigurename}%
       \@mkboth{\listfigurename}{\listfigurename}%
     \thispagestyle{empty}
        \ifsuftesi@twocolumnlof
          \begin{multicols}{2}
	          \@starttoc{lof}%
	        \end{multicols}
	      \else
	         \@starttoc{lof}%
	      \fi
    \vspace{2ex}%
    }
\renewcommand\listoftables{%
    \vspace{2ex}%
    \section*{\listtablename}%
        \@mkboth{\listtablename}{\listtablename}%
      \thispagestyle{empty}
        \ifsuftesi@twocolumnlot
          \begin{multicols}{2}
	          \@starttoc{lot}%
	        \end{multicols}
	      \else
	         \@starttoc{lot}%
	      \fi
    \vspace{2ex}%
    }
\else
%    \end{macrocode}
% The default toc, lof and lot are treated as chapters. 
%    \begin{macrocode}
\renewcommand\tableofcontents{%
    \if@twocolumn
      \@restonecoltrue\onecolumn
    \else
      \@restonecolfalse
    \fi
    \chapter*{\contentsname}%
        \@mkboth{%
            \contentsname}
           {\contentsname}%
      \thispagestyle{empty}
        \ifsuftesi@twocolumntoc
          \begin{multicols}{2}
	          \@starttoc{toc}%
	        \end{multicols}
	      \else
	         \@starttoc{toc}%
	      \fi
    \if@restonecol\twocolumn\fi
   }
\renewcommand\listoffigures{%
    \if@twocolumn
      \@restonecoltrue\onecolumn
    \else
      \@restonecolfalse
    \fi
    \chapter*{\listfigurename}%
      \@mkboth{\listfigurename}%
              {\listfigurename}%
     \thispagestyle{empty}
        \ifsuftesi@twocolumnlof
          \begin{multicols}{2}
	          \@starttoc{lof}%
	        \end{multicols}
	      \else
	         \@starttoc{lof}%
	      \fi
    \if@restonecol\twocolumn\fi
    }
\renewcommand\listoftables{%
    \if@twocolumn
      \@restonecoltrue\onecolumn
    \else
      \@restonecolfalse
    \fi
    \chapter*{\listtablename}%
      \@mkboth{%
          \listtablename}%
         {\listtablename}%
     \thispagestyle{empty}
        \ifsuftesi@twocolumnlot
          \begin{multicols}{2}
	          \@starttoc{lot}%
	        \end{multicols}
	      \else
	         \@starttoc{lot}%
	      \fi
    \if@restonecol\twocolumn\fi
    }
\fi
%    \end{macrocode}
% \subsection*{New commands}
%    \begin{macrocode}
\ifsuftesi@article
\def\chapterintro{%
\ClassError{suftesi}
  {Command \noexpand\chapterintro is undefined\MessageBreak 
     using 'structure=article'}
  {Command \noexpand\chapterintro is undefined\MessageBreak 
     using 'structure=article'}}
\else
\def\chapterintro{\@ifstar{%
  \@tempswafalse\@chapterintro}{\@tempswatrue\@chapterintro}}
\def\@chapterintro{\phantomsection
  \if@tempswa\section*{\SUF@fchapterintroname}\fi
  \addcontentsline{toc}{section}{\SUF@fchapterintroname}}
\newcommand{\chapterintroname}[1]{\def\SUF@fchapterintroname{#1}}
\chapterintroname{Introduzione}
\fi
%    \end{macrocode}
% \paragraph{Manual breaks}
% Active in the table of contents but not in the text.
%    \begin{macrocode}
\newif\ifheadbreak\headbreakfalse
  \DeclareRobustCommand{\headbreak}  
    {\ifheadbreak\\\else\fi}
%    \end{macrocode}
% Active in the text but not in the table of contents.
%    \begin{macrocode}
\newif\ifxheadbreak\xheadbreaktrue
  \def\xheadbreakNL{\ifxheadbreak\newline\else\fi}
  \def\xheadbreakBB{\ifxheadbreak\\\else\fi}
%    \end{macrocode}
%    \begin{macrocode}
\let\origtableofcontents\tableofcontents
  \renewcommand{\tableofcontents}{%
    \begingroup\headbreaktrue\xheadbreakfalse%
  \origtableofcontents\endgroup}
%    \end{macrocode}
% \paragraph{Backward compatibility}
% An environment to manually typeset the bibliography. (Use \sty{biblatex} instead!)
%    \begin{macrocode}
\newenvironment{bibliografia}{%
  \ifsuftesi@article
    \section*{\refname}
     \addcontentsline{toc}{section}{\refname}
  \else
  \chapter{\bibname}%
  \fi
      \normalfont \list{}{%
      \setlength{\itemindent}{-\parindent}
      \setlength{\leftmargin}{\parindent}
      \setlength{\labelwidth}{0pt}
      \setlength{\parsep}{\parskip}
      \let\makelabel}}
  {\endlist}
%    \end{macrocode}
% An environment to manually typeset the list of shorthands. (Use \sty{biblatex} instead!)
%    \begin{macrocode}
\newcommand{\losname}{Sigle}
\newcommand{\itlabel}[1]{\itshape\hbox to 6em{#1}}
\newenvironment{sigle}{%
  \chapter{\losname}
    \normalfont \list{}{%
      \setlength{\labelsep}{0.5em}
      \setlength{\itemindent}{0pt}
      \setlength{\leftmargin}{6em}
      \setlength{\labelwidth}{\leftmargin}
      \setlength{\listparindent}{\parindent}
      \setlength{\parsep}{\parskip}
      \let\makelabel\itlabel}}
  {\endlist}
%    \end{macrocode}
%
% \subsection*{Final settings}
% Renew  |\frontmatter| to have arabic page numbering:
%
%    \begin{macrocode}
\ifsuftesi@article
\renewcommand\frontmatter{\ClassError{suftesi}
  {Command \noexpand\frontmatter is undefined\MessageBreak 
     using 'structure=article'}
  {Command \noexpand\frontmatter is undefined\MessageBreak 
     using 'structure=article'}}
\renewcommand\mainmatter{\ClassError{suftesi}
  {Command \noexpand\mainmatter is undefined\MessageBreak 
     using 'structure=article'}
  {Command \noexpand\mainmatter is undefined\MessageBreak 
     using 'structure=article'}}
\renewcommand\backmatter{\ClassError{suftesi}
  {Command \noexpand\backmatter is undefined\MessageBreak 
     using 'structure=article'}
  {Command \noexpand\backmatter is undefined\MessageBreak 
     using 'structure=article'}}
\else
\renewcommand\frontmatter{\cleardoublepage\@mainmatterfalse} 
\renewcommand\mainmatter{\cleardoublepage\@mainmattertrue} 
\fi
%    \end{macrocode}
% In a previous version the \cmd{hemph} command was provided
% to fix a bug in the hyphenation of some italian expressions
% like ``dell'\emph{encyclopaedia}''. Now the bug has been fixed by the \sty{fixltxhyph} package by Claudio Beccari. The package 
% must be loaded after \sty{babel} or \sty{polyglossia}:
%    \begin{macrocode}
\@ifpackageloaded{babel}
  {\AtBeginDocument{\RequirePackage{fixltxhyph}}}{}
\@ifpackageloaded{polyglossia}
  {\AtBeginDocument{\RequirePackage{fixltxhyph}}}{}
%    \end{macrocode}
% The \cmd{hemph} command is provided only for 
% backward compatibility:
%    \begin{macrocode}
\let\hemph\emph
%    \end{macrocode}
% The first line of all sections is indented by default
% using \XeLaTeX{} with Italian as the main language. Anyway this is 
% incongruous with the \LaTeX{} default.
%    \begin{macrocode}
\unless\ifPDFTeX% if xetex or luatex
  \let\@afterindenttrue\@afterindentfalse
    \else
  \relax
\fi
%    \end{macrocode}
% Just one touch of french typography:
%    \begin{macrocode}
\frenchspacing
%    \end{macrocode}
% \iffalse
%</class>
% \fi
%
%
% \iffalse
%<*bib>
@book{Morison:1111,
  Author = {Stanley Morison},
  Booktitle = {First Principles of Typography},
  Date = {1936},
  Location = {Cambridge},
  Origdate = {2008},
  Origlocation = {Pisa-Roma},
  Origpublisher = {Fabrizio Serra editore},
  Origtitle = {I principi fondamentali della tipografia},
  Publisher = {Cambridge University Press},
  Title = {First Principles of Typography}}

@book{Eco:1980,
  Author = {Umberto Eco},
  Booktitle = {Il nome della rosa},
  Location = {Milano},
  Publisher = {Bompiani},
  Title = {Il nome della rosa},
  Year = {1980}}

@online{Gregorio:frontespizio,
  Author = {Enrico Gregorio},
  Note = {version 1.1},
  Title = {Il pacchetto \textsf{frontespizio}},
  Url = {http://www.guit.sssup.it/phpbb/index.php},
  Year = {2009}}

@article{Valbusa:2010,
  Author = {Ivan Valbusa},
  Journal = {ArsTeXnica},
  Month = {10},
  Number = {9},
  Title = {Creare stili bibliografici con \textsf{biblatex}: 
  l'esperienza del pacchetto {biblatex-philosophy}},
  Year = {2010}}

@book{Bringhurst:1992,
  Author = {Robert Bringhurst},
  Booktitle = {The Elements of Typographic Style},
  Date = {1992},
  Edition = {4th ed. (version 4.0)},
  Location = {Vancouver},
  Publisher = {Hurtley \& Marks Publisher},
  Title = {The Elements of Typographic Style},
  related = {Bringhurst:1992-ITA}
}
@book{Bringhurst:1992-ITA,
  Author = {Robert Bringhurst},
  title = {Gli elementi dello stile tipografico},
  Date = {2009},
  Edition = {5},
  Location = {Milano},
  Publisher = {Sylvestre Bonnard}}
@book{Tschichold:1975,
  Author = {Jan Tschichold},
  Booktitle = {Ausgewählte Aufsätze über Fragen der Gestalt des 
  Buches und der Typographie},
  Location = {Basel},
  Origdate = {2003},
  Origlocation = {Milano},
  Origpublisher = {Sylvestre Bonnard},
  Origtitle = {La forma del libro},
  Publisher = {Birkhäuser Verlag},
  Title = {Ausgewählte Aufsätze über Fragen der Gestalt des Buches 
  und der Typographie},
  Year = {1975}}

@online{Lehman:2010,
  Author = {Philipp Lehman},
  Note = {Versione 0.9a},
  Title = {The \textsf{biblatex} package},
  Url = {http://mirrors.ctan.org/macros/latex/contrib/biblatex/doc/biblatex.pdf},
  Year = {2010}}

@online{Miede:2011,
  Author = {André Miede},
  Title = {A Classic Thesis style},
  Url = {http://mirrors.ctan.org/macros/latex/contrib/classicthesis/ClassicThesis.pdf},
  Date = {2011}}

@book{Pantieri:2011,
  Author = {Lorenzo Pantieri and Tommaso Gordini},
  Booktitle = {L'arte di scrivere con LATEX},
  Date = {2012},
  Edition = {new edition},
  Foreword = {Enrico Gregorio},
  Hyphenation = {italian},
  Subtitle = {Un'introduzione a \LaTeXe},
  Title = {L'arte di scrivere con \LaTeX},
  Url = {http://www.lorenzopantieri.net/LaTeX_files/ArteLaTeX.pdf},
  Annote = {Questa guida Ë diventata il punto di riferimento degli utenti 
  italiani di \LaTeX\ ed Ë apprezzata per la chiarezza espositiva e la 
  completezza dei contenuti}}

@article{Pantieri:2009,
  Author = {Lorenzo Pantieri},
  Title = {L'arte di gestire la bibliografia con \pack{biblatex}},
  Url = {http://www.lorenzopantieri.net/LaTeX_files/Bibliografia.pdf},
  Year = {2009}}

@article{Mori:2008,
  Author = {Lapo F. Mori},
  Journal = {ArsTeXnica},
  Month = {10},
  Pages = {37-51},
  Title = {Gestire la bibliografia con {\LaTeX}},
  Volume = {6},
  Year = {2008}}


@article{valbusa:20122,
	Author = {Ivan Valbusa},
	Date = {2012},
	Date-Added = {2013-08-23 10:45:16 +0200},
	Date-Modified = {2013-08-23 10:48:03 +0200},
	Journaltitle = {\Ars},
	Month = {10},
	Number = {14},
	Pages = {15-30},
	Subtitle = {La classe suftesi},
	Title = {La forma del testo umanistico},
	url={http://www.guitex.org/home/images/ArsTeXnica/AT014/valbusa.pdf}}


@book{munari:arte-come-mestiere,
	Author = {Bruno Munari},
	Booktitle = {Arte come mestiere},
	Date-Added = {2016-03-12 16:39:37 +0000},
	Date-Modified = {2016-03-12 16:40:07 +0000},
	Location = {Roma-Bari},
	Publisher = {Laterza},
	Title = {Arte come mestiere},
	Year = {1966}}

%</bib>
% \fi
%
% \Finale
