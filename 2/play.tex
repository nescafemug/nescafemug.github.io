\documentclass{article}
\usepackage[left=1.6cm, right=1.6cm, top=1.6cm, bottom=1.6cm]{geometry}
\usepackage{multicol}
\usepackage{amssymb}
\renewcommand{\labelitemi}{$\checkmark$}
\renewcommand\labelitemii{$\square$}
\begin{document}
\thispagestyle{empty}
\begin{center} \large{\textsc{\texttt{\underline{Mechanical Engineering}}}} \end{center}
\hrule
{\footnotesize
\begin{multicols}{2}
\begin{itemize}
\item \textbf{Engineering Mathematics}
\begin{itemize}
\item \texttt{Linear Algebra:} Matrix algebra, systems of linear equations, eigenvalues and eigenvectors.
\item \texttt{Calculus:} Functions of single variable, limit, continuity and differentiability, mean value theorems, indeterminate forms; evaluation of definite and improper integrals; double and triple integrals; partial derivatives, total derivative, Taylor series (in one and two variables), maxima and minima, Fourier series; gradient, divergence and curl, vector identities, directional derivatives, line, surface and volume integrals, applications of Gauss, Stokes and Green’s theorems.
\item \texttt{Differential equations:} First order equations (linear and nonlinear); higher order linear differential equations with constant coefficients; Euler-Cauchy equation; initial and boundary value problems; Laplace transforms; solutions of heat, wave and Laplace's equations.
\item \texttt{Complex variables:} Analytic functions; Cauchy-Riemann equations; Cauchy’s integral theorem and integral formula; Taylor and Laurent series.
\item \texttt{Probability and Statistics:} Definitions of probability, sampling theorems, conditional probability; mean, median, mode and standard deviation; random variables, binomial, Poisson and normal distributions.
\item \texttt{Numerical Methods:} Numerical solutions of linear and non-linear algebraic equations; integration by trapezoidal and Simpson’s rules; single and multi-step methods for differential equations.
\end{itemize}
\item \textbf{Applied Mechanics and Design}
\begin{itemize}
\item \texttt{Engineering Mechanics:} Free-body diagrams and equilibrium; trusses and frames; virtual work; kinematics and dynamics of particles and of rigid bodies in plane motion; impulse and momentum (linear and angular) and energy formulations, collisions.
\item \texttt{Mechanics of Materials:} Stress and strain, elastic constants, Poisson's ratio; Mohr’s circle for plane stress and plane strain; thin cylinders; shear force and bending moment diagrams; bending and shear stresses; deflection of beams; torsion of circular shafts; Euler’s theory of columns; energy methods; thermal stresses; strain gauges and rosettes; testing of materials with universal testing machine; testing of hardness and impact strength.
\item \texttt{Theory of Machines:} Displacement, velocity and acceleration analysis of plane mechanisms; dynamic analysis of linkages; cams; gears and gear trains; flywheels and governors; balancing of reciprocating and rotating masses; gyroscope.
\item \texttt{Vibrations:} Free and forced vibration of single degree of freedom systems, effect of damping; vibration isolation; resonance; critical speeds of shafts.
\item \texttt{Machine Design:} Design for static and dynamic loading; failure theories; fatigue strength and the S-N diagram; principles of the design of machine elements such as bolted, riveted and welded joints; shafts, gears, rolling and sliding contact bearings, brakes and clutches, springs.
\end{itemize}
\item \textbf{Fluid Mechanics and Thermal Sciences}
\begin{itemize}
\item \texttt{Fluid Mechanics:} Fluid properties; fluid statics, manometry, buoyancy, forces on submerged bodies, stability of floating bodies; control-volume analysis of mass, momentum and energy; fluid acceleration; differential equations of continuity and momentum; Bernoulli’s equation; dimensional analysis; viscous flow of incompressible fluids, boundary layer, elementary turbulent flow, flow through pipes, head losses in pipes, bends and fittings.
\item \texttt{Heat-Transfer:} Modes of heat transfer; one dimensional heat conduction, resistance concept and electrical analogy, heat transfer through fins; unsteady heat conduction, lumped parameter system, Heisler's charts; thermal boundary layer, dimensionless parameters in free and forced convective heat transfer, heat transfer correlations for flow over flat plates and through pipes, effect of turbulence; heat exchanger performance, LMTD and NTU methods; radiative heat transfer, Stefan-Boltzmann law, Wien's displacement law, black and grey surfaces, view factors, radiation network analysis.
\item \texttt{Thermodynamics:} Thermodynamic systems and processes; properties of pure substances, behaviour of ideal and real gases; zeroth and first laws of thermodynamics, calculation of work and heat in various processes; second law of thermodynamics; thermodynamic property charts and tables, availability and irreversibility; thermodynamic relations.
\item \texttt{Applications:} Power Engineering: Air and gas compressors; vapour and gas power cycles, concepts of regeneration and reheat. I.C. Engines: Air-standard Otto, Diesel and dual cycles. Refrigeration and air-conditioning: Vapour and gas refrigeration and heat pump cycles; properties of moist air, psychrometric chart, basic psychrometric processes. Turbomachinery: Impulse and reaction principles, velocity diagrams, Pelton-wheel, Francis and Kaplan turbines.
\end{itemize}
\item \textbf{Materials, Manufacturing and Industrial Engineering}
\begin{itemize}
\item \texttt{Engineering Materials:} Structure and properties of engineering materials, phase diagrams, heat treatment, stress-strain diagrams for engineering materials.
\item \texttt{Casting, Forming and Joining Processes:} Different types of castings, design of patterns, moulds and cores; solidification and cooling; riser and gating design. Plastic deformation and yield criteria; fundamentals of hot and cold working processes; load estimation for bulk (forging, rolling, extrusion, drawing) and sheet (shearing, deep drawing, bending) metal forming processes; principles of powder metallurgy. Principles of welding, brazing, soldering and adhesive bonding.
\item \texttt{Machining and Machine Tool Operations:} Mechanics of machining; basic machine tools; single and multi-point cutting tools, tool geometry and materials, tool life and wear; economics of machining; principles of non-traditional machining processes; principles of work holding, design of jigs and fixtures.
\item \texttt{Metrology and Inspection:} Limits, fits and tolerances; linear and angular measurements; comparators; gauge design; interferometry form and finish measurement; alignment and testing methods; tolerance analysis in manufacturing and assembly.
\item \texttt{Computer Integrated Manufacturing:} Basic concepts of CAD/CAM and their integration tools.
\item \texttt{Production Planning and Control:} Forecasting models, aggregate production planning, scheduling, materials requirement planning.
\item \texttt{Inventory Control:} Deterministic models; safety stock inventory control systems.
\item \texttt{Operations Research:} Linear programming, simplex method, transportation, assignment, network flow models, simple queuing models, PERT and CPM.
\end{itemize}
\end{itemize}
\end{multicols}
}
\end{document}